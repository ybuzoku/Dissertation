\documentclass[11pt]{report}
\usepackage{amssymb,amsmath,amsthm}
\usepackage{amsfonts,amsmath,amssymb,graphicx}
%\usepackage{cmbright} %sans serif font if uncommented
\usepackage[parskip]{bbkproject}
\usepackage{url}
\usepackage{tikz-cd}
\usepackage{graphicx}
\graphicspath{ {./images/} }

%The below 4 lines put chapter starts NOT on a new page
%\usepackage{etoolbox}
%\makeatletter
%\patchcmd{\chapter}{\if@openright\cleardoublepage\else\clearpage\fi}{}{}{}
%\makeatother

% Must be last package
%\usepackage{hyperref}

\newtheorem{thm}{Theorem}[section]
\newtheorem{lemma}[thm]{Lemma}
\newtheorem{prop}[thm]{Proposition}
\newtheorem{cor}[thm]{Corollary}
\newtheorem*{thm*}{Theorem}
\theoremstyle{definition}
\newtheorem{defn}[thm]{Definition}
\newtheorem{example}[thm]{Example}
\newtheorem*{example*}{Example}
\newtheorem*{problem}{Problem}
\newenvironment{sproof}{%
  \renewcommand{\proofname}{Sketch Proof}\proof}{\endproof}


%%%% These lines are only needed for the example text which follows
\DeclareMathOperator{\im}{Im}
\DeclareMathOperator{\Hom}{Hom}


%%%%%%%%%%%%%%%%%%%%%%%%%%%%%%%%%%%%%%%%%%%%%%
%%%%%%%%%%%%%%%%%%%%%%%%%%%%%%%%%%%%%%%%%%%%%%%%
%%%%%%%%%%%%%%%%%%%%%%%%%%%%%%%%%%%%%%%%%%%%%%%%%
%YOU FILL IN THESE AS APPROPRIATE
\subject{Mathematics} %write your degree subject here
\title{The Uniformisation Theorem of Riemann surfaces}
\author{Y. Buzoku}
\supervisor{Dr. B. Fairbairn} %
\submissiondate{$3^{\mathrm{rd}}$ September 2021}%




\begin{document}
\maketitle
\setcounter{tocdepth}{1}  %Show chapter and section headings only!
\tableofcontents

\newpage
\begin{abstract}
This dissertation is an in-depth discussion of Riemann surfaces and their classification by way of the uniformisation theorem. The dissertation is split into two parts; the first being a discussion of Riemann surfaces themselves, some methods of their construction, and interesting properties and symmetries that they exhibit, concluding by putting ourselves into a position where we can present the problem of classifying Riemann surfaces naturally. The standard references for this section are the book by Jones \& Singerman on Complex Functions, \cite{comfun}. The second part focusses on providing a more analytical account of the proof of the classification of Riemann surfaces. It is an analytically heavy treatment, motivated in a conceptually different way via the theory of electrostatics. We classify compact Riemann surfaces in detail and only sketch the case for non-compact, simply connected Riemann surfaces as a large part of the arguments are similar in the two cases. The uniformisation theorem then follows naturally. The standard reference for the second section is the book by Donaldson on Riemann Surfaces, \cite{donaldson}, though we take a somewhat different approach, converging our argument with Donaldson's near the end of the proof. 
\end{abstract}
\declaration % Includes the declaration - don't delete this!

%Intro to Riemann Surfaces, its own chapter. Talk about $\log(z)$ and it as a motivational entry into the subject. Insert image of IBM PC XT 5160 render of im(log(z))
\chapter*{Introduction}
\addcontentsline{toc}{chapter}{Introduction}
Classically, when given a complex function such as $w=z^2$, we have the problem that visualising this function is a non-trivial task as this is not a bijection between the two planes. We are taught to use branch cuts, to consider specific regions of the domain of definition, to avoid these problems. But, with the notion of a Riemann surface, we dont have leave it at that. In the case of the function $w = z^2$, we have the usual two branches for the $w$-plane. The idea of a Riemann surface conceptually comes from ``stitching" together these two branches, to form a single sheet with multiple layers, such that if one were to move around in the $z$-plane, the image in the $w$-plane should also be free to move about. Thus, as expected, a single point $z$ gets mapped to two points $w$, as we can imagine that the two layers of the $w$-plane both cover the point in the $z$-plane. This example along with functions like $w = \log(z)$ are what were studied by Riemann himself, and what motivated his study of what became known as Riemann surfaces. Riemann initially attempted to generalise the notion of analytic continuation, by defining the notion of a \emph{germ of a function} which he then used to define Riemann surfaces. We, however, will approach this topic from a more modern perspective, using modern tools and definitions with the goal of somehow classifying all Riemann surfaces in a meaningful way. This classification that we chase is called the Uniformisation theorem, and was first rigorously proven independently by Poincar\`{e} and Koebe in 1907, though its statement was conjectured much earlier. Before we begin, we wish to remark that the field of Riemann surfaces is the field of some truly remarkable mathematics, for it is one of the rare cases where notions from Algebraic geometry, Complex analysis and Topology and Group theory converge to provide equivalent descriptions of the same object, each in their own language. This is truly the domain of beautiful mathematics and studying this topic over the past year has been a pleasure.

\chapter{Preliminaries}

In this dissertation, we take $\mathbb{F}$ to mean either the field of Real or Complex numbers. We denote open balls of a metric space, $(X,d)$, by $B_r(x) = \{y \in X \vert d(x,y) < r\}$ and closed balls by $\overline{B_r(x)} = \{y \in X \vert d(x,y) \leq r\}$. Given an open set $U$, its closure will be denoted as $\overline{U}$.
\begin{defn}[Region]
  A region is a non-empty, connected, open subset of $X$, where $X$ is a topological space. The closure of a region is called a closed region. In this dissertation, $X$ is taken to be a Riemann surface.
\end{defn}

\begin{defn}[Holomorphic, Meromorphic, Entire and Analytic functions]
  Given a region $U$ in $\mathbb{C}$, a function $f$ is said to be holomorphic if it differentiable at all points $z \in U$. The function $f$ is said to be analytic if it can be expressed as a power series, i.e $f(z) = \sum\limits^{\infty}_{j=0}\lambda_jz^j$ for some $\lambda_j \in \mathbb{C}$. A function whose domain of definition is $\mathbb{C}$ is called entire. It is called meromorphic, if $f(z)$ is undefined at a discrete subset of points $\Delta \subset U$. Meromorphic functions $f(z)$ can be written as the quotient of two entire polynomials $\frac{p(z)}{q(z)}$ with $q(z) \neq 0$. The set of discrete points $\Delta$ corresponds to the zeros of $q(z)$.
\end{defn}
\begin{defn}[Paths and Loop spaces]
  Given a topological space $X$, a path in $X$ is a continuous map $\gamma \colon [0,1] \rightarrow X$, such that $\gamma(0)$ and $\gamma(1)$ are the endpoints of $\gamma$. If a path has starting and endpoint the same, ie $\gamma(0)=\gamma(1) = x$, then it is said to be a loop based at $x$. The set of loops based at $X$ is called the loop space at $x$ and is denoted by $\Omega_xX$.
\end{defn}
\begin{defn}[Fundamental group]
  Given a topological space $X$, the fundamental group $\pi_1(X,x)$ at some point $x \in X$ is defined as the quotient of the loop space $\Omega_xX$ by the relation $\gamma_1 \sim\gamma_2$ if and only if $\gamma_1$ and $\gamma_2$ are homotopic and have the same basepoint. The group operation is concatentation of loops, ie for two loops $\gamma_1,\gamma_2 \in \Omega_xX$, we can concatenate them by writing 
  \begin{align*}
    (\gamma_2\cdot\gamma_1)(t) = \begin{cases}
      \gamma_1(2t), &t \in [0,1/2] \\
      \gamma_1(2t-1), &t \in [1/2,1]
    \end{cases}
  \end{align*}
  Recall that if this fundamental group is trivial for any $x \in X$, then the space $X$ is said to be simply connected.
\end{defn}
\begin{defn}[Covering Spaces and the Universal cover]
  A continuous map $p\colon Y \rightarrow X$ of topological spaces is called a covering map if there is a collection $\mathcal{U}$ of open sets $U \subset X$, called elementary neighbourhoods with the following property: For each $x \in X$, there is a $U\in \mathcal{U}$ which contains $x$ and such that for each $y \in p^{-1}(x)$, there is a continuous map $q:U \rightarrow p^{-1}(U)$ with $p\circ q = Id_U$ and such that $q(U)$ is the path componant of $p^{-1}(U)$ containing y. The space $Y$ is called a cover of $X$. If $Y$ is simply connected, then it is said to be the universal cover of $X$.  A proof of the existence and uniqueness of the universal cover for most topological spaces (subject to certain, technical requirements) can be found in \cite[p.63]{Hatchers}.
\end{defn}
\begin{defn}[Covering Transformation]
  Let $X$ be a topological space and let $p_1:Y_1 \rightarrow X$ and $p_2:Y_2 \rightarrow X$ be two covering spaces of $X$. A continuous map $F:Y_1 \rightarrow Y_2$, such that $p_1 = p_2 \circ F$ is called a covering transformation. If $F$ is a homeomorphism, then it is called a covering isomorphism.
\end{defn}
\begin{defn}[Deck Transformation]
  Given a covering space $p:Y \rightarrow X$ of a topological space $X$,  covering isomorphisms $F:Y \rightarrow Y$ are called covering self-isomorphisms or Deck transformations. Deck transformations of a covering space $(Y,p)$ form a group, called the Deck group of the covering space $(Y,p)$, and is written as $\text{Deck}(Y,p)$.
\end{defn}
\begin{defn}[Normal Cover]
A connected covering space $p:Y \rightarrow X$ is called \emph{normal} or \emph{regular} if the deck group acts transitively on $p^{-1}(x)$ for any point $x \in X$.
\end{defn}

\begin{defn}[Support of a function]
Let $U$ be an open set. Given a function $f: U \rightarrow \mathbb{F}$, its support, $\text{supp}(f) \subset U$, is defined as the closure of the set $\{x \in U \vert f(x)\neq 0\}$. If this set is compact, then $f$ is said to have compact support contained in $U$.
\end{defn}
\begin{defn}[Bump Function]
  Let $U$ be an open subset of $\mathbb{C}$ about the origin. We define, for some $\epsilon > 0$, a bump function $\beta: U \rightarrow \mathbb{R}$ to be a smooth function with support in $U$ defined as,
  \[
    \beta(z)=
    \begin{cases}
      1, &|z| \leq \frac{1}{4}\epsilon \\
      0, &|z| > \frac{3}{4}\epsilon
    \end{cases}
  \]
\end{defn}
\chapter{Riemann Surfaces}\label{RSChapter}
\section{Riemann surfaces and their properties}\label{bdefns}

We begin by defining and constructing some Riemann surfaces.

\begin{defn}[Riemann Surface]\label{rsdefn}
A Hausdorff topological space $X$ is said to be a Riemann Surface if:
\begin{itemize}
\item There exists a collection of open sets $U_{\alpha} \subset X$, where
  $\alpha$ ranges over some index set $\mathcal{A}$ such that $\bigcup\limits_{\alpha \in \mathcal{A}} 
  U_{\alpha}$ cover $X$.
\item There exists for each $\alpha$, a homeomorphism, called a chart map,
  $ \psi_{\alpha}\colon U_{\alpha} \rightarrow \tilde{U}_{\alpha}$, where $\tilde{U}_{\alpha}$ is 
  an open set in $\mathbb{C}$, with the property that for all $\alpha$,
  $\beta$, the composite map $\psi_{\alpha} \circ \psi_{\beta}^{-1}$ is
  holomorphic on its domain of definition, with a holomorphic inverse map, i.e, are conformal. (These composite maps are sometimes called transition maps or transition functions).
\end{itemize}
We call the triple $(\{U_\alpha\},\{\tilde{U}_{\alpha}\},
\{\psi_\alpha\})$ an 
atlas of 
charts for the Riemann Surface $X$, though we also use the common
notation $(U,
\tilde{U}, \psi)$ to denote this, where $U=\{U_\alpha\}$,
$\tilde{U}=\{\tilde{U}
_{\alpha}\}$ and $\psi=\{\psi_\alpha\}$.
\end{defn}
The condition that the transition functions need to be conformal maps is an example of an imposed structure. Such a structure is called a conformal, or Riemann surface structure. If we replace this condition with that of being smooth (infinitely differentiable), then the resultant objects have what is known as a smooth structure. Our requirement that our surfaces have a conformal transition maps is very strong and has some nice properties as we shall shortly see. Unfortunately, in practise, this definition of Riemann surfaces can be quite difficult to work with as infinite collections of open sets can be difficult to manipulate. Luckily, we can construct some very nice, basic examples using this definition to help us gain an understanding of them. Let us begin with some examples.
\begin{example}[The Complex Plane]
  The complex plane $\mathbb{C}$ can be defined as a Riemann surface. To see this, we pick a cover for the complex plane, say $U = \mathbb{C}$, and just identity map each set back into the complex plane. Hence the triple $(U,U,\phi)$ is an atlas for $\mathbb{C}$ where $\phi=Id$.
\end{example}
Similarly, we can also define the upper half plane $\mathbb{H}=\{ z \in \mathbb{C} \colon \Im(z)>0\}$, as we can cover $\mathbb{H}$ with an open cover, and identity map each open set into the upper half plane on $\mathbb{C}$. A perhaps more interesting example of a Riemann surface is that of the Riemann sphere.

\begin{example}[The Riemann sphere]
  The Riemann sphere, commonly denoted as $\widehat{\mathbb{C}}$, can be defined as a topological space $\mathbb{C} \cup \{\infty\}$, where all sequences of complex numbers that do not converge to a value in $\mathbb{C}$ can be said to converge at this additional point, denoted by $\infty$, otherwise called the point at infinity (note this implies our space is sequentially compact and hence compact). Let us construct an atlas of charts for the Riemann sphere as this will show us that the Riemann sphere is indeed a Riemann surface. The following system of constructing charts on a sphere is known as stereographic projection. Let $U=\{U_1,U_2\}$, with $U_1=\{z \in \mathbb{C} : |z| < 2\}$ and $U_2=\{z \in \mathbb{C} : |z| > \frac{1}{2}\}\cup \{\infty\}$. Let $U_1=\tilde{U}_1=\tilde{U}_2$ and we let $\psi_1:U_1 \rightarrow \tilde{U}_1$, be the identity map. However, we let $\psi_2:U_2 \rightarrow \tilde{U}_2$ with $\psi_2(z)=\frac{1}{z}$ and $\psi_2(\infty)=0$. Both of these maps are clearly holomorphic so all that is needed is to consider if the transition maps composed of these two maps are also holomorphic.
  It turns out that both $\psi_1 \circ \psi_2 ^{-1}$ and $\psi_2 \circ \psi_1 ^{-1}$ are the map $z \mapsto \frac{1}{z}$, which is also holomorphic on its domain of definition $\{z\in \mathbb{C} : \frac{1}{2} < |z| < 2\}$ and so we have our atlas for the Riemann sphere proving it is a Riemann surface.
\end{example}

Recall the Riemann mapping theorem \cite[p.221]{ahlfors}, a result that allows us to conformally map open subsets of $\mathbb{C}$ to one another. We specifically care about the mapping between the open disk $\mathbb{D}=\{z \in \mathbb{C} \colon |z| < 1\}$ and $\mathbb{H}$. Both of these are simply connected regions and subsets of $\mathbb{C}$ so we can find a conformal map between them; such a map looks like $f:\mathbb{H} \rightarrow \mathbb{D}$ where $f(z) = \frac{z-i}{z+i}$. This map is conformal as we can write the inverse map as $f^{-1}(w) = i\frac{1+w}{1-w}$.
Since we have that up to conformal equivalence, all simply connected, open subsets of $\mathbb{C}$ are equivalent to $\mathbb{D}$, what is to say that there do not exist more conformal equivalences, perhaps some that equate all the spaces that we have constructed to one another? 
\begin{lemma}
  The Riemann sphere, the complex plane and the upper half plane are \textbf{not} conformally equivalent to one another.
\end{lemma}
\begin{proof}
  We first distinguish $\widehat{\mathbb{C}}$ from $\mathbb{C}$ and $\mathbb{H}$. In the case of the Riemann sphere, we note that it is a compact space, whereas the complex plane and upper half plane are non-compact spaces. Since, conformal equivalence is a continuous bijection and the fact that the image of a compact set is compact, if we were to suppose that we had a conformal equivalence between $\widehat{\mathbb{C}}$ and $X$ (where $X = \mathbb{C}$ or $\mathbb{H}$), we would get that $X$ is compact, which is a contradiction. Therefore, the Riemann sphere is not conformally equivalent to either the complex plane or the upper half plane. 
  Now to distinguish between $\mathbb{C}$ and $\mathbb{H}$. We have that $\mathbb{H}$ is conformally equivalent to $\mathbb{D}$ so we can pose the question for $\mathbb{D}$ instead. In that case, by Liouville's theorem (\cite[p.122]{ahlfors}) we have that since $\mathbb{D}$ is bounded, holomorphic functions on $\mathbb{D}$ are therefore constant functions. Recall that holomorphic functions on $\mathbb{D}$ are those that are $f\colon \mathbb{C} \rightarrow \mathbb{D}$. Therefore, for any such $f$ we have for all $z \in \mathbb{C}$ that $f(z) = k$ for some $k \in \mathbb{D}$. For $f$ to be a conformal map, $f^{-1}$ must also be holomorphic, however, we have that $f^{-1}$ is not even bijective since $f^{-1}(k) = \mathbb{C}$. Hence, no such conformal equivalence exists between $\mathbb{C}$ and $\mathbb{H}$. 
\end{proof}
So we indeed have constructed three unique surfaces up to conformal equivalence, $\widehat{\mathbb{C}}$, $\mathbb{C}$ and $\mathbb{H}$, i.e three distinct Riemann surfaces. We now show a more abstract and special property of Riemann surfaces; that is, that their transition maps have a positive Jacobian. This restricts the types of surfaces which can arise as Riemann surfaces to being only the orientable surfaces. Furthermore, since this implies that Riemann surfaces are a subset of orientable surfaces, any result which holds true for orientable surfaces, will hold true for Riemann surfaces. 
\begin{lemma}[Riemann surfaces are orientable]
  Let $X$ be a Riemann surface. Then, the determinant of the Jacobian of the transition maps all have positive determinant.
\end{lemma}
\begin{proof}
  Pick an arbitrary transition map between arbitrary open sets on $X$, with a non-empty intersection. Let us call this map $f$. By the definition of transition maps, $f$ is holomorphic on its domain of definition. Hence we can write for any point $z=x+iy$ in the intersection, $f(z) = f(x+iy)=u(x,y)+iv(x,y)$ for some real valued functions $u,v$, with $u,v$ satisfying the Cauchy-Riemann equations $\frac{\partial u}{\partial x} = \frac{\partial v}{\partial y}$ and $\frac{\partial u}{\partial y} = -\frac{\partial v}{\partial x}$. 
  Thus, the determinant of the Jacobian $J[f]$ of our transition map $f$ is 
  $\text{det}(J[f])=\begin{vmatrix}
    \frac{\partial u}{\partial x} & \frac{\partial u}{\partial y} \\
    \frac{\partial v}{\partial x} & \frac{\partial v}{\partial y} \\
  \end{vmatrix} = \frac{\partial u}{\partial x}\frac{\partial v}{\partial y} -  \frac{\partial v}{\partial x}\frac{\partial u}{\partial y}= (\frac{\partial u}{\partial x})^2 + (\frac{\partial u}{\partial y})^2 \geq 0$, where the last equality holds by the Cauchy-Riemann equations. This is strictly positive barring case of $f=k$ for some $k\in \mathbb{C}$, which makes the Jacobian determinant zero. However, we discard this case as the inverse map of such an $f$ would not be holomorphic (ie $f^{-1}(k)$ would not be a single point) and thus $f$ cannot be constant. Hence, we have that the Jacobian determinant for any transition map on a Riemann surface is strictly positive.
\end{proof}

So, using atlases we have constructed a few examples of Riemann surfaces, and have even grouped them into three classes up to conformal equivalence.
One may ask the question, are there any more Riemann surfaces that are not conformally equivalent the three that we have constructed? How might they arise? It turns out there are a number of different and interesting methods of constructing Riemann surfaces, and indeed there are infinitely many Riemann surfaces that are not conformally equivalent to the above three. However, as we will see shortly, the above three Riemann surfaces are certainly special in their own right.

\section{Quotients of the complex plane}\label{QuotientSection}
One powerful method of constructing Riemann surfaces, arises from quotients of topological spaces. In some cases, we can endow these quotient spaces with Riemann surface structures. We recall that in topology, we can construct quotient spaces by taking some topological space $X$ and defining an equivalence relation $\sim$ on it, to create a new space $X/\sim$. If this space has a projection map $p \colon X \rightarrow X/\sim$, that maps elements of $X$ to their equivalence classes under $\sim$, then this new space is also a topological space since we can define a topology on this space (called the quotient topology), by taking $U \subset X/\sim$ to be open if and only if $p^{-1}(U)$ is open.
\begin{defn}[Topological group]
  A topological group is a topological space $G$ which is also a group, in which the operations
  \begin{itemize}
    \item $m:G \times G \rightarrow G$, defined as $m(g,h)=gh$
    \item $i:G \rightarrow G$, defined as $i(g)=g^{-1}$
  \end{itemize}
  are continuous.
\end{defn}
We can see easily that $\{\mathbb{C}, + \} \ ,\{GL(2,\mathbb{C}, +)\}$ are a pair of examples of topological groups, where to see that $GL(2,\mathbb{C})$, the matrix group of $2\times 2$ invertible matrices over $\mathbb{C}$, is a topological space we identify it with $\mathbb{C}^4$, where each entry of the matrix has its own basis vector. We also define the notion of a right translation on a topological group.
\begin{defn}[Right Translation]
  Let $G$ be a topological group. Then for some $g \in G$, we have a map $m_g:G \rightarrow G$ such that $m_g(x) = xg$, and this map is a continuous bijection with a continuous inverse $m_{g^{-1}}$. Therefore, these maps, known as right translations, are homeomorphisms, and such the group of homeomorphisms of a topological group $G$ is transitive, i.e. can map any point in $G$ to any other.
\end{defn}
Since we can map any point in $G$ to any other using these right translations, that means that we can map any point to the identity element $e \in G$. This motivates the following definition.
\begin{defn}[Discrete subgroup]
  Let $G$ be a topological group. We call the subgroup $\Omega \subset G$, a discrete subgroup of $G$ if it has the property that there exists a neighbourhood $U$ of $e$ in $G$ such that $U \cap G = \{e\}$.
\end{defn}
\begin{example}[The cylinder]
We can construct an infinitely long cylinder by using the notion of a quotient space from topology and endowing the quotient space with a conformal structure. We consider the (additive) subgroup $2\pi \mathbb{Z}$ in $\mathbb{C}$ and form the quotient space $p \colon \mathbb{C} \rightarrow \mathbb{C}/2\pi \mathbb{Z}$. This space is homeomorphic to the infinite cylinder\footnote{To prove this, take the unit square in $\mathbb{C}$ which is itself homeomorphic to $\mathbb{R}^2$ and consider the identification $[x,0] \sim [x,1]$ for any point $x \in \mathbb{R}$ and show that $\mathbb{C}/\sim$ is homeomorphic to $\mathbb{R}\times S^1$} $\mathbb{R} \times S^1$ . To endow a Riemann Surface structure on this infinite cylinder, we consider a disc $B_{\frac{1}{2}}(z)$ around a point $z \in \mathbb{C}$. If $z_1, z_2 \in B_{\frac{1}{2}}(z)$ and if $z_1 = z_2 + 2\pi n$ for some $n \in \mathbb{Z}$, then $n=0$ and hence $z_1=z_2$. This occurs since the radius of the disc is less than $\pi$. Hence, the projection map $p$ maps $B_{\frac{1}{2}}(z)$ bijectively into $\mathbb{C}/2\pi\mathbb{Z}$, where the equivalence classes are all the points whose modulus is less than half away from $z$, ie collections of points which are of distance less than one away from each other. We can construct an atlas from this data by letting $U = p(B_{\frac{1}{2}}(z))$, $\tilde{U} = B_{\frac{1}{2}}(z)$ and the local inverses of $p$, $q_z$ (which map an element $p(B_{\frac{1}{2}}(z))$ to the centre point of the disc $z$), be the chart maps. We cover $\mathbb{C}/2\pi\mathbb{Z}$ with this data. The transition maps will have the form $(q_z \circ p)(z) = z + 2\pi n$ for some $n \in \mathbb{Z}$. This map is clearly holomorphic with holomorphic inverse and as such we have that the infinite cylinder is a Riemann surface. 

Note further that we can find a conformal equivalence between $\mathbb{C}\setminus \{0\}$ and $\mathbb{C}/2\pi\mathbb{Z}$, by ways of the mapping $z \mapsto e^{iz}$. This map never attains zero, and hence, is holomorphic with holomorphic inverse. Thus the cylinder is conformally equivalent to the punctured plane and so, the punctured plane is also a Riemann surface, obtaining its' conformal structure via the exponential map. 
\end{example}
 
\begin{example}[The complex torus]
We begin by picking two different, non zero, complex numbers $\omega_1, \omega_2$ and define $\Omega = \Omega(\omega_1,\omega_2) = \{n\omega_1 + m\omega_2 \vert n,m \in \mathbb{Z} \}$. This object, $\Omega$, is an example of a lattice in $\mathbb{C}$, an additive subgroup of $\mathbb{C}$. As proven in \cite[p.58]{comfun}, such lattices are isomorphic as addititve groups to $\mathbb{Z} \times \mathbb{Z}$ and so by \cite[Example 1.43]{Hatchers}, this $\mathbb{C}/\Omega$ is topologically a torus. But we want to give this torus a conformal structure. We note that, we can write this lattice as $\Omega(1,\omega)$ where $\omega=\frac{\omega_i}{\omega_j}$ where $i,j = \{1, 2\}$ with $i \neq j$, chosen so that $\Im(\omega) > 0$. We call this number $\omega$ the modulus of the complex torus. Working with this modified lattice, we proceed by picking a point and considering a disc of radius $r$ around $z$, called $B_r(z)$. The radius is yet to be determined, though we need to ensure that, in the same way as before, we can pick points to be ``sufficiently close'' to our chosen point $z$ so that we can define our open sets properly and choose appropriate chart maps. Picking two different points $z_1,z_2 \in B_r(z)$ if $z_1 = z_2 + n + \omega m$ for $n,m \in \mathbb{Z}\setminus \{0\}$ then, if we want $z_1=z_2$ (implying $n=m=0$) then we need that $2r < \min\limits_{n,m}|n + m\omega|$. In fact, since for $m \neq 0$ we have $|n+m\omega| > \Im(\omega)$ and if $m=0$ but $n\neq 0$ then we have $|n + m\omega| \geq 1$, then we can get a better bound on $r$, namely that $2r < \min(1, \Im(\omega))$. Then, as before, $p$ maps $B_r(z)$ bijectively to $\mathbb{C}/\Omega$, so we can construct an atlas for this torus, with charts being $U = p(B_r(z))$, with $\tilde{U} = B_r(z)$ and the chart maps being the local inverses to the covering maps. So we see that, in much the same way as in the case of the cylinder, we can bestow a conformal structure on this torus, hence making the torus a Riemann surface. In particular this is a compact Riemann surface. If we construct a torus in such a way, we call it a complex torus, and represent it by $\mathbb{T}$ or if the number $\omega$ is important $\mathbb{T}_{\omega}$.
\end{example}

In both of these cases, our strategy was to find a discrete subgroup of the additive group $\mathbb{C}$. This method, by which we search for discrete subgroups of the space we are quotienting, is our main tool when it comes to studying quotient spaces of Riemann surfaces. As we will eventually show, all but three Riemann surfaces arise as a quotient of some other Riemann surface, and in all cases, the subgroups which define the Riemann surface are discrete. 

Going back to our complex tori for a moment, we see that in the process of creating our conformal structure on the torus we associated a complex number $\omega$, called the modulus of the lattice $\Omega$, that is an element of the upper half plane. This makes us wonder, does this number hold any special meaning? We have that seemingly unrelated spaces, like $\mathbb{C}/2\pi \mathbb{Z}$ and $\mathbb{C}\setminus \{0\}$ are conformally equivalent, so perhaps complex tori are also conformally equivalent. It turns out that this is not quite the case. Showing this fact, is however a little trickier.  The crucial point is that, if two tori have the same modulus, then they are conformally equivalent by \cite[(p.272, Theorem 6.1.4)]{comfun}. An in depth discussion of this can be found in \cite[(p.272, Section 6.1)]{comfun}. What is striking is that the space of all such moduli itself forms a space called a Teichm\"{u}ller or Moduli space (of genus $1$ surfaces). This Moduli space obtains a Riemann surface structure, where each point of this space is an equivalence class of complex tori with a particular modulus.

So if we can have an infinite number of distinct conformal structures on the complex torus, does the same hold true on the Riemann Sphere? We investigate this problem below.
\section{Maps between Riemann surfaces}
As with any mathematical theory, we want to investigate maps between our objects. Through composing our maps with the charts of the surfaces, we can define well behaved maps. However, globally, these maps may differ somewhat from their local descriptions. 
We begin with a powerful lemma and corollary about holomorphic maps between Riemann surfaces.
\begin{lemma}
  Let $X$ and $Y$ be connected Riemann surfaces and $F:X \rightarrow Y$ a non-constant holomorphic map. For each point $x \in X$, there is a unique integer $k_x \geq 1$ such that we can find charts around $z$ in $X$ and $F(x)$ in $Y$ such that $F(z) = x^{k_x}$.
\end{lemma}
\begin{sproof}
  This lemma gives that, locally, holomorphic maps between Riemann surfaces can be represented by a power map. A proof of this lemma can be found in \cite[(p.43, Proposition 5)]{donaldson}. To sketch the idea of the proof, we see that by letting $(U,\tilde{U},\phi)$ be the chart around $z \in X$, with $\phi(z)=0\in \tilde{U} \subset \mathbb{C}$, which we can choose by just linearly translating the chart maps, and by letting $(V,\tilde{V},\psi)$ be a chart about $F(x)$ such that $\psi(F(x)) = 0 \in \tilde{V} \subset \mathbb{C}$, similarly, gives us that the composite $(\psi \circ F \circ \phi^{-1})(x)=x^{k_x}$, as in the charts, $F$ is represented by holomorphic function $f$ with a power series, in which we find all terms except for one vanish.
\end{sproof}
\begin{cor}
  Let $F:X \rightarrow Y$ be a non-constant holomorphic map between connected Riemann surfaces. Then there is a discrete subset $R \subset X$ of points in $X$ where $k_x > 1$.
\end{cor}
\begin{proof}
  Locally, $F$ is represented by the function $f$. Since $f$ can be locally written as a power map, we see that for all the points where $k_x = 1$ then the derivative of $f$ is constant and cannot be zero. So for points where $k_x > 1$, the derivative of $f$ gives the function $k_xx^{k_x-1}$, so the zeros of the derivative of $f$, which is also a holomorphic function, gives us the points in $X$ which have $k_x > 1$. Since the zeros of a holomorphic function form a discrete set, we have that this set of points is discrete.
\end{proof}

When considering holomorphic maps between Riemann surfaces, frequently such maps are taken to be proper, that is, the preimage of a compact set is compact. Whilst it is not strictly necessary to do so, it is natural to study proper maps between topological spaces that may themselves not be compact, as it allows us to define and study the idea of the "ends" of a topological space\footnote{First introduced by Freudenthal on page 701 of his 1931 paper \"{U}ber die Enden topologischer R\"{a}ume und Gruppen for the Journal Mathemtische Zeitschrift 33, 692-713}. We discuss this in a bit more depth at the end of this dissertation, though for now, we accept that for a richer theory of maps between our topological spaces with Riemann surface structures on them, we wish them to be proper. Thus, we append the requirement that such maps be proper to our standard requirements from complex analysis and Riemann surface theory (such as holomorphicity). 

So what does a map being proper give us? If we let $F:X \rightarrow Y$ be a non-constant, proper, holomorphic map between connected Riemann surfaces and let $R$ represent the set of points in $X$ where $k_x > 1$. Then, rather amazingly we have that the image $\Delta = F(R)$ is discrete in $Y$ and that for any $y \in Y$ the preimage $f^{-1}(y)$ is also a finite subset of $X$. The fact that $\Delta$ is a discrete subset of $Y$, is analogous to our example in the introduction. We recall that under the mapping $w=z^2$, we had that for each value $z$ there were two values $w$. These two values formed ``sheets'' over the $z$ plane. We use this idea to say that $F$ has the property that if it is a proper, non-constant holomorphic map, then it must have only finitely many sheets. Thus we note that the points in $R$ are particularly special when studying $F$ and are called the \emph{critical points of $f$}, with the points in $F(R)$ called the \emph{branch} or \emph{ramification points}. The value $k_x$ is called the \emph{multiplicity} of $x$ under $F$. If, for example, we let $f:\widehat{\mathbb{C}}\rightarrow \widehat{\mathbb{C}}$ such that $f(z) = z + \frac{1}{z}$, then this is clealy a map, where every point has multiplicity $2$, since if $p$ is a particular branch point under $f$, then it is a solution to $p=z + \frac{1}{z}$, or the polynomial $z^2 -pz + 1=0$. Since $p$ is a branch point, then both \emph{branches} of $f$ must converge on $p$ so it is a repeated root. Considering the discriminant of this polynomaial gives $p^2 - 4 = 0$ implying the branch points on $\widehat{\mathbb{C}}$ under the map $f$ are $\{2,\ -2\}$.

We will unfortunately not go into the deep topic of covers of Riemann surfaces with ramification points, though an excellent resource on the topic of ramified covers of the Riemann sphere (and a general theory of branched covers of surfaces), can be found in \cite{algebra}. However, what is important to our discussion is the following.
\begin{defn}
  Given a proper, non-constant holomorphic map between Reiamnn surfaces $F:X \rightarrow Y$, with $Y$ connected, we can define for all $y\in Y$ an integer \[
    d(y) = \sum\limits_{x \in F^{-1}(X)}k_x
  \]
  Since the sum runs over a finite set, by the above lemma, $d(y)$ will be finite.  
\end{defn}
It turns out that this integer $d$ is independent of chosen point $y \in Y$, a proof of which can be found in \cite[(p.44, Proposition 7)]{donaldson}, and thus is an invariant of the map $F$. We call this $d$ the \emph{degree} of the map $F$. The theory thus developed gives us a nice extension of our definition of meromorphic functions on Riemann surfaces.
\begin{defn}[Redefining Meromorphic functions]
  Given a Riemann surface $X$, a meromorphic function on $X$ is said to be a function $F:X\rightarrow \widehat{\mathbb{C}}$, that is not equal to $\infty$ for all $x \in X$. The poles of $F$ are the points $F^{-1}(\infty)$ in $X$, and the order of a pole $x$ is the integer $k_x$. 
\end{defn}
We can therefore say something important for our study of Riemann surfaces.
\begin{prop}
  Let $X$ be a compact. connected Riemann surface. If there is a meromorphic function on $X$ with exactly one pole, and that pole has order $1$, then $X$ is conformally equivalent to the Riemann sphere.
\end{prop}
\begin{proof}
  Let $F:X \rightarrow \widehat{\mathbb{C}}$ be the meromorphic function in the hypothesis. Since $X$ is compact, we have that $F$ is proper, i.e, the preimage of any compact set is compact. The degree of this map is $1$ since the pole is mapped to once. Thus for all $y \in \widehat{\mathbb{C}}$, there exists exactly one $x \in X$. Thus, by the open mapping theorem, $F$ is a bijection. The inverse map of $F$ is continuous since for any closed set $U \subset X$ we have that $F(U)$ is compact in $\widehat{\mathbb{C}}$, and thus closed. Therefore, $F$ is a homeomorphism. Thus, since $F$ has degree one, then in a chart centered on each point $p \in X$ $F^{\prime}(p)\neq 0$, and so by the inverse function theorem $F^{-1}$ is also holomorphic. 
\end{proof}
Hence, we see that any two complex structures on the Riemann sphere are compatible (meaning that we can pass from one to the other conformally) and so there exists exactly one conformal structure on $\widehat{\mathbb{C}}$ up to conformal equivalence.

\section{Covering spaces of Riemann surfaces}
We begin this section with a strong claim: All Riemann surfaces arise as a quotient of a simply connected Riemann surface by some group action. This means that if we can understand simply connected Riemann surfaces then we can perhaps construct all possible Riemann surfaces. In particular, we wish to study how these simply connected surfaces can return to us a general Riemann surface.
\subsection{Automorphisms of Riemann surfaces}

\begin{defn}[Automorphism of a Riemann surface]
  Let $X$ be a Riemann surface. A function $f\colon X \rightarrow X$ is called an automorphism if it is a homeomorphic conformal equivalence.
\end{defn}

\begin{lemma}
  Let $p:Y \rightarrow X$ be a covering of a Riemann surface. Then there is a unique complex structure on $Y$ such that $p:Y \rightarrow X$ is holomorphic.
\end{lemma}
\begin{proof}
  Since $X$ is a Riemann surface, we can choose a cover of $X$ such that the sets $\{U_{\alpha}\}$ are small enough to be mapped to homoeomorphically from sets $\{V_{\alpha}\}$ on $Y$ (this process is described in depth in \cite[(p.171, Theorem 4.11.2)]{comfun}). We let $\{\phi_{\alpha}\}$ be the chart maps for each $U_{\alpha}$ from $X$, and hence we can extend these to define charts from $Y$ as $\psi_{\alpha} = \phi_{\alpha} \circ p$, for each $\alpha$. Hence, we define the atlas for $Y$ to be the triple $(\{V_{\alpha}\},\{\tilde{V}_{\alpha}\},\{\psi_{\alpha}\})$, where $\tilde{V}_{\alpha} = \tilde{U}_{\alpha}$, the open sets in the complex plane from the atlas of $X$. The transition maps of this atlas are conformal, since if we take two sets $V_a$ and $V_b$ such that $V_a \cap V_b \neq \emptyset$, then we have that the transition function is $\psi_{a}\circ\psi_{b}^{-1} = (\phi_{a}\circ p)\circ (\phi_{b} \circ p)^{-1} = \phi_{a}\circ p \circ p^{-1}\circ \phi_{b}=\phi_{a}\circ\phi_b^{-1}$ which is a conformal map since this is a transition function of $X$ itself. Similarly, for any two atlases on $X$, we get that the atlases they induce on $Y$ are compatible, and so we have a well defined conformal structure on $Y$. 

  To show that $p$ is holomorphic, we apply a coordinate transformation to any set on $Y$, say $V\in \{V_{\alpha}\}$. Then, we have that $\psi_{V} \circ \psi_{V}^{-1} = \phi_{U}\circ p\circ p^{-1}\circ \phi_{U}^{-1} = \text{Id}$. So these maps are conformal equivalences, implying that $p$ must be a holomorphic map. Uniqueness of the complex structure follows from the fact that for any two atlases, the projections will map the same points in $Y$ to the same point in $X$.
\end{proof}
\begin{cor}
  Let $p:Y \rightarrow X$ be a covering of a Riemann surface. Then, each deck  transformation $F:Y \rightarrow Y$ is an automorphism of $Y$ as a Riemann surface.
\end{cor}
\begin{proof}
  Let $g \in \text{Deck}(Y,p)$, and points $y_1,y_2 \in Y$ such that $y_2 = g(y_1)$. Pick open sets around each point so $y_1 \in V_1$ with chart map $psi_1$ and $y_2 \in V_2$ with chart map $\psi_2$. Then we get the following transformation of coordinates
  \begin{align*}
    \psi_2\circ g \circ\psi_1^{-1} &= (\phi_2 \circ p) \circ g \circ (p^{-1}\circ \phi_1^{-1})\\
    &= \phi_2 \circ (p \circ g) \circ p^{-1}\circ \phi_1^{-1}\\
    &= \phi_2 \circ \phi_1^{-1} \qquad\text{since }p\circ g = p
  \end{align*}
  However, $\phi_2 \circ \phi_1^{-1}$ is a conformal transformation, therefore $g$ is holomorphic. Replacing $g$ with $h = g^{-1}$ in the above proof gives that $g^{-1}$ is also holomorphic and hence $g$ is a conformal transformation of $Y$. Therefore, it is an automorphism of $Y$.
\end{proof}
These two results give us that the deck group of a cover of a Riemann surface is precisely the group of automorphisms of the cover. This covering space is also a Riemann surface.
\begin{lemma}\cite[(p.71, Proposition 1.39)]{Hatchers}\label{LemmaOnCoveringSurfaces}
  A topological covering space $p:Y \rightarrow X$ is normal if and only if $p_*\pi_1(Y,y)$ is a normal subgroup in $\pi_1(X,x)$ for some $y \in p^{-1}(x)$.
\end{lemma}
This theorem, though not hard to prove is a standard result and so we refer the reader to a proof. Note that the author of \cite{Hatchers}, uses the notation $G(X)$ to mean $\text{Deck}(Y,p)$ for the covering space $p:Y \rightarrow X$. What is important however is the following corollary.
\begin{cor}
  If $X$ is a connected Riemann surface, then the universal covering surface $p:Y \rightarrow X$ is a regular covering surface of $X$.
\end{cor}
\begin{proof}
  Since $Y$ is a universal cover of $X$ that implies that it is simply connected, and hence $\pi_1(Y,y)=1$ for all $y\in Y$. Therefore, it is trivially a normal subgroup in $\pi_1(X,x)$, for all $x \in X$, under the induced map $p_*\colon \pi_1(Y,y) \rightarrow \pi_1(X,x)$. Hence, by lemma \ref{LemmaOnCoveringSurfaces}, we have that the universal covering surface $Y$ of a Riemann surface $X$ is a regular covering of $X$.
\end{proof}
Since we have that every Riemann surface admits a universal cover, and that it is a regular cover, then we would like to link that fact to these groups of covering transformations. In particular, we would like to show that the group of automorphisms of the universal cover of a Riemann surface can be associated with the space it covers.
\begin{prop}
  Let $p:Y \rightarrow X$ be a regular covering of $X$. Let $G = \text{Deck}(Y,p)$ be a of deck transformations of the cover $(Y,p)$. Then there is a homeomorphism $q:X\rightarrow Y/G$ given by $q(x)=[\tilde{x}]_G$ where $x \in X$ and $\tilde{x} \in p^{-1}(x)$, with $[\: \cdot \: ]_G$ meaning orbit under the action of $G$. 
\end{prop}
With this lemma, we can identify all Riemann surfaces with automorphisms of their universal covers, by setting $Y$ to be the universal cover of $X$ and then $G \leq \text{Deck}(Y,p)$, to be some subgroup of the group of covering transformations of the universal cover.
\begin{proof}
  Let $\pi : Y \rightarrow Y/G$ be the projection, sending each point $\tilde{x}$ to its orbit $[\tilde{x}]_G$. Hence, we define $q$ as $q(x) = \pi(\tilde{x})$, where $p(\tilde{x})=x$. We need to show that this only depends of $x$ and not on our choice of $\tilde{x} \in p^{-1}(x)$. 
  
  Let $y_1,y_2 \in p^{-1}(x)$. Since we have that $(Y,p)$ is a regular covering, the deck group acts transitively on $p^{-1}(x)$. Therefore, we have a $g \in G$ such that $g(y_1) = y_2$. Hence we have that $\pi(y_1) = \pi(g(y_1))=\pi(y_2)$ as needed. Hence, $q:X \rightarrow Y/G$ is a well defined, surjective function. 
  
  We want $q$ to be a bijection, so we pick two points $x_1,x_2 \in X$ such that $q(x_1)=q(x_2)$. We want to show they are the same point. Consider that for some $y_1 \in p^{-1}(x_1)$ and $y_2 \in p^{-1}(x_2)$, we have that $\pi(y_1) = \pi(y_2)$ and so hence we have $g \in G$ such that $y_2 = g(y_1)$. Therefore, $x_1 = p(y_1) = p(g(y_1))=p(y_2)=x_2$. Hence we have that $q$ is a bijection. Since $p$ is a continuous map and the fact that $\pi:Y \rightarrow Y/G$ is a continuous map from the definition of the quotient topology on $Y/G$, then so is $q$. Since $q$ is a surjection, $\pi$ is a open map, and hence so are $p$ and $q$. Hence, $q$ is a homeomorphism.
\end{proof}

Hence, we have shown that we can reduce the problem of classifying Riemann surfaces to the problem of studying the automorphisms of the simply connected Riemann surfaces. This is because as we saw, the Deck transformations of a covering space of a Riemann surface are precisely the automorphisms of the cover. These automorphisms and their groups, $Aut(X)$, play a pivotal role in the classification of Riemann surfaces as the define exactly which Riemann surfaces can be constructed from a particular simply connected Riemann surface. Thus we need to know how many simply connected Riemann surfaces there are. It turns out the three spaces we discovered earlier, $\widehat{\mathbb{C}},\, \mathbb{C},\,\mathbb{H}$ are indeed the only simply connected Riemann surfaces up to conformal equivalence. Proving this point is the focus of the next portion of this dissertation. For now, let us conclude by computing the automorphism groups of these three simply connected spaces as this will give us a good idea of which spaces arise as a quotient of which spaces.

\subsection{Automorphism groups of simply-connected Riemann surfaces}
We begin by computing the automorphism group of $\mathbb{C}$, as it is the simplest case.
\begin{thm}\label{Aut(C)} 
  The automorphism group of the complex plane is ~\\$Aut(\mathbb{C})=\{f(z)=az+b \vert a, b \in  \mathbb{C}, a \neq 0\}$
\end{thm}
\begin{proof}
  Let $f(z) = az+b$ such that $a\neq 0$ and $a,b \in \mathbb{C}$. This is clearly a holomorphic function so we just need to find a holomorphic inverse. Let $w = f(z)$. Then we have that if $g(w) = \frac{w - b}{a}$ then $g(f(z))=z$ and as such $f^{-1}(z)=g(z)$. $g$ is clearly also holomorphic since $a \neq 0$ and so linear functions are clearly automorphisms of $\mathbb{C}$. Now we wish to show that they are the only possible maps that give automorphisms of $\mathbb{C}$. \newline
  Since $f(z)$ is an automorphism, it is crucially a holomorphic function and hence admits a series expansion about any point $z \in \mathbb{C}$. So, we write $f(z) = \sum\limits_{j=0}^{\infty}\lambda_jz^j$ for $\lambda_j \in \mathbb{C}$ for all $j$. Writing $g(z)=f(z^{-1})$ gives that $g$ is also holomorphic and injective but only on the anulus $0 < |z| < 1$. Hence, $g$ has a singularity (perhaps many), at $0$, and hence $f$ has one at $\infty$. So we need to classify this singularity, to try and understand $f$. If $f$ had an essential singularity, then by the Great Picard Theorem (\cite[p.300]{conway}), $f$ would fail to be injective, which it is because it is an automorphism. If the singularity were removable, then by Liouville's theorem (\cite[p.122]{ahlfors}), $f$ would be a constant function, which has no holomorphic inverse. Hence, $f$ must have a pole at $\infty$. If for $j\geq 2$ we had that $\lambda_j \neq 0$ then $f$ would fail to be injective on $\mathbb{C}$ as every point would be mapped to by multiple other points. Hence $j \leq 1$ and cannot be constant. Therefore, $f$ has a simple pole at infinity and $f(z) = az + b$ for $a \neq 0$. Hence, all automorphisms of $\mathbb{C}$ have this form.
\end{proof}

\begin{thm}\label{AutSphere}
  The automorphism group of the Riemann sphere is ~\\
  $Aut(\widehat{\mathbb{C}}) = \bigl\{f:\widehat{\mathbb{C}} \rightarrow \widehat{\mathbb{C}} \ \vert \ f(z) = \frac{az+b}{cz+d}, \ ad-bc \neq 0, \enspace a, b, c ,d \in \mathbb{C}\bigr\}$
\end{thm}
\begin{proof}
  If $f(z)=\frac{az+b}{cz+d}$, for complex numbers as in the hypothesis, then $f$ is clearly a bijective, and holomorphic map to the Riemann sphere, as it has a single pole at $z = -\frac{d}{c}$. This map also has a holomorphic inverse to the Riemann sphere, as $f^{-1}(z) = \frac{b-zd}{cz-a}$, which also has a single pole at $z=\frac{a}{c}$. So we have that $\bigl\{f(z) = \frac{az+b}{cz+d}\ \vert \ ad-bc \neq 0, \enspace a, b, c ,d \in \mathbb{C}\bigr\} \subseteq Aut(\widehat{\mathbb{C}})$. For the converse, given an $f \in Aut(\widehat{\mathbb{C}})$, we have that $f$ is a holomorphic bijection with a holomorphic inverse on the Riemann sphere. Such maps, are rational maps, which from complex analysis, we know are polynomials of the form $f(z)=P(z)/Q(z)$ for coprime polynomials $P$ and $Q$. But $f$ and its inverse must both be injective onto the whole Riemann sphere, and as such, $P(z)$ and $Q(z)$ be linear maps. Therefore, we have that $f(z)=\frac{az+b}{cz+d}$, for $a,b,c,d \in \mathbb{C}$, with the condition that $ad-bc \neq 0$ required to ensure that we can find a map $f^{-1}$. Therefore, we have that $Aut(\widehat{\mathbb{C}}) \subseteq \bigl\{f(z) = \frac{az+b}{cz+d}\ \vert \ ad-bc \neq 0, \enspace a, b, c ,d \in \mathbb{C}\bigr\}$ and so they are equal.
\end{proof} 
\begin{cor}
  We can identify automorphisms of the Riemann sphere with a matrix group, called $PSL(2;\mathbb{C}) = \Bigg\{ \begin{pmatrix} a & b\\ c & d \end{pmatrix} \ \Bigg\vert \ ad-bc \neq 0, \enspace a, b, c ,d \in \mathbb{C}\Bigg\}$, the set of invertible $2\times 2$ complex valued matrices. 
\end{cor}
\begin{proof}
  Consider the map $\Phi : Aut(\widehat{\mathbb{C}}) \rightarrow PSL(2;\mathbb{C})$. Let $f \in Aut(\widehat{\mathbb{C}})$, so $f(z) = \frac{az+b}{cz+d}$. We have that \[\Phi\Big(\frac{az+b}{cz+d}\Big) = \begin{pmatrix} a & b\\ c & d \end{pmatrix}\] is a bijective map between the two groups with a bijective inverse. Since $Aut(\widehat{\mathbb{C}})$ is a group under functional composition, we have that $\Phi$ is actually a group isomorphism. As such, $Aut(\widehat{\mathbb{C}})$ and $PSL(2;\mathbb{C})$ are isomorphic as groups and so we can identify automorphisms of the Riemann sphere with elements of $PSL(2;\mathbb{C})$. 
\end{proof}
Recall that elements of discrete subgroups of the automorphism groups of simply connected Riemann surfaces must act transitively for us to be able to create a quotient Riemann surface. However, when we consider an automorphism of the Riemann sphere, we see that every automorphism has at least one fixed point. Thus the only subgroup of $Aut(\widehat{\mathbb{C}})$ which we can quotient $\widehat{\mathbb{C}}$ by is the trivial subgroup. Thus, $\widehat{\mathbb{C}}$ is the universal cover of only itself, and no other Riemann surfaces can arise as a quotient of the Riemann sphere by a discrete subgroup of $Aut(\widehat{\mathbb{C}})$.

When considering the automorphisms of the complex plane, recall that we showed that $Aut(\mathbb{C})= \{f:\mathbb{C}\rightarrow \mathbb{C} \ \vert \ f(z) = az + b, \ a,b\in\mathbb{C}, \ a \neq 0\}$. We see that this has a fixed point for $z = b/(1-a)$, if $a \neq 1$. So our subgroup of automorphisms of $\mathbb{C}$ which act transitively on $\mathbb{C}$ are of the type $f(z) = z + b$. Thus, our automorphisms depend on $b$. But we showed, that if $b$ is a pure real or pure imaginary number, then the action on the plane gives us the infinite cylinder, which can also be conformally mapped to the punctured plane. If $b$ is a complex number, then the action gives us a complex torus. The last case, if $b$ is zero, then we have the identity map, which generates the trivial subgroup of $Aut(\mathbb{C})$, which gives us the complex plane itself. Thus, in our example above, it turns out that we classified all possible Riemann surfaces which have the complex plane as a universal cover!

What remains is to study the quotients of the upper half plane. What is more striking is that, if any other Riemann surfaces exist, all such Riemann surfaces have the upper half plane as a universal cover. So what does the automorphism group of $\mathbb{H}$ look like?
\begin{thm}
  The automorphism group of the upper half plane $\mathbb{H}$ is ~\\ $Aut(\mathbb{H})=PSL(2;\mathbb{R}) = \Bigg\{ \begin{pmatrix} a & b\\ c & d \end{pmatrix} \ \Bigg\vert \ ad-bc = 1, \enspace a, b, c ,d \in \mathbb{\mathbb{R}}\Bigg\}$.
\end{thm} 
We will not prove this fact, though the proof of this can be found in \cite[(p.201, Theorem 4.17.3 (iii))]{comfun}. In fact, by \cite[(p.32, Theorem 2.9.1)]{comfun}, we can identify $PSL(2;\mathbb{R})$ with $\bigl\{f(z) = \frac{az+b}{cz+d}\ \vert \ ad-bc = 1, \enspace a, b, c ,d \in \mathbb{R}\bigr\}$, using the same isomorphism as in the above corollary. So, if we quotient $\mathbb{H}$ by discrete subgroups of $PSL(2;\mathbb{R})$, then we get not just new Riemann surfaces, but in fact, all other Riemann surfaces. The discrete subgroups of $Aut(\mathbb{H})$ are called Fuchsian groups. Let us see some examples.

\begin{example*}[The Modular Group]
  If we consider the modular group $PSL(2;\mathbb{Z})$, this is clearly a subgroup of $PSL(2;\mathbb{R})$. We simply need to show it is a discrete subgroup. We give a topology to $PSL(2;\mathbb{R})$, by associating it with a quotient space of the subspace $X = \{(a, b, c, d) \in \mathbb{R}^4 \ \vert \ ad-bc=1 \} \subset \mathbb{R}^4$ under the homeomorphism $\delta: X \rightarrow X$ where $\delta(a, b, c, d) = (-a, -b, -c, -d)$. The group $PSL(2;\mathbb{R})$, is then given the quotient topology of $X$ under the action  of $\delta$ and the identity map on $X$. Thus, we see that $PSL(2;\mathbb{Z})$ is a lattice of points in $PSL(2;\mathbb{R})$, and thus it is clearly a discrete subgroup of $PSL(2;\mathbb{R})$, and thus, a Fuchsian group.
\end{example*}
\begin{example*}[$PSL(2,7)$]
  We consider $PSL(2,7)$, the (additive) group of $2\times 2$ matrices over $\mathbb{F}_7$. This is clearly a subgroup of $PSL(2;\mathbb{R})$, so we want to show it is also a discrete subgroup. Recall, that since we dont want our discrete subgroups to have any fixed points in $\mathbb{H}$, then the action of the map on $\mathbb{H}$, that is, the automorphism $f(z)=\frac{az+b}{cz+d}$ must have no complex valued solutions $z$ if we set $f(z)=z$, for $\Im(z) > 0$, i.e, we want all solutions to lie on $\mathbb{R} \cup \{\infty\}$. If $c = 0$, then unless $f$ is the identity map, we have no fixed points. So for $c \neq 0$, we get that $cz^2 + (d-a)z - b = 0$ must have non-negative discriminant for there to be a real solution, i.e, $(d-a)^2 - 4bc \geq 0$. But we have that $ad-bc = 1$, which gives that $(d-a)^2 +4ad -4 \geq 0$, which by grouping terms gives $(a + d)^2 \geq 4$. Crucially, this means that if for real roots we want to have $\vert a + d \vert \geq 2$ for some suitable values of $a$ and $d$. Thus when considering $PSL(2,7)$, whose elements are in $\mathbb{F}_7$, if we set $a$ and $d$ equal to $1$ mod $7$, then their sum is $2$ giving that $PSL(2,7)$ is another example of a Fuchsian group.
\end{example*}
In fact, for all primes $p \geq 5$, the above construction holds true with the same argument, and it can also be shown to be true for all prime $p$, that $PSL(2,p)$ is a Fuchsian group. Thus, with such simple examples we seem to have constructed, somewhat abstractly, infinitely many Riemann surfaces! Since they are discrete subgroups of $Aut(\mathbb{H})$, these groups in fact tessellate not only $\mathbb{H}$, but also $\mathbb{D}$, and we can describe these Riemann surfaces visually in terms of these tessellations. Thus we have the amazing fact that every Riemann surface whose universal cover is $\mathbb{H}$, defines a tessellation of $\mathbb{H}$.

\newpage 
\chapter{Calculus on Riemann Surfaces}

\section{Calculus on smooth, orientable surfaces}
We begin our study of calculus on Riemann surfaces by recalling some basic ideas from the theory of calculus on manifolds. This section takes the work found in chapter 5 of \cite{donaldson}, chapters 2, 3, 8, 9 and 10 of \cite{calcohomo} and chapters 4 and 5 of \cite{spivak}, to provide a comprehensive set of consistant definitions and examples that should give the reader a good overview of this broad subject. The reader ideally should have a prior familiarity with at least differential forms and one homology theory, though the above quoted references make for excellent learning resources, especially \cite{calcohomo}. 

In the following definitions $S$ will denote a smooth, orientable surface. Recall that Riemann surfaces are examples of smooth, orientable surfaces. 

\begin{defn}[Smooth paths]\label{Smooth Path}
  Let $\epsilon > 0$. Then we say $\gamma:(-\epsilon,\epsilon) \rightarrow X$ is a smooth path on $S$ if we can define $\frac{d\gamma}{dt}$ for all $t \in (-\epsilon,\epsilon)$.
\end{defn}

\begin{defn}[Tangent Space]\label{TpX}
  Let $p \in S$ be a point on $S$. We define $T_pS$, the tangent space of $S$ at
  $p$, to be the space of equivalence classes of smooth paths $\gamma$ through $p$ such that $\gamma(0)=p$. Two paths $\gamma_1, \gamma_2$ are said to be
  equivalent if $\frac{d\gamma_1}{dt}=\frac{d\gamma_2}{dt}$.
\end{defn}

\begin{defn}[Cotangent Space]\label{T*pX}
  For all $p \in S$ we define the real cotangent space at $p$ to be
  $T^*_pS = \Hom_{\mathbb{R}}(T_pS, \mathbb{R})$ and the complex cotangent
  space at $p$ to be $T^*_pS^{\mathbb{C}} = \Hom_{\mathbb{R}}(T_pS,\mathbb{C})$, where $\Hom_{\mathbb{R}}(T_pS, \mathbb{F})$ is the real vector space of linear maps from the tangent space to the field $\mathbb{F}$.
\end{defn}

We state the following definitions in such a way that will be useful to us, however they all generalise to the case of $n$-dimensional manifolds.
\begin{defn}[Alternating Space]\label{AltSpc}
  Let $V$ be a $\mathbb{F}$ vector space. A $k$-linear map $\omega:V^k\rightarrow \mathbb{R}$ is said to be alternating if $\omega(v_1,\ldots,v_k)=0$ when $v_i=v_j$ where $i\neq j$. The vector space of alternating, $k$-linear maps is denoted by $\text{Alt}^k(V)$.
\end{defn}
\begin{defn}[Exterior Product]
  Let $V$ be a $\mathbb{F}$ vector space. Let $\omega_1 \in \text{Alt}^p(V)$ and $\omega_2 \in \text{Alt}^q(V)$. We define the exterior product $\wedge : \text{Alt}^p(V) \times \text{Alt}^q(V) \rightarrow \text{Alt}^{p+q}(V)$ as 
  \begin{align*}
    &(\omega_1 \wedge \omega_2)(v_1,\ldots,v_{p+q})= \\
    &\sum_{\sigma \in S(p,q)}sign(\sigma)\omega_1(v_{\sigma(1)},\ldots,v_{\sigma(p)})(\omega_2(v_{\sigma(p+1)},\ldots,v_{\sigma(q)})
  \end{align*}
  where $S(p,q)$ is defined as a permutation of $\{1,\ldots,p+q\}$ such that $\sigma(1) < \ldots < \sigma(p)$ and $\sigma(p+1) < \ldots < \sigma(p+q)$.
\end{defn}
The alternating space $\text{Alt}^k(V)$ becomes an alternating $\mathbb{F}$-algebra when considering it with the associative exterior product. Refer to \cite[p.11]{calcohomo} for a more in-depth treatment.

\begin{defn}[Differential $k$-forms]\label{k-form}
  Let $U$ be an open set in $S$. A differential $k$-form on $U$ is a smooth map $\omega : U \rightarrow \text{Alt}^k(S)$. The vector space of all such maps, over the field $\mathbb{F}$ on $U$ is denoted by $\Omega^k(U;\mathbb{F})$. When $\mathbb{F}$ can be any field, then we write $\Omega^k(U)$.
\end{defn}

\begin{defn}[Support of a $k$-form]
  Let $\omega$ be a $k$-form on an open set $U$ in $S$. The support of $\omega$ is defined as $\text{supp}(\omega) = \overline{\{p \in S \vert \omega(p) \neq 0\}} \subset U$.
\end{defn}

\begin{defn}[Differential $k$-forms of compact support]
  Let $U$ be an open set in $S$. A differential $k$-form with compact support is a smooth map $\omega : U \rightarrow \text{Alt}^k(S)$, where $\text{supp}(\omega) \subset U$. The vector space of differential $k$-forms with compact support in $U$, is denoted as $\Omega_c^k(U)$. Note that we can extend $\omega$ onto the whole of $S$ by setting $\omega = 0$ at all points $p \in S \setminus \text{supp}(\omega)$. We call this process, extending over $S$ by zero, or just extending by zero for short.
\end{defn}

  Note that, if our space $S$ is compact, that for all $p\in \{0,1,2\}$ we have that $\Omega_c^p(S) = \Omega^p(S)$. The space of differential $0$-form is also sometimes written as $C^{\infty}(U)$. We also note that if $k > 2$ then $\text{Alt}^k(\mathbb{R}^2)=0$ and as such, in our case, we only need to study $0,1$ and $2$-forms. An important note is the differential $1$-forms evaluated at a point $p$ in an open set $U$ of $\mathbb{R}^2$ can be considered as elements of the cotangent space at $p$ \footnote{Such elements are sometimes called vectors of the cotangent space, or co-vectors when they need to be distinguished from vectors in the tangent space at $p$.}. Similarly, for differential $2$-forms, we can consider such objects as elements of the $1$-dimensional algebra called the external algebra of the cotangent space of $U$ at $p$, represented symbollically as $\Lambda^2T^*_p(U)$. If $T^*_p(U)$ has as basis elements $x_1, x_2$ then the basis element of $\Lambda^2T^*_p(U)$ is $dx_1\wedge dx_2$, with $dx_i$ defined as the exterior derivative of the coordinate functions $x_i$. 
\begin{defn}[Exterior Derivative]\label{exteriorD}
  Let $U$ be an open set in $S$ . We define the exterior derivative of $0$-forms and $1$-forms.
  
  If $f \in \Omega^0(U)$, such that $f=f(x_1,x_2)$ then $df$ is a $1$-form on $U$, with $df = \frac{\partial f}{\partial x_1}dx_1 +  \frac{\partial f}{\partial x_2}dx_2$. This definition is independent of parameterisation, a proof of which can be found at \cite[p.69]{calcohomo}. If $g \in \Omega^0(U)$ then $d(fg) = fdg + (df)g$.

  If $\alpha \in \Omega^1(U)$ such that $f=\alpha_1 dx_1 + \alpha_2 dx_2$, where $\alpha_1, \alpha_2 \in \Omega^0(U)$, then we can define $d\alpha = \left(\frac{\partial \alpha_2}{\partial x_1} - \frac{\partial \alpha_1}{\partial x_2}\right) dx_1\wedge dx_2$. The exterior derivative has some important properties for $1$-forms. Firstly, if $f \in \Omega^0(U)$ then $d(df) = 0$, giving rise to the identity $d^2 = d\circ d = 0$. Secondly, if $f \in \Omega^0(U)$ and $\alpha \in \Omega^1(U)$ then we have $d(f\alpha) = fd\alpha + df \wedge \alpha$. 
\end{defn}

\begin{defn}[Closed and Exact forms]
  Let $S$ be a smooth, orientable surface. A $k$-form $\alpha$, where $k \in \{0,1,2\}$, is called closed if $d\alpha = 0$. A $k$-form $\alpha$ is called exact if it can be written as $d\beta = \alpha$ for some $(k-1)$-form $\beta$.
\end{defn}
We see from the above definitions that every exact form is closed, i.e, on some smooth surface $S$, if we have that $\alpha = d\beta $ then $d\alpha = d(d\beta) = d^2\beta = 0$. The question to then ask is, is every closed form exact? That is the question that cohomology aims to answer.
\begin{defn}[Orientation and Volume forms]\label{volumeform}
  Given a smooth surface $S$, if we have a strictly positive $2$-form, say $\rho \in \Omega^2(S)$ on $S$, then we can take $\rho$ to be an orientation form on $S$. Two orientation forms $\omega, \mu$ are equivalent if they differ by a positive smooth function $f$, i.e, $\omega = f\mu$. If the integral (in the sense of the following definition) of an orientation form $\omega$ is $1$ over the surface, then it is said to be a volume form, or on a surface, an area form.
\end{defn}
Orientation forms are helpful as they allow us to identify smooth functions on $S$ with smooth $2$-forms on $S$. By that, we mean that, $\rho$ forms a basis element for $\Omega^2(S)$ and that we can then write \emph{every} smooth $2$-form, $\omega$ on $S$ as $\omega = f\rho$, for some smooth function $f$ on $S$. This concept is already familiar to us, as when we integrate over regions on $\mathbb{R}^2$, we have that $dx\wedge dy$ is our cartesian volume form (often written as $dxdy$), and so we can integrate smooth functions $f$ by constructing a new $2$-form $fdx\wedge dy$. The main take away though is that these orientation forms are not unique and can be adapted to suit the problem at hand. However, we wish to restrict to studying area forms, rather than general orientation forms as we want to keep the values of integrals the same when changing coordinates. Since they are supposed to represent the same unit area, by changing area form in a problem, we area essentially changing basis of $\Omega^2(S)$ and as such, must adapt the problem via the Jacobian. As we know from integration on $\mathbb{R}^2$, we can go from cartesian coordinates to, for example, polar coordinates. This change might completely change the look of our integrand (the $2$-form) but the end result should be the same, i.e problems are independent of choice of area form. Though we dont prove this, we actually have a stronger result, that is proven in \cite{calcohomo}, namely that studying differential forms is independent of choice of coordinates on $S$. Before we move onto the integral of a differential form, we must first state the following lemma, that defines partitions of unity. Partitions of unity are a powerful tool that allow us to define integration of $2$-forms on orientable surfaces.

\begin{lemma}\cite[(p.64, Lemma 7)]{donaldson}
  Let $K$ be a compact subset of $S$ and let $U_1,\ldots,U_n$ be open sets in $S$ with $K \subset \bigcup\limits_{i=1}^n U_i$. There, there exist non-negative functions $f_1,\ldots,f_n$ on $S$, each with compact support and with $\text{supp}(f_i) \subset U_i$ such that $\sum\limits_{i=1}^n f_i = 1$.
\end{lemma}
The set of functions $\{f_i\}$ are called a \emph{partition of unity subordinate to the cover $\{U_i\}$}.

\begin{defn}[Integration]\label{Integration}
  Integration on a surface can be defined for $1$-forms and $2$-forms. Given a smooth path living in a single chart on $S$, $\gamma \colon [0,1] \rightarrow S$, we can define the integral of a $1$-form $\alpha = \alpha_1 dx_1 + \alpha_2 dx_2$ along $\gamma$, by 
  \[\int_{\gamma}\alpha = \int_0^1\alpha_1\frac{d\gamma_1}{dt}+\alpha_2\frac{d\gamma_2}{dt}\] where $\gamma_1(t),\gamma_2(t)$ are the $x_1, x_2$ coordinate functions of the smooth path $\gamma$. If the path crosses over multiple charts, then our technique is to break up the interval $[0,1]$ into subintervals corresponding to different charts and to sum the integrals over each subinterval to evaluate the integral. Furthermore, if we have some smooth function $\psi \colon [0,1] \rightarrow [0,1]$ such that $\psi(0)=\gamma(0)$ and $\psi(1)=\gamma(1)$, then the integral over the new path $\gamma \circ \psi$ remains the same as the integral over $\gamma$. This means that we can deform our paths to make then easier to integrate over, so long as this deformation keeps the endpoints intact.

  When integrating $2$-forms, in much the same way as for $1$-forms, we have to be careful about whether the support of the $2$-form we integrate goes over many charts or not. Suppose first that a $2$-form, $\rho$, has its compact support contained within a single chart $U$. Then, using the coordinates of that chart, we can write, for some $f\in \Omega_c^0(U)$ that $\rho=f(x_1,x_2)dx_1\wedge dx_2$ and we can integrate this $2$-form as 
  \[\int_U \rho = \int_U f(x_1,x_2)dx_1\wedge dx_2\]
  
  Let us now suppose that $\rho$ has compact support on a set $K$, that crosses finitely many sets coordinate charts on $S$ (this can be assumed as the set of compact support is compact and is hence finitely covered). The above lemma gives us the existence of a partition of unity subordinate to this finite cover. Writing these functions as $\{\chi_i\}$, we have that $\sum\limits_{i = 0}^n \chi_i = 1$ and that for each chart $U_i$, $\chi_i$ is compactly supported within it. Hence, we define the integral $\rho$ over $K$ to be $\int_K \rho = \sum\limits_{i=1}^n\int_{U_i}\chi_i\rho$ which must be finite as $\chi_i$ has compact support within each $U_i$ and $\rho$ also has compact support within $K$. Since, in both cases, we can extend $\rho$ by 0, outside of its support on $S$, then we can write this as 
  \[\int_S \rho = \sum\limits_{i_0}^n\int_S\chi_i\rho\]
  and as such, $\rho = \sum\limits_{i_0}^n\chi_i\rho$. Whilst we do not show this here, the linearity of the integral gives that this definition of integration holds true for any arbitrary partition of unity.
\end{defn}

Whilst from the above definition, it might not be clear how to integrate expressions of the type $\int_U f(x_1,x_2)dx_1\wedge dx_2$, we can do so by alleviate this by writing it as 
\[\int_U f(x_1,x_2)dx_1\wedge dx_2 = \int_U f(x_1,x_2)dx_1dx_2\]
A minor, but important point to make is that this equality arises when given an area form on either an open set, or even on the whole of $S$. In these cases, we can define a Lebesgue integral on our surface, which coincides with the usual notion of a surface integral in $\mathbb{R}^2$. We will not go into details here, though more information about the connection between measures and volume forms can be found under \cite[(Volume measure)]{volformiste}. Defining the Lebesgue integral on a surface properly requires careful detail that we do not have the space to go into here and as such it shall be avoided. However, whenever mention of a measure is hence made, it will refer to this equality between a volume form and the Lebesgue measure, unless explicitly stated otherwise. The Lebesgue measure is frequently written in literature as $d\mu$ or $d\mu_x$ where the $x$ represents the variable of integration. However, to keep in line with classical vector calculus, in this dissertation, whenever we make use of the Lebesgue measure, we will denote it as $dS$ or $dS_x$, to represent the unit surface element. The reader is invited to read \cite[Chapter 11]{babyRudin} for a more depth introduction to measure theory.

Our final, important result is that of Stokes theorem, a proof of which can be found in \cite[(p.124, Theorem 5-5)]{spivak}.
\begin{thm}[Stokes' Theorem]
  If $\alpha$ is a compactly supported $1$-form on a smooth, orientable surface $S$, then for any bounded set $U$ in $S$, we have that 
  \[\int_{\partial U} \alpha = \int_U d\alpha \]
  where $\partial U$ denotes the boundary of $U$.
\end{thm} 
In particular, if $U$ has no boundary, then we have that $\int_U d\alpha = 0$. Taking $U$ to be the whole surface, we get that if our surface has no boundary, then $\int_S d\alpha = 0$. This means that any exact form on a smooth, orientable surface without boundary has integral $0$ over the whole surface. Note that our definition of Riemann surface does not account for the notion of Riemann surface with boundary, so this case holds for us in most cases.

\section{Complex structures}
Below, let $X$ denote a general Riemann surface. Recall that the definition of the complex cotangent space $T^*_pX^{\mathbb{C}}$ was simply the vector space of linear maps from $T_pX$ to $\mathbb{C}$. This gives us that the derivatives of any complex valued function on $X$ is an element of this space. So what do elements of this space look like locally? 

\begin{defn}[Complex structure]\label{Cstructure}
  Given a vector space $V$ over $\mathbb{R}$, we can define a complex structure $J$ on $V$, by defining $J : V \rightarrow V$ such that $J^2 = -1$ to be an $\mathbb{R}$-linear map on $V$.
\end{defn}
\begin{defn}[Complex linear and anti-linear maps]
  Given a vector space $V$ over $\mathbb{R}$, with a complex structure $J$, a linear map is said to be complex linear if for any linear map $A: V \rightarrow \mathbb{C}$ we have that $A(Jv)=iA(v)$ and complex anti-linear if $A(Jv) = -iA(v)$, for all $v \in V$.
\end{defn}
Using these definitions, we state the following lemma.
\begin{lemma}
  Any $\mathbb{R}$-linear map from a vector space $V$ to $\mathbb{C}$, can be written in a unique way as a sum of complex linear and antilinear maps.
\end{lemma}
The proof of this follows from the fact that we can decompose such a map into two, such that, $A = A^{\prime} + A^{\prime\prime}$, where $A^{\prime}(v) = \frac{1}{2}(A(v)-iA(Jv))$ and $A^{\prime\prime} = \frac{1}{2}(A(v)+iA(Jv))$. Considering this in the context of our complex cotanagent space, if we let $A$ be the derivative - a linear operator - then we have a complex structure on $T_p^*X$, where the derivative of holomorphic functions in a neighbourhood of $p$ is complex linear. This therefore allows us to decompose the complex cotangent space into linear and antilinear spaces, i.e, $T_p^*X^{\mathbb{C}} = T_p^*X^{\prime} \oplus T_p^*X^{\prime\prime}$, where if we have a holomorphic function $f$ in a neighbourhood of $p$ on $X$, then $\frac{df}{dz}$ lives in $T_p^*X^{\prime}$ and $\frac{df}{d\overline{z}}$ lives in $T_p^*X^{\prime\prime}$. This decomposition holds for $1$-forms also, and thus we can split complex valued $1$-forms into linear and antilinear parts, such that $\Omega^1(X;\mathbb{C}) = \Omega^{1,0}(X) \oplus \Omega^{0,1}(X)$. Given a at $p \in X$, elements of $\Omega^{1,0}(X)$ evaluated at $p$ lie in $T_p^*X^{\prime}$ and similarly $\Omega^{0,1}(X)$ evaluated at $p$, lie in $T_p^*X^{\prime\prime}$. This corresponds to the compex conjugates. This splitting, allows us to define the Dolbeault complexes on our Riemann surface.
\begin{defn}[Dolbeault complexes]
  Given a complex coordinate on $X$, say $z=x+iy$ where $x,y$ are real coordinates, then, we have that $dz = dx+idy$. Similarly, if we consider the conjugate coordinate then we get $d\overline{z} = dx - idy$. The form $dz$ gives a basis of $\Omega^{1,0}(X)$, a real vector space of forms of the form $\alpha = f(z)dz$, where $f \in \Omega^0(X)$, a real, smooth function. These forms are called $(1,0)$-forms. Similarly, $d\overline{z}$ gives a basis of $\Omega^{0,1}(X)$, with elements $\beta = g(\overline{z})d\overline{z}$, where $g \in \Omega^0(X)$, a real, smooth function. These forms are called $(0,1)$-forms.
\end{defn}

Given a function, $f \in \Omega^0(X)$, we can take its exterior derivative. This yields the usual $df = \frac{\partial f}{\partial x}dx + \frac{\partial f}{\partial y}dy$. However, now given the definitions of $dz$ and $d\overline{z}$, we have that we can rewrite $df$ as 
\[ 
  df = \frac{1}{2}\Bigl(\frac{\partial f}{\partial x} - i\frac{\partial f}{\partial y}\Bigr)dz + \frac{1}{2}\Bigl(\frac{\partial f}{\partial x} + i\frac{\partial f}{\partial y}\Bigr)d\overline{z} 
\]
where we have  $dx = \frac{1}{2}(dz+d\overline{z})$ and $dy = \frac{1}{2i}(dz - d\overline{z})$. Thus we define two new derivative operators on our compelx valued functions, $(1,0)$ and $(0,1)$-forms:
\begin{defn}[Dolbeault differentials]
  We define 
  \begin{align*} 
    \partial f = \frac{\partial f}{\partial z}dz, 
    \qquad 
    \overline{\partial}f = \frac{\partial f}{\partial \overline{z}}d\overline{z}
  \end{align*}
  where the derivatives are defined as \[
  \frac{\partial f}{\partial z} = \frac{1}{2}\Bigl(\frac{\partial f}{\partial x} - i\frac{\partial f}{\partial y}\Bigr), 
  \qquad 
  \frac{\partial f}{\partial \overline{z}} = \frac{1}{2}\Bigl(\frac{\partial f}{\partial x} + i\frac{\partial f}{\partial y}\Bigr)\]
\end{defn}

From this, we can see that the exterior derivative $d$ has itself been decomposed and can be thus written as $d = \partial + \overline{\partial}$. If we have a holomorphic function $f$ then $df = \partial f + \overline{\partial} f$. However, the homorphicity of $f$ gives that the second term vanishes, giving us a criterion for holomorphic functions; namely that $df = \partial f$.

The Dolbeault differentials are not necessarily restricted to just functions, but can also be applied to $(1,0)$ and $(0,1)$-forms. We have that $(1,0)$-form $\alpha = f(z)dz$ that, $\partial\alpha = \frac{\partial f}{\partial z}dz \wedge dz = 0$ and $\overline{\partial}\alpha = \frac{\partial f}{\partial z}d\overline{z} \wedge dz$. Similarly, for the $(0,1)$-form $\beta = g(\overline{z})d\overline{z}$ we have that $\partial\beta = \frac{\partial g}{\partial z}dz \wedge d\overline{z}$ and $\overline{\partial}\beta = \frac{\partial g}{\partial \overline{z}}d\overline{z}\wedge d\overline{z}=0$, thus giving us the following diagram of derivatives
\[
  \begin{tikzcd}
    \Omega^{0,1}(X) \arrow{r}{\partial} & \Omega^2(X) \\
    \Omega^0(X) \arrow{r}{\partial}  \arrow{u}{\overline{\partial}} & \Omega^{1,0}(X) \arrow{u}{\overline{\partial}}
  \end{tikzcd}  
\]

Thus we conclude with the following definition.
\begin{defn}[Holomorphic and Meromorphic $1$-forms]
  Given $\alpha \in \Omega^{1,0}(X)$, it is said to be a holomorphic $1$-form if $\overline{\partial} \alpha = 0$. It is said to be a meromorphic $1$-form if it is a holomorphic $1$-form on $X\setminus \Delta$, where $\Delta$ is a discrete subset of $X$ around which, $\alpha$ can be written as $f(z)dz$ and $f(z)$ is a meromorphic function at those points.
\end{defn}

\section{Cohomology of surfaces}
We conclude this section on calculus on Riemann surfaces by briefly introducing the de-Rham cohomology and the Dolbeault cohomology groups. We wish to use results from the theory of both of them, though how they tie together, will be the subject of the next section.
\begin{defn}[de-Rham Cohomology]\label{deRham}
  Let $S$ be a smooth surface. We define the (additive) de-Rham cohomology groups $H^i(S)$ for $i=\{0,1,2\}$ to be the cohomology of the sequence $\Omega^0(S)\rightarrow \Omega^1(S)\rightarrow \Omega^2(S)$ with the arrows representing the exterior derivatives. Each cohomology group is defined as follows,
  \begin{itemize}
    \item $H^0(S)=\ker(d:\Omega^0(S)\rightarrow\Omega^1(S))$
    \item $H^1(S)=\ker(d:\Omega^1(S)\rightarrow\Omega^2(S))/\im(d:\Omega^0(S)\rightarrow \Omega^1(S))$
    \item $H^2(S)=\Omega^2(S)/\im(d:\Omega^1(S)\rightarrow \Omega^2(S))$
  \end{itemize}
\end{defn}

\begin{example*}
  If our underlying surface is $S=\mathbb{C}=\mathbb{R}^2$, we see that the functions that live in the kernel of $d:\Omega^0(S)\rightarrow\Omega^1(S)$ are the constant functions, and only the constant functions. Hence, $\dim(H^0(S)) = 1$ and we say that $H^0(S)=\mathbb{R}$, as this group is generated by the constant functions. Furthermore, on the plane, we can show that all closed forms are exact\footnote{Recall this means that given a $k$-form, $\alpha$, if $d\alpha = 0$ then there exists a $k-1$-form $\beta$ such that $d\beta = \alpha$} using a generalisation of the fundamental theorem of calculus, implying that $H^1(S)=H^2(S) = 0$. This result generalises to if we have any star-like open set with a single connected componant. If we have $n$ disconnected star like open sets $U$, then $H^0(U)=\mathbb{R}^n$.
\end{example*}
If $S$ is any surface which is 
We can also define the de-Rham cohomology with compact support, $H^i_c(X)$, by swapping $\Omega^i(S)$ with $\Omega_c^i(S)$, in the definitions of the cohomology groups. This notion is useful on surfaces that are not compact, as $H^i_c(S) = H^i(S)$ if $S$ is compact.
\begin{example*}
  Considering the example of the plane again, this time, we must be a bit more careful as the only constant function with compact support in $\Omega^0_c(S)$ is the zero function, therefore, $H^0_c(S)=0$. We also have that $H^1_c(S)=0$. However, since the plane is orientable, we do have the notion of integration and integrable $2$-forms in $\Omega^2_c(S)$ and as such we can extend the integration map $\int_S : \Omega^2_c(S) \rightarrow \mathbb{R}$ to a be a map in cohomology, since any element $[\theta] \in H^2_c(S)$, can be written as $\theta + d\phi$, for some compactly supported $1$-form $\phi$ and by Stokes theorem, the integral of $d\phi$ vanishes so
  \[\int_S:H^2_c(S)\rightarrow \mathbb{R}\]
  can be computed by $\int_S [\theta] = \int_S \theta$. This map defines a (group) isomorphism, therefore we have that $H^2_c(S)=\mathbb{R}$. A proof of these results can be found in \cite[(p.91, Theorem 10.13)]{calcohomo}.
\end{example*}
Note that there is a symmetry in the cohomology groups of the plane when considering cohomology with and without compact support. This is an example of Poincar\'{e} duality, a powerful tool which helps describe the relation cohomology groups for surfaces, be they compact or not. An excellent resource on Poincar\'{e} duality can be found at \cite[Chapter 13]{calcohomo}. The theorem states that given a connected orientable surface $S$, then $H^p(S) \cong H^{2-p}_c(S)^*$, where the $H^{2-p}_c(S)^*$ is the dual space of $H^{2-p}_c(S)$, for $p \in \{0,1,2\}$.
Crucially though, for compact, connected Riemann surfaces, Poincar\'{e} duality gives that $H^1(S)$ must be even dimensional, as it allows us to define a bilinear form $H^1(S)\times H^1_c(S) \rightarrow \mathbb{R}$, which allows us to make a dimensional argument. A deeper expos\'{e} into this can be found at \cite[p.130]{calcohomo}.

We state now some results now that will be useful for later. Let $S$ be a connected, path-connected orientable surface and $\Sigma$ be a connected, compact, orientable surface.
\begin{enumerate}
  \item $\int_{\Sigma} :H^2(\Sigma) \rightarrow \mathbb{R}$ is an isomorphism. \cite[(p.91, Corollary 10.14)]{calcohomo}
  \item $H^0(\Sigma)\cong H^2(\Sigma)$, by Poincar\'{e} duality.
  \item If $\Sigma$ has genus $g \geq 0$, then $H^1(\Sigma) \cong \mathbb{R}^{2g}$ by \cite[p.69]{donaldson}.
  \item If $S$ is simply connected, then $H^1(\Sigma) = 0$. In fact, as we see in \cite[(p.166, Theorem 2A.1)]{Hatchers}, we have that if our surface $S$ is path connected, then $H^1(S)$ is isomorphic to the abelianisation of the fundamental group $\pi_1^{ab}(S) = \pi_1(S)/[\pi_1(S),\pi_1(S)]$ where $[\pi_1(S),\pi_1(S)]$ is the subgroup of elements of the form $aba^{-1}b^{-1}$ for $a,b \in \pi_1(S)$. 
\end{enumerate}
Therefore, we see that the Riemann sphere has de-Rham cohomology
\[
  H^0(\widehat{\mathbb{C}}) = \mathbb{R} \qquad 
  H^1(\widehat{\mathbb{C}}) = 0 \qquad 
  H^2(\widehat{\mathbb{C}}) = \mathbb{R}
\]
 The complex torus of modulus $\omega$, $\mathbb{T_\omega}$ similarly has de-Rham cohmology
\[
   H^0(\mathbb{T_\omega}) = \mathbb{R} \qquad
   H^1(\mathbb{T_\omega}) = \mathbb{R}^2 \qquad
   H^2(\mathbb{T_\omega}) = \mathbb{R} \qquad
\]
Finally, we have that a compact, connected, orientable, genus $g$, surface $\Sigma_g$ has de-Rham cohomology
\[
   H^0(\Sigma_g) = \mathbb{R} \qquad
   H^1(\Sigma_g) = \mathbb{R}^{2g} \qquad
   H^2(\Sigma_g) = \mathbb{R} \qquad
\]
Before we move on, we also want to briefly justify that connected, compact, orientable surfaces also admit a Riemann surface structure, thus giving us that for all genus $g$, we have a Riemann surface. If $S$ is a connected, smooth, oriented surface in $\mathbb{R}^3$, then we can define \emph{isothermal coordinates} on $S$. Isothermal coordinates are a tool from Riemannian geometry that gives charts on $S$, such that for $U \subset S$, we have $\psi : U \rightarrow \tilde{U} \subset \mathbb{R}^2$ such that $\psi^{-1}$ takes orthogonal vectors from $TpS$ for some $p \in U$ to orthogonal vectors in $\mathbb{R}^3$. Thus the composition of any such chart with the inverse of another chart \emph{preserve} the orientation of any two vectors on $S$, which gives $S$ a conformal structure thereby making it a Riemann surface. A particularly nice proof of the fact that we can cover such surfaces with isothermal coordinates is given in \cite{chern}. Hence, we can talk about compact Riemann surfaces of genus $g$. 

We conclude by defining a complex theory of cohomology, which works with the Dolbeault complexes and differentials more naturally, and are natural spaces to consider when studying Riemann surfaces.
\begin{defn}[Dolbeault Cohomology]
  Let $X$ be a Riemann surface. We define the (additive) Dolbeault cohomology groups as follows,
  \begin{itemize}
    \item $H^{0,0}(X) = \ker(\overline{\partial}:\Omega^0(X)\rightarrow \Omega^{0,1})$
    \item $H^{1,0}(X) = \ker(\overline{\partial}:\Omega^{1,0}\rightarrow \Omega^2(X))$
    \item $H^{0,1}(X) = \Omega^{0,1}/\ker(\overline{\partial}:\Omega^0(X)\rightarrow \Omega^{0,1}(X))$
    \item $H^{1,1}(X) = \Omega^2(X)/\ker(\overline{\partial}:\Omega^{1,0}(X) \rightarrow \Omega^2(X))$
  \end{itemize}
\end{defn}
These spaces seem rather abstract, and it is difficult to attach any meaning to them. We see that $H^{0,0}(X)$ is the space of holomorphic functions on $X$ and similarly $H^{1,0}(X)$ is the space of holomorphic $1$-forms. The other two spaces are more difficult to understand, and in the next section, we will hopefully start to understand them a bit better and how, in general, the Dolbeault cohomology groups relate to the de-Rham cohomology groups of Riemann surfaces.
\newpage
\chapter{The Uniformisation Theorem}
\section{The Laplacian and Hilbert spaces}
\subsection{The $\Delta$ operator and Harmonic functions}
We begin our treatment of the uniformisation theorem, with a discussion of a particularly important and famous differential operator on Riemann Surfaces; the Laplacian.

\begin{defn}[The Laplacian]\label{LaplacianDef}
  Let $X$ be a Riemann Surface. We define a linear map $\Delta:\Omega^0(X)\rightarrow \Omega^2(X)$, where $\Delta=2i(\overline{\partial}\circ \partial)$. We call this linear map the Laplacian. Often, we simply write $\Delta=2i\overline{\partial}\partial$. Where convenient, we also may use the fact that $\overline{\partial}\partial = -\partial\overline{\partial}$ which arises from the fact that $d^2 = 0$, and write $\Delta = -2i\partial\overline{\partial}$.
\end{defn}
In local co-ordinates, $\Delta$ simply recovers the expected Laplacian of a function since for a given function $f\in \Omega^0(X)$ we get that 
\begin{align*}
  \Delta f &= 2i\overline{\partial }\partial f = 2i\Big(\frac{\partial}{\partial\overline{z}}\Big)\Big(\frac{\partial}{\partial z}\Big)f(d\overline{z}\wedge dz) \\
  &=\frac{i}{2}\Big(\frac{\partial}{\partial x} + i\frac{\partial}{\partial y}\Big)\Big(\frac{\partial}{\partial x} - i\frac{\partial}{\partial y}\Big)f(2idx\wedge dy) \\
  &= -\Big(\frac{\partial^2 f}{\partial x^2} + \frac{\partial^2 f}{\partial y^2}\Big)dx\wedge dy 
\end{align*}
In this dissertation, write $\nabla^2$ for the classical laplacian in local coordinates. Therefore we can say $\Delta f = -(\nabla^2 f) dx\wedge dy$.

A special class of function that is also important to our theory is that of the 
Harmonic functions. They play an important role in studying the Laplacian operator generally and have a number of very nice properties that we will rely on in the coming sections.
\begin{defn}[Harmonic function]\label{HarmonicDef}
  Let $X$ be a Riemann surface. A function $f\in \Omega^0(X)$\footnote{We note that we can relax the differentiability condition considerably to make the functions only twice differentiable. Thus harmonic functions need to be at least twice differentiable.} is said to be harmonic if $\Delta f = 0$. If however, $f$ is a smooth complex function, it is said to be harmonic if both its real and imaginary parts are real valued harmonic functions.
\end{defn}
We claim that any holomorphic function is harmonic and show this using our language of differential forms. Recall that any complex function $f(z)$ where $z=x+iy$ can be written as a sum of two real valued functions of $(x,y)$, i.e, $f(z)=u(x,y)+iv(x,y)$.
\begin{lemma}\label{HolIsHarm}
  Holomorphic functions are harmonic functions, ie they have harmonic real and imaginary parts. 
\end{lemma} 
\begin{proof}
  Let $X$ be a Riemann surface and $f$ a holomorphic function defined on $X$.
  Let us consider the Laplacian applied to the sum $\frac{1}{2i}(f \pm \overline{f})$, which represents both the real and imaginary parts of $f$.
  \[\frac{1}{2i}\Delta(f \pm \overline{f}) = \overline{\partial}\partial(f \pm \overline{f})=-\partial(\overline{\partial}f) \pm \overline{\partial}\overline{(\overline{\partial}f)}=-\partial(0) \pm \overline{\partial}(0) = 0 \pm 0 = 0\]
  where both brackets $\overline{\partial} f = 0$ because $f$ is holomorphic.
\end{proof}

Note that whilst holomorphic implies harmonic, the converse isn't always true. However, we can at least say the following.

\begin{lemma}\label{HarmRealHol}
  Let $X$ be a Riemann Surface, $U$ an open set around a point $p \in X$ and let $\phi \in \Omega^0(U)$ be a harmonic function. Then there exists an open neighbourhood $V \subset U$ of the point $p$ and a holomorphic function $f \colon V \rightarrow \mathbb{C}$ with $\phi = \Re(f)$.
\end{lemma}
\begin{proof}
  Let $A \in \Omega^1(U)$ be the real one-form $A = i\overline{\partial}\phi + \overline{(i\overline{\partial}\phi)} = -\frac{\partial \phi}{\partial y} dx + \frac{\partial \phi}{\partial x} dy$. Since $\phi$ is harmonic, $\overline{\partial}\partial \phi = 0$ and $d = \partial + \overline{\partial}$, we get that $dA = 0$. Hence, if $V$ is an open, simply-connected set (such that $H^1(V)=0$), then we can find a function $\psi \in \Omega^0(V)$ such that $d\psi = A$. By equating coefficients we get that $\partial \psi = -i \partial \psi$ and $\overline{\partial} \psi = i \overline{\partial} \phi$. So if we construct a function $f \colon V \rightarrow \mathbb{C}$ where $f = \phi + i \psi$, we see that $\overline{\partial}f = \overline{\partial}(\phi + i\psi) = \overline{\partial}\phi +i(i\overline{\partial}\phi) = 0$ and hence, $f$ is a holomorphic function whose real part is $\phi$. 
\end{proof}

%Let us now give some examples of Harmonic functions.

%\begin{example}[Examples of various Harmonic functions on different domains]\label{harmonicexamples}
%  Over open sets in $\mathbb{C}$ we have:
%  \begin{itemize} 
%    \item Constant functions are trivially harmonic.
%    \item Any holomorphic function is automatically a harmonic function since its real and imaginary parts satisfy the Cauchy-Riemann equations.
%    \item $f(z)=e^z=e^{x+iy}=e^x\sin(y)$
%  \end{itemize}
%  Taking our domain of definition to be open subsets of $\mathbb{R}^3$, we have:
%  \begin{itemize} 
%    \item $f(x,y,z)=\frac{1}{\sqrt{x^2+y^2+z^2}}$
%  \end{itemize}
%  Finally, on the punctured plane $\mathbb{C}\setminus \{0\}$, we also have a nice family of examples:
%  \begin{itemize}
%    \item $f(x+iy) = \log(x^2 + y^2)$ and hence $g(x+iy)=K \log(x^2+y^2)$ for %all $K \in \mathbb{R}$.
%  \end{itemize}
%  All of the aforementioned functions satisfy the equation $\Delta f = 0$ on their domain of definition. This concludes the example.
%\end{example}

Harmonic functions are, clearly, quite well behaved and have nice properties that are not generally exhibited by other functions. These properties, play an important role in our treatment of the uniformisation theorem. One such property is the mean value theorem for harmonic functions.

\begin{lemma}[The Mean Value Theorem for Harmonic Functions]\label{MVT}
  Let $U$ be a region in $\mathbb{C}$ around a point $a \in U$ and let $\phi \in \Omega^0(U)$ be harmonic on $U$. Let $\gamma$ be a closed circle of radius $R > 0$ contained within $U$ that encircles the point $a$. Then, the mean value of $\phi$ over the set bounded by $\gamma$ is equal to the value of $\phi$ at the point $a$. 
\end{lemma}
We note that while the statement of this lemma talks about regions of $\mathbb{C}$, we could equally talk about regions on any Riemann surface as by definition of a Riemann Surface, we have chart maps that map regions on a Riemann Surface to regions in $\mathbb{C}$ homeomorphically. 
\begin{proof}
  Since $\phi$ is a real-valued harmonic function, then by lemma \ref{HarmRealHol} then we have a holomorphic function $f$ such that the real part of $f$ is $\phi$. We now consider the statement of Cauchys Integral formula about $\gamma$.
  \begin{align*}
    f(a) = \frac{1}{2\pi i}\int_{\gamma}\frac{f(z)}{z}dz
  \end{align*}
  Parameterising by $\gamma$ gives us that $z= a + Re^{i\theta}$ for $\theta \in [0, 2\pi)$ on $\gamma$ and that $dz = iRe^{i\theta}$. Plugging these into the above equation gives us the relation that
  \begin{align*}
    f(a) &= \frac{1}{2\pi i}\int_{\gamma}\frac{f(z)}{z}dz = \frac{1}{2\pi}\int_0^{2\pi}f(z)d\theta
  \end{align*}
  Since $\phi$ is the real part of $f$, we can split $f$ into real and imaginary parts with the integrals splitting into real and imaginary parts, thus giving us that the average value of $\phi(z)$ along $\gamma$ equals $\phi(a)$.
\end{proof}

\subsection{Hilbert Spaces and the space $\mathcal{H}(X)$}

Hilbert spaces are in some ways, the most familiar class of space that one may study. They crop up everywhere in Mathematics and more and more in Theoretical Physics and indeed they play a major part in the proof of the Uniformisation theorem. Specifially, we will want to construct a particular Hilbert space that will contain a function that we will use to prove the uniformisation theorem. Let us provide some definitions first.

\begin{defn}[Hilbert Space]
  An inner product space $V$ is called a Hilbert Space if it is a complete metric space with a norm induced by the inner product.
\end{defn}
Hilbert spaces over the real numbers are called real Hilbert spaces and those over the complex numbers are called complex Hilbert spaces. We see that Hilbert spaces are very similar to just normal inner product spaces and as such a lot of our examples are familiar vector spaces. Examples include:
\begin{enumerate}
  \item $\mathbb{R}^n$ with the usual Euclidean inner product, is a Hilbert space as it is an inner product space in which all Cauchy sequences attain their limits. In fact, one definition of $\mathbb{R}$ is the completion of the space of sequences of rational numbers. In this space for example, we can define the number $1$ as the limit of the sequence $\{\frac{9}{10^n}\}$.
  \item Given a space $A$, on which we can define integration with respect to a \emph{measure}\footnote{See reference \cite[(Chapter 11)]{babyRudin}} $d\mu$, we define the Hilbert space $L^2(A)$ to be the space of functions $f: A \rightarrow \mathbb{F}$ such that $\int_A |f|^2 d\mu$ is finite. Such functions are called square integrable. We define an inner product of two functions $f,g \in L^2(A)$ to be $\langle f, g\rangle = \int_A \overline{g} f d\mu$, where the overline denotes conjugation. We call this inner product, the $L^2$ inner product on $A$ and its associated norm $\Vert f \Vert^2 = \int_A |f|^2 d\mu$, the $L^2$ norm.
\end{enumerate}

We are now in a position to introduce a space of importance to our study of the Laplacian on a Riemann surface. Note that henceforth we consider only connected Riemann surfaces to alleviate issues related to having multiple connected componants.

\begin{lemma}
  Let $X$ be a connected Riemann Surface. We define a relation on $\Omega^0(X)$, whereby two functions are equivalent if they differ by a constant function. This relation, given the symbol $\sim$ defines an equivalence relation.
\end{lemma}

\begin{proof}
  Let $f,g,h \in \Omega^0(X)$ and let $a, b \in \mathbb{R}$. Since any function $f = f + 0$ then we have that $f \sim f$. If $f \sim g$ implying $f = g + a$, then we have that $g + (-a) = f$ and thus $g \sim f$, since $-a \in \mathbb{R}$. Finally, if $f \sim g$ and $g \sim h$ implying that $f = g + a$ and $g = h + b$, then we have that $f \sim h$ since $f = g + a = (h + b) + a = h + (b + a)$ and $b+a \in \mathbb{R}$. Hence, $\sim$ is indeed an equivalence relation. 
\end{proof}
With this equivalence relation we can now define the space of cosets of functions differing by a constant.
\begin{defn}[The space $\mathcal{H}(X)$]
  If $X$ is a compact, connected Riemann Surface we define $\mathcal{H}(X) = \Omega^0(X)/\sim$, to be the space of cosets under the above defined equivalence relation $\sim$. If $X$ is non-compact, then we simply take $\mathcal{H}(X)=\Omega_c^0(X)$.
\end{defn}

We note that $\mathcal{H}(X)$ is a vector space. This follows from the fact that $\Omega^0(X)$ and $\Omega^0_c(X)$ are themselves vector spaces, and in the compact case, the vector space structure descends through the quotient. Though this new space is somehow more complicated than the space of smooth functions, there is an advantage to working with these cosets; namely that we can define a particular inner product and associated norm which will be of particular interest to our study of the Laplacian on Riemann surfaces.

\begin{defn}[The Dirichlet Inner Product]\label{dInnerProduct}
  Let $X$ be a connected Riemann Surface and let $f, g \in \mathcal{H}(X)$.
  We define the Dirichlet Inner Product $\langle \cdot, \cdot \rangle_D : \mathcal{H}(X) \times \mathcal{H}(X) \rightarrow \mathbb{R}$ as 
  \[
    \langle f, g \rangle_D =\langle df, dg \rangle = 2i \int_X \partial f \wedge \overline{\partial} g 
  \]
  where the inner product in the middle is the $L_2$ inner product.
  This inner product of course, also defines a norm called the Dirichlet norm by $\Vert f\Vert _D^2 = \langle f, f \rangle_D $
\end{defn}

If the functions were to not supported within this region then, the value of the inner product and norm might be $+\infty$, so by considering only functions of suitable support, we guarantee that we have an inner product and norm. With all of this theoretical set up, it would be nice if we could somehow get back our claimed main object of study for this section; the Laplacian. Although hidden, the Dirichlet norm (and the associated inner product) already contain the Laplacian in their definition. The following proposition shows how.

\begin{prop}\label{InnerLaplacian}
  Given a connected Riemann surface $X$, if at least one function $f,g$ is in $\mathcal{H}(X)$ (hence having compact support on $X$), then 
  \begin{align*}
    \langle f, g \rangle_D = \int_X f \Delta g = \int_X g \Delta f
  \end{align*}
\end{prop}
\begin{proof}
  We first consider the following two identities:
  \begin{itemize}
    \item $\overline{\partial}(f\overline{\partial}g) = f (\overline{\partial}^{\, 2} g) + \overline{\partial}f \wedge \overline{\partial}g = 0 + 0 = 0$
    \item $\partial(f\overline{\partial}g)=\partial f \wedge \overline{\partial}g + f(\partial\overline{\partial}g)=\partial f \wedge \overline{\partial}g - \frac{1}{2i}f\Delta g$
  \end{itemize}
  Now let us expand the inner product. By the first identity above, we have that $f\overline{\partial}g$ is holomorphic, so that $d = \partial$. Hence, we can apply Stokes theorem. Note that since $X$ has no boundary, $\partial X = \emptyset$ and hence, that integral vanishes.
  \begin{align*}
    \langle f, g \rangle_D &= 2i\int_X \partial f \wedge \overline{\partial}g = 2i\int_X \partial(f\overline{\partial}g)+\int_X f\Delta g \\
    &= 2i\int_X \partial f \wedge \overline{\partial}g = 2i\int_{\partial X} f\overline{\partial}g + \int_X f\Delta g \\
    &= \int_X f\Delta g
  \end{align*}
  Since the Dirichlet inner product is a real inner product, it is symmetric and hence we have that $\langle f, g \rangle_D = \langle g, f \rangle_D $ and so $\int_X f\Delta g = \int_X g \Delta f$ as claimed.
\end{proof}

\section{The Dirichlet Energy Functional}

A point that we will not be proving explicitly is that all functionals we consider hereon are "continuous", in the sense that given a functional $F:\mathcal{H}(X) \rightarrow \mathbb{R}$, we have that for any $\epsilon > 0$ we have a $\delta > 0$ such that for all $x,y \in \mathcal{H}(X)$, $\Vert x - y \Vert < \delta \Rightarrow |F(x) - F(y)| < \epsilon$. This continuity allows us to search for limits, by looking at how the functional acts on Cauchy sequences and their corresponding Cauchy sequences in $\mathbb{R}$.

\begin{defn}[The $\hat{\rho}$ functional]
  Let $X$ be a connected Riemann Surface and $\rho \in \Omega^2_c(X)$ be a $2$-form of compact support in $X$ such that $\int_X \rho = 0$. Then, we define the functional $\hat{\rho} \colon \mathcal{H}(X) \rightarrow \mathbb{R}$ to be defined as, for some $f$ of compact support in $\mathcal{H}(X)$,
  \begin{align*}
    \hat{\rho}(f) = \frac{1}{2}\int_X f\rho
  \end{align*}
\end{defn}

Since $\int_X \rho = 0$ then we have that this functional is linear on functions of compact support in $\mathcal{H}(X)$. Note that, if $X$ is a compact Riemann surface, then all functions and $2$-forms on $X$ automatically have compact support in $X$, so the compact support conditions really only matter in the cases of non-compact Riemann surfaces. We claim that $\hat{\rho}$ is a linear functional.
\begin{lemma}[$\hat{\rho}$ is a linear functional]\label{rhohatlinear}
  Let $X$ be a connected Riemann surface. Then $\hat{\rho}$ is a linear functional, ie $\hat{\rho}(\lambda f+ \mu g) = \lambda \hat{\rho}(f) + \mu \hat{\rho}(g)$
\end{lemma}
Recall that elements of $\mathcal{H}(X)$ are equivalence classes of smooth functions with compact support on $X$ which differ by a constant. We denote cosets as $\tilde{f}$, where $f$ is the coset representative.
\begin{proof}
  Let $\tilde{f},\tilde{g} \in \mathcal{H}(X)$ and $\lambda,\mu \in \mathbb{R}$. Then for some real constant, $a,b \in \mathbb{R}$ we have that $\tilde{f}(z) = f(z) + a$ and $\tilde{g}(z) = g(z) + b$. Recall also that $\int_X \rho = 0$.
  Hence, we get that:
  \begin{align*}
    \hat{\rho}(\lambda \tilde{f}+ \mu \tilde{g}) &= \hat{\rho}(\lambda(f + a) + \mu(g + b)) \\
    &=\frac{1}{2}\int_X (\lambda(f + a) + \mu(g + b))\rho \\
    &=\frac{\lambda}{2} \int_x f\rho + \frac{\lambda a}{2}\int_X \rho + \frac{\mu}{2} \int_X g\rho + \frac{\mu b}{2}\int_X \rho \\
    &=\lambda \hat{\rho}(f) + \mu \hat{\rho}(g)=\lambda \hat{\rho}(\tilde{f}) + \mu \hat{\rho}(\tilde{g})
  \end{align*}
  with the last equality holding since clearly $f \in \tilde{f}$ and $g \in \tilde{g}$.
\end{proof}
We now define a functional in terms of $\hat{\rho}$ and the Dirichlet norm on $\mathcal{H}(X)$ of crucial importance. We will be using this functional to help us set up a problem whose solution allows us to prove the Uniformisation theorem. 

\begin{defn}[Dirichlet Energy Functional]
  Let $X$ be a connected Riemann Surface, and suppose $f \in \mathcal{H}(X)$. Suppose further that we have $2$-form $\rho \in \Omega^2 (X)$ (or $\Omega^2_c (X)$ if $X$ is non-compact) such that $\int_X \rho = 0$. Then we can define a functional, $\mathcal{L}: 
  \mathcal{H}(X) \rightarrow \mathbb{R}$ such that $\mathcal{L}(f) = \Vert f\Vert ^2_D - 2\hat{\rho}(f)$. This $\mathcal{L}$ is called the Dirichlet energy functional, or sometimes Dirichlet energy for short.
\end{defn}

So how can we use $\mathcal{L}$ to set up this problem? We motivate this problem physically, using the theory of Electrostatics; a subject which mathematicians in Riemann's time were already very familiar with. The physical principle of least action gives that physical systems always act to minimise their energy. Thus we too should look for a physical description of the electrostatic energy on an arbitrary Riemann surface and minimise it. The description of the functions which minimise this energy that we thus get will give us our relation between the different cohomology groups. Given a Riemann surface $X$, we imagine it is made of a electrically conductive material. We impose on $X$ a static, smooth distribution of charge, such that the net charge on $X$ due to $\rho$ is zero. We call such a $\rho$ an \emph{externally imposed} charge distribution. The fact that the net charge on $X$ is zero gives us that $\int_X \rho = 0$. To get a concrete idea of what sort of externally imposed charge distributions have this property, let us consider $X=\widehat{\mathbb{C}}$. 
An example distribution, say $\rho$ would be to take two small disjoint circular regions, of diameter $\epsilon > 0$ on the Riemann sphere such that our smooth charge distribution is compactly supported inside these two regions say $A$ and $B$. After fixing an area form on $\widehat{\mathbb{C}}$, say $dz \wedge d\overline{z}$, let us say that the function representing the distribution on $A$ is a bump function $\beta$, with the function for $B$ being $-\beta$. We see clearly that the integral of our distribution splits into two. Hence, we get that since $\rho$ is zero outside $A \cup B$ that $\int_{\widehat{\mathbb{C}}} \rho = \int_A \beta dz\wedge d\overline{z} + \int_B (-\beta) dz\wedge d\overline{z} = 0$, by the definition of $\rho$. So acceptable charge distributions must be particularly well behaved. In fact, this example is similar to defining a pair of opposing point charges on our Riemann surface. Such a charge distribution is called a dipole. 

The potential energy felt by a test charge\footnote{A hypothetical object which interacts with electric charge, but doesnt act to change the overall description of charge in the system} at a point $p$, denoted as $\phi(p)$, may be non zero everywhere on $X$. This $\phi$ is an example of an electric potential on $X$. So what is the potential energy of this system? We find it is energy is split into two parts. Firstly, there is the energy stored in $\phi$ itself, which is given by $\frac{1}{2}\Vert \phi \Vert^2_D$, half the Dirichlet norm of $\phi$. The second term, is the energy contribution due to $\rho$ interacting with $\phi$ which is described by $-\hat{\rho}(\phi)$. An in depth discussion of the physics needed for this can be found in \cite[(p.45-46, Chapter 1)]{electromagentismBook}. Since we are considering the total energy of the system, we can sum up these two terms up to give that $\frac{1}{2}\Vert \phi \Vert^2_D - \hat{\rho}(\phi)$ describes the total energy of our problem. We note though that $\frac{1}{2}\Vert \phi \Vert^2_D - \hat{\rho}(\phi) = \frac{1}{2}\mathcal{L}(\phi)$. Thus, we seek to minimise $\mathcal{L}$ over all functions in $\mathcal{H}(X)$, to find the minimum energy of our system. We call such an object which encapsulates the energy of the system, the \emph{Lagrangian} of the system. Hence, we can can state the problem we want to solve:
\begin{problem}
  Given a charge distribution $\rho$, of integral zero, over our Riemann surface $X$, which functions $\phi \in \mathcal{H}(X)$ exist that minimise $\mathcal{L}(\phi)$?
\end{problem}
The answer, as we will find, is that precisely those fields which arise as the potential due to the \emph{same} charge distribution $\rho$. We must be careful however, as we might be looking for functions which minimise $\mathcal{L}$, but such functions may not exist. As we will show, this functional is bounded from below, giving us a necessary condition for a minimum to exist. However, we note that this condition is not sufficient, and we will deal with that in the latter part of this section. Let us, for a moment, suppose such a potential function exists and that $\mathcal{L}$ is bounded from below. Which functions act to minimise $\mathcal{L}$? We begin by computing $\delta \mathcal{L}= 0 $, the first variation of $\mathcal{L}$. A formal introduction to the notion of the first varitation of a functional can be found in \cite[(p.4)]{jost}, though it is sufficient to imagine it acts much the same as a derivative for functions in that its' zeros give the stationary points of the functional. To compute it, we let $\epsilon > 0$ and let $g$ be function on $X$. The conditions on $g$ will be more precisely defined later, so for now, we assume such a $g$ exists and it is, at least, integrable. Computing the first variation of $\mathcal{L}$ gives:
\begin{gather*}
  \delta \mathcal{L} = \frac{d}{d\epsilon}\biggr\rvert_{\epsilon = 0}\mathcal{L}(f+\epsilon g) = 0 \\
  \frac{d}{d\epsilon}\biggr\rvert_{\epsilon = 0}(\Vert f+\epsilon g\Vert _D^2 -2\hat{\rho}(f+\epsilon g)) = 0 \\
  \frac{d}{d\epsilon}\biggr\rvert_{\epsilon = 0} \langle f + \epsilon g, f + \epsilon g \rangle_D -2\frac{d}{d\epsilon}\biggr\rvert_{\epsilon = 0}(\int_X f\rho  + \epsilon \int_X g\rho) = 0 \\
  \frac{d}{d\epsilon}\biggr\rvert_{\epsilon = 0}(\langle f, f \rangle_D + 2\epsilon \langle f, g \rangle_D +\epsilon^2 \langle g, g \rangle_D) -2 \hat{\rho}(g) = 0 \\
  2\langle f, g \rangle_D -2\hat{\rho}(g) = 0 \\
  \int_X g\Delta f - \int_X g\rho = 0 \\
  \int_X g(\Delta f - \rho) = 0 
\end{gather*}
The appearence of the Laplacian comes from proposition \ref{InnerLaplacian}.
Since we have that $f$ and $\rho$ have compact support on $X$, then for this integral to be zero, we need the following statement to be true: There exists a unique solution $f$, up to addition by a constant, of the equation 
\[ \Delta f = \rho \Longleftrightarrow \int_X \rho = 0\]
The equation on the left, known as Poissons' equation, is precisely the problem we are seeking. We call finding such an $f$, ``solving the Poisson equation''. It turns out, to prove the uniformisation theorem we really only need to solve the Poisson equation on two types of Riemann surface; connected, compact Riemann surfaces and connected, simply-connected, non-compact Riemann surfaces. Before we study Poissons' equation too deeply however, we must be careful and show that $\mathcal{L}$ is really bounded from below thus motivating our study of Poissons' equation.  We do this by showing that $\hat{\rho}$ is a bounded functional on $\mathcal{H}(X)$, and thus, by the positivity of $\Vert \cdot \Vert^2_D$, we get a lower bound for $\mathcal{L}$.

\subsection{Showing the Dirichlet energy $\mathcal{L}$ is bounded from below}
To prove $\hat{\rho}$ is bounded, we need the following theorem and its subsequent corollary. We proceed by working in a bounded, convex, open set in the complex plane. We can do so as we can map any bounded, convex, open set from our Riemann surfaces conformally to the such sets in the complex plane using chart maps. Similarly, due to conformal equivalence, we can assume such sets are circular discs in the complex plane. We omit the proofs of the following theorem and its corollary due to them being long arguments that are well described in the quoted references.
\begin{thm}\cite[(p.122, Theorem 11)]{donaldson}\label{quotedTheorem11}
  Let $\Omega$ be a bounded, convex, open set in $\mathbb{R}^2$ and $\psi$ be a smooth function on an open set containing the closure $\overline{\Omega}$ with $\overline{\psi}$ denoting the average 
  \[\overline{\psi} = \int_{\Omega}\psi dS\] 
  where $A$ is the area of $\Omega$. Then, for $x \in \Omega$, we have 
  \[|\psi(x) - \overline{\psi}| \leq \frac{d^2}{2A}\int_\Omega \frac{1}{|x - y|}|\nabla\psi(y)|dS_y\]
\end{thm}
We also have the following important corollary.
\begin{cor}\cite[(p.123, Corollary 6)]{donaldson}\label{quotedCorollary6}

  Under the hypothesis from Theorem \ref{quotedTheorem11}, we have that 
  \[\int_\Omega |\psi(x) - \overline{\psi}|^2 dS_x\leq \left(\frac{d^3\pi}{A}\right)^2\int_\Omega |\nabla\psi|^2 dS\]
  where $\nabla$ denotes the usual gradient operator in $2$ dimensions.
\end{cor}
Using these two estimates we can prove the following theorem.
\begin{prop}\label{PartitionOfUnity}
  Let $X$ be a connected Riemann surface. Then the functional $\hat{\rho}:\mathcal{H}(X) \rightarrow \mathbb{R}$ is bounded, ie there exists a constant $C$ such that $|\hat{\rho}(\tilde{f})|  \leq C \Vert \tilde{f} \Vert_D$, for all $\tilde{f} \in \mathcal{H}(X)$.
\end{prop}
The following proof follows the proof found on \cite[p.125]{donaldson}, though we have modified our notation to make certain aspects of the proof clearer and easier to follow.
\begin{proof}
  We split this proof into two parts. First we show that $\hat{\rho}$ is bounded in the case when the compact set $\text{supp}(\rho) = K$ is contained within a single coordinate chart. Then we show that we can extended the boundedness of $\hat{\rho}$ over the whole of $K$ if it cannot be contained in a single coordinate chart. Note that this gives that if $X$ is compact, that we can extend the boundedness of $\rho$ to the whole of $X$ as $K=X$. We consider a bounded, convex set $U$ in $\mathbb{C}$. Let $\tilde{f} \in \mathcal{H}(X)$. We write $\hat{\rho}(\tilde{f}) = \int_U (f + a) \rho$ for some $a \in \mathbb{R}$ and set the constant $a$ be equal to $-\overline{f}$, the average value of $f$ on $U$, giving that  $\hat{\rho}(\tilde{f}) = \int_U (f - \overline{f}) \rho$. By fixing an area form in this coordinate chart, say $dS$, we can write $\rho = g dS$, where $g \in \Omega^0(X)$. This therefore allows us to write $\hat{\rho}(\tilde{f}) = \int_U (f - \overline{f}) g dS$ which by the Cauchy-Schwarz inequality gives
  \[ |\hat{\rho}(\tilde{f})| = \left| \int_U (f - \overline{f}) g dS \right| \leq \Vert g \Vert \Vert f-\tilde{f} \Vert  \]
  Hence, by Corollary \ref{quotedCorollary6}, we get that $|\hat{\rho}(\tilde{f})| \leq C \Vert \nabla f \Vert = C \Vert df \Vert$ where $C = d^3\pi A \Vert g \Vert$, and we have used the fact that the norm of the gradient and exterior derivative of a function are equal.
  Finally, if we now compose with a chart map to map back to $X$ from $U$, and letting $ \Vert \cdot \Vert_U$ indicate the usual norm on $U$, we have that $ \Vert df \Vert_U \leq \Vert df \Vert_X = \Vert \tilde{f} \Vert_{D,X}$. Hence we have that if $\hat{\rho}$ is supported in a single coordinate chart on $X$, then $\hat{\rho}$ is a bounded operator.

  So to extend this to the case where the compact set $K$ is contained in multiple coordinate charts on $X$, we first fix a finite cover of $K$ by coordinate charts of the type considered in the previous case. This is possible since $K$ is compact. We call this family of coordinate charts $U_{\alpha}$. Since we showed that for each coordinate chart, a $2$-form of compact support in that chart of integral zero is bounded, we wish to somehow stitch together these forms to form a $2$-form of integral zero over the whole of $K$. To do this, we introduce a partition of unity $\chi_{\alpha}$, subordinate to our choice of finite cover $U_{\alpha}$. Since integration of $2$-forms on a compact, connected set $V \subseteq X$ defines an isomorphism between $H^2(V)$ and $\mathbb{R}$, we can write the $2$-form from our functional $\hat{\rho}$ as $\rho = d\theta$ for some $\theta \in \Omega^1(K)$. Now on each $U_{\alpha}$ we derive a $2$-form of compact support from $\theta$ using our partition of unity by setting $\rho_{\alpha} = d(\chi_{\alpha}\theta)$. Each $2$-form $\rho_{\alpha}$ is of compact support in $U_{\alpha}$ which it gets from $\chi_{\alpha}$. Thus we have for each $\alpha$
  \[ \int_K\rho_{\alpha} = \int_K d(\chi_{\alpha}\theta) = \int_{\partial K} \chi_{\alpha}\theta = 0 \]
  which vanish by Stokes theorem if $K=X$ since $X$ has no boundary by or if $X$ is not compact since $\rho_\alpha$ must vanish on the boundary of $K$\footnote{Technically, $\rho_{\alpha}$ isn't defined on $\partial K$ but we can extend $\rho$ out of $K$ over the whole of $X$ by zero.}. So $\int_K \rho_{\alpha} = 0$. Finally, since $\sum_{\alpha} \chi_{\alpha} = 1$, we have that 
  \[ \rho = d\theta = d\left(\sum_{\alpha}\chi_{\alpha}\theta\right) = \sum_{\alpha}d(\chi_{\alpha}\theta) = \sum_{\alpha}\rho_{\alpha}\]
  As such, we have constructed a $2$-form $\rho$ such that $\int_K \rho = 0$. Since we showed in the first part that each $\hat{\rho}_{\alpha}$ is bounded in its coordinate chart, and since $\rho$ is equal to a finite sum of $\rho_{\alpha}$ we conclude that the functional $\hat{\rho} = \sum_{\alpha}\hat{\rho}_{\alpha}$ is also bounded since it is a finite sum of bounded linear maps. Furthermore, the boundedness holds even outside $K$ since we can extend $\hat{\rho}$ by zero over the whole of $X\setminus K$.
\end{proof}

So we have that the functional $\hat{\rho}$ is bounded on both when $X$ is compact and non-compact. Let us now directly show that $\mathcal{L}$ is hence bounded from below. We have that $\mathcal{L}(f) = \Vert f \Vert^2_D -2\hat{\rho}(f) $ for some function $f \in \mathcal{H}(X)$. We showed above that $|\hat{\rho}(f)| \leq C \Vert f \Vert_D$ for the constant $C$ defined above. Since $\hat{\rho}(f) \leq |\hat{\rho}(f)|$ we have that $ \mathcal{L}(f) \geq \Vert f \Vert^2_D -2C \Vert f \Vert_D = (\Vert f \Vert_D - C)^2 - C^2 \geq - C^2$. Hence, we have that the Dirichlet energy is indeed bounded from below, irrespective of the compactness of $X$.

\section{Poisson's equation on compact Riemann surfaces}
Since $\mathcal{L}$ is bounded on both compact and non compact Riemann surfaces, we need to study the two cases separately. We begin this section by stating our problem of solving Poisson's equation for the compact Riemann surfaces.
\begin{thm}[Poissons's equation on compact Riemann surfaces]\label{compactPoisson}
  Let $X$ be a connected, compact Riemann surface. Let $\rho \in \Omega^2(X)$ be a $2$-form on $X$. Then, there exists a unique, smooth solution $f \in \Omega^0(X)$ to the equation $\Delta f = \rho$, up to the addition of a constant, if and only if $\int_X \rho = 0$. This equation is called the Poisson equation.
\end{thm}
We can immediately prove two things:
\begin{enumerate}
  \item If $f$ is a smooth solution to $\Delta f = \rho$, then $\int_X \rho= 0$
  \item $f$ is unique up to a constant.
\end{enumerate}
Unofrtunately, the \emph{only if} direction is much tricker to prove and the majority of this section is dedicated to proving it. So let us begin the proof of Theorem \ref{compactPoisson} by proving the \emph{if} direction and uniqueness of the solution.
\begin{proof}
  We wish to first prove that if $f$ is a smooth solution to $\Delta f = \rho$ then $\int_X \rho = 0$. To show this, we note that since $d = \partial + \overline{\partial}$ and $d(\partial f) = \partial^2 f + \overline{\partial}\partial f = \overline{\partial}\partial f$ since $\partial^2 = 0$. Recall that  $\Delta f = 2i \overline{\partial}\partial f$. Combining these two gives us that $\Delta f =2id(\partial f)$. So let us integrate $\rho$.
  \begin{align*}
    \int_X \rho = \int_X \Delta f = 2i \int_X \overline{\partial}\partial f = 2i \int_X d(\partial f) = 2i\int_{\partial X}\partial f = 0
  \end{align*}
  where the last equality follows from Stokes theorem and the fact that $X$ has no boundary. 

  Now to show that such a solution is unique up to a constant. We assume $f$ and $g$ are both solutions to the Poisson equation, ie $\Delta f = \rho$ and $\Delta g = \rho$. Then we have that $\Delta (f - g) = 0$, ie that $f-g$ is a harmonic function. Thinking of the Dirichlet norm for a moment, if we consider $\Vert f-g\Vert _D$ we get $\int_X (f-g)\Delta(f-g) = 0$. Hence, we have $\Vert f-g\Vert _D = 0$. Unfortunately, the Dirichlet norm is not a norm on $\Omega^0(X)$ but on $\mathcal{H}(X)$, but elements of $\Omega^0(X)$ can easily be mapped to elements of $\mathcal{H}(X)$. So let us consider $\Vert \tilde{f}-\tilde{g}\Vert _D = \Vert df - dg\Vert $. It follows that $\Vert \tilde{f}- \tilde{g}\Vert _D = 0$ since $\Delta(c) = 0$, for any constant $c$, so by the definition of $d$ and the Dirichlet norm, we get that $ df - dg = d(f-g) = 0$ implying that $f - g = c$ where $c$ is a constant. So any solution of the Poisson equation is unique up to the addition of a constant.
\end{proof} 

To complete our proof of Theorem \ref{compactPoisson}, now must now show that given a $2$-form $\rho \in \Omega^2(X)$ such that $\int_X \rho = 0$, we can find a smooth function $f$ such that $\Delta f = \rho$. To do so, we first lay out some technical ground work in the form of convolutions, after which, we show that whilst minimising functions are indeed obtained as solutions to Poissons equation, they may live "outside" our space of all possible smooth functions on $X$. We conclude by identifying these functions with smooth functions on $X$.

\subsection{Convolutions and related technical lemmas}
\begin{defn}[Convolution of functions]\label{ConvolutionDefn}
  Let $f,g$ be smooth, complex functions defined on $\mathbb{R}^n$. Then their convolution is defined as $(f * g)(x) = \int_{\mathbb{R}^n}f(y)g(x-y)dy$ where $dy$ is volume form on $\mathbb{R}^n$, provided the integral exists for all $x \in \mathbb{R}^n$ .
\end{defn}
Note that this definition implies we have the identity $\langle f, g*h \rangle = \langle g*f, h \rangle$, and we will refer to this as the triple product identity for convolutions. Also note that, one can relax the requirement for the integral to exist on all $x \in \mathbb{R}^n$ to requiring $x$ to be defined on $\mathbb{R}^n\setminus{U}$, where $U \subset \mathbb{R}^n$ is a set of measure zero. The definition of a set of measure zero can be found at \cite[p.50]{spivak}.
For a detailed treatment of convolutions of functions and the theory of distributions and a more in depth discussion of the theory of Partial Differential Equations, one can refer to \cite[(Chapter 6)]{rudin}. 
Let us prove some useful lemmas involving convolutions with the Laplacian that we will be using later. Before we begin, we recall Green's identity, though modified for the case of the plane, as we will be using it shortly. 
\begin{thm}[Green's identity]
  Let $u,v \in \mathbb{R}^2$ be smooth functions on a path connected, bounded set $\Omega \subset \mathbb{R}^2$. Then we have that  
  \[\iint_{\Omega} (u \nabla^2 v - v \nabla^2 u) dS= \int_{\partial \Omega} \left(u \frac{\partial v}{\partial n} - v \frac{\partial u}{\partial n}\right) dl, \] 
  where $n$ is the oriented normal of the boundary curve, $dS$ is the standard surface area element and $dl$ is the standard arc length element of the boundary.
\end{thm}

We now prove an important technical lemma about convolutions involving the Laplacian.
\begin{lemma}\label{ConvProperties}
  Consider the function $V(z) = \frac{1}{2\pi}\log|z|$. We call this function the Newtonian potential. $V$ is well defined on all of $\mathbb{C}\setminus\{0\}$. Let us further consider two smooth functions $\sigma, \tau$ of compact support in a neighbourhood around a point $z$ in $\mathbb{C}$. Then we have the following two points:
  \begin{itemize}
    \item $(V * \Delta \sigma)(z)=\sigma(z)$
    \item $\Delta(V * f)(z) = f(z)$
  \end{itemize}
\end{lemma}

\begin{proof}
  Since we can change coordinates linearly, we can set $z=0$ and evaluate the convolution of $(V * \Delta \sigma)(0)$. Since $\log(|z|)$ is harmonic on $\mathbb{C}\setminus \{0\}$, then on $\mathbb{C}$ we have that $\Delta \log(|z|) = 0$ for $\vert z \vert > 0$. By setting $dS$ to be our area form, we get:
  \begin{align*}
    (V * \Delta \sigma)(0) &= \iint_{\mathbb{C}}\frac{1}{2\pi}\log(|w|)\Delta \sigma dS
  \end{align*}
  Let $\text{supp}(\sigma) = U \subset \mathbb{C}$. By the Heine-Borel theorem, since this is compact subset of the plane, it is also closed and bounded and hence, $U$ has a boundary $\partial U$. Thus we can replace our domain of integration from $\mathbb{C}$ to $U$ since $\sigma$ is zero outside of $U$. Since $\sigma(w) = 0$ for all $w \in \mathbb{C}\setminus U$ then the integral needs only to be evaluated on $\text{supp}(\sigma) \in U$. Care must be taken at the origin however. We begin by cutting out from $U$, for some small $\delta >0$, the set $B_{\delta}(0)$, such that the ball $B_{\delta}(0) \subset U$. Then we can split $U$ into a union of two sets $U=B_{\delta}(0) \cup U_{\delta}$, where $U_{\delta} = U\setminus B_{\delta}(0)$. In the limit of $\delta \rightarrow 0$ we get that $U=U_{\delta}$. So let us evaluate the above integral in this limit.
  We begin by applying Green's identity, over the set $U_{\delta}$, since by letting $u = \frac{1}{2\pi}\log(|w|)$ and $v = \sigma$ we get the statement of the left hand side of Green's identity since $\frac{1}{2\pi}\log(|w|)$ is harmonic on $U_{\delta}$. So we proceed as follows.
  \begin{align*}
    (V * \Delta \sigma)(0) &= \lim_{\delta \rightarrow 0} \iint_{U_{\delta}} \frac{1}{2\pi}\log(|w|)\Delta\sigma(w) dS \\
    &= \lim_{\delta \rightarrow 0} \int_{\partial U_{\delta}}\left(\frac{1}{2\pi}\log(|w|)\frac{\partial \sigma(w)}{\partial n} - \sigma(w)\frac{\partial}{\partial n}\left(\frac{1}{2\pi}\log(|w|)\right)\right)dl
  \end{align*}
  Careful thought gives us that $\partial U_{\delta}$ has two componants, one on the outer edge of $U_{\delta}$, the boundary of the support of $\sigma$, called the outer boundary, and one on the circle around the origin, of radius $\delta$, the inner boundary. Since $\sigma$ is continuous, then on the outer boundary, the value of $\sigma$ must be zero and so the integral there is zero. Hence, we only need to consider what happens on the interior boundary. So to compute the integral, we must consider the normal outwards vector of this circle. However, the outwards vector of this inner circle points inwards towards the origin. So we define the normal vector $\hat{\underline{n}}= -\frac{1}{r}\hat{\underline{r}}$ and that the normal derivative is $\frac{\partial}{\partial n} = -\frac{\partial}{\partial r}$.
  Finally, since we are taking a radial limit it makes sense to parameterise $w$ using polar coordinates, so $w=re^{i\theta}$ with $r\in [0, \infty)$ and $\theta \in [0, 2\pi)$. Our integral now becomes
  \begin{align*}
    (V * \Delta \sigma)(0) &= \lim_{\delta \rightarrow 0} \int_{\partial U_{\delta}}\left(\frac{1}{2\pi}\log(|w|)\frac{\partial \sigma(w)}{\partial n} - \sigma(w)\frac{\partial}{\partial n}\left(\frac{1}{2\pi}\log(|w|)\right)\right)dl \\
    &= \lim_{\delta \rightarrow 0} \int_{\partial U_{\delta}}\left( -\frac{1}{2\pi}\log(r)\frac{\partial \sigma(re^{i\theta})}{\partial n} + \frac{1}{2\pi r}\sigma(re^{i\theta})\right)dl
  \end{align*}
  Since we are evaluating this integral along the edge of the circle of radius $\delta$, we can set $r=\delta$ and $dl$ to $d\theta$ since we are parameterising our position on the circle by the angle $\theta$. Hence our integral becomes
  \begin{align*}
    (V * \Delta \sigma)(0) &= \lim_{\delta \rightarrow 0} \int_{\partial U_{\delta}}\left( -\frac{1}{2\pi}\log(r)\frac{\partial \sigma(re^{i\theta})}{\partial n} + \frac{1}{2\pi r}\sigma(re^{i\theta})\right)dl \\
    &= \lim_{\delta \rightarrow 0} \left( -\frac{1}{2\pi}\log(\delta) \int_{0}^{2\pi} \frac{\partial \sigma(\delta e^{i\theta})}{\partial n} d\theta + \frac{1}{2\pi \delta}\int_{0}^{2\pi}\sigma(\delta e^{i\theta})d\theta\right)
  \end{align*}
  If we now consider the averages of these integrals, we can attempt to directly compute the limit. We note that the average value of each integral will be the total arc length of the circle, $2\pi \delta$, multiplied by some value, as we are intgerating over the whole circle. We let the symbol $\overline{\mu_{\delta}}$ indicate the average value of the first integral and $\overline{\sigma_{\delta}}$ to mean the average value of the second integral, at radius $\delta$. Substituting these into the integral therefore gives us
  \begin{align*}
    (V * \Delta \sigma)(0) &=\lim_{\delta \rightarrow 0} \left( -\frac{1}{2\pi}\log(\delta) \int_{0}^{2\pi} \frac{\partial \sigma(\delta e^{i\theta})}{\partial n} d\theta + \frac{1}{2\pi \delta}\int_{0}^{2\pi}\sigma(\delta e^{i\theta})d\theta\right) \\
    &= \lim_{\delta \rightarrow 0} \left( -\delta \log(\delta)  \overline{\mu_{\delta}}  + \overline{\sigma_{\delta}}\right) = \sigma(0)
  \end{align*}
  where the last equality holds since $\lim\limits_{\delta \rightarrow 0}\delta \log(\delta) = 0$ and $\overline{\sigma_{\delta}} \rightarrow \sigma(0)$ as $\delta \rightarrow 0$ since we have that the average value of the integral of $\sigma$ on smaller and smaller circles around the orgin tends to the value of $\sigma$ at the origin.
  Hence we have shown that $(V * \Delta \sigma)(0) = \sigma(0)$. \newline
  To show now that $\Delta(V*f)(z) = f(z)$, we begin by bringing the Laplacian into the integral, which we can do since the integral is with respect to the dummy variable $w$ but $\Delta$ is the Laplacian on the coordinate $z$. Hence we have that
  \begin{align*}
    \Delta(V*f)(z) &= \Delta \int_\mathbb{C} V(w)f(z-w)dS_w \\
    &= \int_\mathbb{C} V(w)\Delta f(z-w)dS_w
  \end{align*}
 since $V$ is a function of the coordinate $w$ but $\Delta$ is an operator on the coordinate $z$. Hence, it passes through to $f$ which is a smooth function of compact support. Hence, we have shown that $\Delta(V*f)(z) = V*(\Delta f)(z)$ which we have just shown is $f(z)$. Therefore we have that $\Delta(V*f)(z) = f(z)$ as claimed.
\end{proof} 

\subsection{The completion of $\mathcal{H}(X)$ and Weyl's lemma}\label{Hilbertsraumsection}
Though we showed earlier that $\mathcal{L}$ is bounded, we must be careful as its minimum may not live in $\mathcal{H}(X)$. We begin this section by constructing the completion of the space $\mathcal{H}(X)$, an extension of the space $\mathcal{H}(X)$ that contain the limits of all \emph{Cauchy sequences} of functions on $\mathcal{H}(X)$. What does this space look like? We construct this space abstractly as follows.
\begin{defn}[The space $\overline{\mathcal{H}}(X)$]\label{completeH}
  Let $X$ be a connected Riemann surface. The space $\overline{\mathcal{H}}(X)$, the abstract completion of $\mathcal{H}(X)$, can be constructed from the inner product space $\mathcal{H}(X)$ with the Dirichlet inner product, as the space of equivalence classes of Cauchy sequences of elements in $\mathcal{H}(X)$, under the equivalence relation $\sim$ where two Cauchy sequences $\{\psi_i\}$ and $\{\phi_i\}$ are equivalent if $\lim\limits_{i \rightarrow \infty}\Vert \psi_i - \phi_i\Vert_D \rightarrow 0 $.
  Therefore, a point in $\overline{\mathcal{H}}(X)$ is an equivalence class of Cauchy sequences of functions that differ by a constant, that have the same limit.
\end{defn}
This space is clearly very complicated, but its usefulness to us comes from the fact that we can guarantee that Cauchy sequences of functions will attain their limit in $\overline{\mathcal{H}}(X)$. This space is also a vector space because we can add points of this vector space, by summing up the sequences term by term. The space $\overline{\mathcal{H}}(X)$ also has an inner product, which we define as, for any two points $\psi, \phi \in \overline{\mathcal{H}}(X)$, where $\psi = \{\psi_i\}$ and $\phi = \{\phi_i\}$, the product $( \psi, \phi )_D = \lim\limits_{i \rightarrow \infty} \langle \psi_i, \phi_i \rangle_D$. Note, that we use the notation $( \cdot, \cdot)_D$ for the inner product in $\overline{\mathcal{H}}(X)$ and $\langle \cdot, \cdot \rangle_D$ for the inner product in $\mathcal{H}(X)$. Note that the sequence of values in $\mathbb{R}$ of the inner product, is itself a Cauchy sequence in $\mathbb{R}$. Since we have a complete inner product space, with a norm that we can derive from our new inner product, $( \cdot, \cdot )_D$, then $\overline{\mathcal{H}}(X)$ is in fact, a Hilbert space. 

We now wish to extend our previously defined functionals to $\overline{\mathcal{H}}(X)$. We identify these extended functionals with an underline, except for the extended Dirichlet inner product, which we denote as $(\cdot, \cdot)_D$. Note that this extended Dirichlet inner product also extends our norm to $\overline{\mathcal{H}}(X)$ by defining it as $\underline{\Vert\psi\Vert}_D^2 = (\psi, \psi)_D$ for some $\psi \in \overline{\mathcal{H}}(X)$.
So we now need to extend $\hat{\rho}$. Its extension to $\overline{\mathcal{H}}(X)$ can be accomplished similarly to the Dirichlet inner product by
\[\underline{\hat{\rho}}(\psi) = \lim_{i \rightarrow \infty} \hat{\rho}(\psi_i) = \lim_{i \rightarrow \infty} \int_X \psi_i\rho\]
for some $\psi = \{\psi_i\} \in \overline{\mathcal{H}}(X)$. We note that this functional is also Cauchy in $\mathbb{R}$ since if we take a Cauchy sequence from $\mathcal{H}(X)$, say $\{\phi_i\}$, then the sequence $\hat{\rho}(\phi_i)$ is Cauchy in $\mathbb{R}$. Furthermore, $\underline{\hat{\rho}}$ is bounded since we have that for some $\psi \in \overline{\mathcal{H}}(X)$ $|\underline{\hat{\rho}}(\psi)| = \lim_{i \rightarrow \infty}|\hat{\rho}(\psi_i)| \leq \lim_{i \rightarrow \infty}C\Vert \psi_i \Vert = C\underline{\Vert \psi\Vert}$.
As such, we can define our extended version of the Dirichlet energy to be
$\underline{\mathcal{L}}(f) =  -2\underline{\hat{\rho}}(f) + \underline{\Vert f \Vert}_D^2$, for any $f \in \overline{\mathcal{H}}(X)$. It follows that $\underline{\mathcal{L}}$ is also bounded from below. The proof is identical to that of $\mathcal{L}$. Therefore, since $\underline{\mathcal{L}}$ is a functional on $\overline{\mathcal{H}}(X)$ we can now guarantee that \emph{if} there is an abstract element $\Phi$ for which $\underline{\mathcal{L}}(\Phi) = -C^2$ then $\Phi \in \overline{\mathcal{H}}(X)$. To obtain such an element, we need to construct a Cauchy sequence of functions in $\mathcal{H}(X)$ for which $\mathcal{L}$ attains this minimum, thereby guaranteeing the existance of an element of $\overline{\mathcal{H}}(X)$ for which $\underline{\mathcal{L}}$ attains its minimum.
\begin{lemma}
  Let $X$ be a compact, connected Riemann surface. We let $\mathcal{L}$ be the Dirichlet energy functional, and let it be bounded from below by a number $M \in \mathbb{R}$, ie $\mathcal{L} \geq M$ (which holds true by the previous section with $M=-C^2$).
  Let $\{\phi_i\}$ be a sequence of functions such that $\lim\limits_{i \rightarrow \infty}\mathcal{L}(\phi_i) = M$. Then $\{\phi_i\}$ is a Cauchy sequence in $\mathcal{H}(X)$ which defines our abstract element $\Phi \in \overline{\mathcal{H}}(X)$, such that $\underline{\mathcal{L}}(\Phi) = M$.
\end{lemma}
Note that the work in this lemma and proof follow the proposition on \cite[p.30]{notes}.
\begin{proof}
  Let $\epsilon > 0$. Then there exists some $N \in \mathbb{N}$ such that for $i,j \geq N$ we have that $\mathcal{L}(\phi_i) \leq M + \frac{1}{4}\epsilon$ and $\mathcal{L}(\phi_j) \leq M + \frac{1}{4}\epsilon$. We define a function $I:[0,1] \rightarrow \mathbb{R}$, as a function of t, with $I(0) = \mathcal{L}(\phi_j)$ and $I(1) = \mathcal{L}(\phi_i)$, which interpolates between the two giving
  \begin{align*}
    I(t) &= \mathcal{L}(t\phi_i + (1-t)\phi_j) = \Vert t\phi_i + (1-t)\phi_j \Vert^2_D - 2\hat{\rho}(t\phi_i + (1-t)\phi_j) \\
         &= \Vert \phi_j + (\phi_i - \phi_j)t \Vert^2_D - 2\hat{\rho}(\phi_j + (\phi_i - \phi_j)t) \\
         &= \Vert \phi_i - \phi_j \Vert^2_Dt^2 + 2\langle \phi_i, \phi_j\rangle_Dt + \Vert \phi_j \Vert^2_D - 2\hat{\rho}(\phi_j) - 2\hat{\rho}(\phi_i - \phi_j)t
  \end{align*}
  So we have that $I(t)$ is a quadratic polynomial. By expanding $2(I(0) - 2I(\frac{1}{2})+I(1))$ we get the coefficient of $t^2$. Considering that $I(0)= \mathcal{L}(\phi_j) \leq  M + \frac{1}{4}\epsilon$ and $I(1) = \mathcal{L}(\phi_i) \leq  M + \frac{1}{4}\epsilon$ and that for all $t$ we have that $I(t) \geq M$, since $\mathcal{L} \geq M$ for all functions, then we can apply the following approximation
  \begin{align*}
    \Vert \phi_i - \phi_j \Vert^2_D &= 2(I(0) - 2I(1/2)+I(1)) \leq 2(M + \frac{1}{4}\epsilon -2M + M + \frac{1}{4}\epsilon) = \epsilon
  \end{align*}
  Therefore $\{\phi_i\}$ is Cauchy so there exists an element $\Phi = \lim\limits_{i \rightarrow \infty}\phi_i$ such that $\underline{\mathcal{L}}(\Phi) = M$.
\end{proof}
So we have proven the existance of the minimising element $\Phi$ of $\underline{\mathcal{L}}$ in $\overline{\mathcal{H}}(X)$. But we want to understand it a bit better. As things stand, it is simply an equivalence class of Cauchy sequences with limit $\Phi$; not a very usable object. We want to instead show it is an element of $\mathcal{H}(X)$, that is, show it can be identified with an equivalence class of smooth functions (up to the addition of a constant). This implies Weyl's lemma; a theorem that states that every solution of the Poisson equation is a smooth function, which in our case, allows us to conclude our proof of the reverse direction of Theorem \ref{compactPoisson}.

Before we embark on this final part of our proof, let us first show that indeed, we are on the right track and that this minimising element does indeed give a solution to the Poisson equation. The method mirrors what we did before for $\mathcal{L}$ with the arbitrary function $g$, however, now we use $\underline{\mathcal{L}}$ and $\Phi$ instead of an arbitrary function. Recall, earlier on we \emph{assumed} that there existed a function $g$ which minimised $\mathcal{L}$. As things stand, we still dont know that. What we do know however is that we can minimise $\underline{\mathcal{L}}$ by applying it to $\Phi$. Let us compute the first variation of $\underline{\mathcal{L}}$, ie for some $f\in \mathcal{H}(X)$ computing
 \[ \delta\underline{\mathcal{L}} =  \frac{d}{d\epsilon}\biggr\rvert_{\epsilon = 0} \underline{\mathcal{L}}(\Phi + \epsilon f) = 0\]
In an argument, largely the same as before we reach the following conlcusion
\begin{align*}
  (\Phi,f)_D = \underline{\hat{\rho}}(f) \Longleftrightarrow
  \lim_{i \rightarrow \infty}\langle \phi_i,f\rangle_D = \underline{\hat{\rho}}(f) \Longleftrightarrow
  \lim_{i \rightarrow \infty}\int_X f \Delta \phi_i = \underline{\hat{\rho}}(f) 
\end{align*}
Since $\underline{\hat{\rho}}$ is a bounded linear functional, we have that the sequence of real numbers, defined by the integrals as the limit is taken to infinity is, in fact, a Cauchy sequence in $\mathbb{R}$ and so, we can precisely write that 
\begin{align*} 
  \int_X f\Delta\Phi = \underline{\hat{\rho}}(f) \Longleftrightarrow
  \int_X f\Delta\Phi - f\rho = 0 \Longleftrightarrow
  \int_X f(\Delta\Phi - \rho) = 0
\end{align*}
In other words, $\Phi$ is indeed a solution to the Poisson equation. Such a $\Phi$ is often called a \emph{Weak Solution} to the Poisson equation since it is not necessarily a smooth, or even continuous function.
Now all that is left to do, is show $\Phi$ can be identified with a smooth function. We begin by stating the infamous Weyl's Lemma.
\begin{thm}[Weyl's Lemma]\label{WeylsLemmaCompact}
  Let $X$ be a compact, connected Riemann Surface. If $\rho \in \Omega^2(X)$ such that $\int_X \rho = 0$, then a weak solution to Poissons equation $\phi \in \overline{\mathcal{H}}(X)$ in fact is an element of $\mathcal{H}(X)$, i.e. $\phi$ is a smooth function.
\end{thm}

We also want to use the following lemma, that we call the ``Local Weyl Lemma''.
\begin{lemma}[Local Weyl Lemma]\label{WeylsLemmaLocal}
  Let $U$ be a bounded region in $\mathbb{C}$ and let $\rho \in \Omega^2(U)$ be a $2$-form on $U$. Suppose $\phi \in L^2(U)$ such that, for any smooth function $h$ of compact support in $U$ we have that 
  \[\int_U \Delta h\phi = \int_U \chi\rho\]
  Then $\phi$ is smooth and satisfies the equation $\Delta \phi = \rho$.
\end{lemma}
We see that, if this lemma were to be true in a single coordinate chart and if we could somehow stitch up a function all over $X$ to have this property hold, then it implies Weyls lemma. To avoid simply copying pages from \cite[(Chapter 10)]{donaldson}, we instead refer the reader to the reference and sketch the steps involved in the argument.
\begin{sproof}\emph{(Theorem \ref{WeylsLemmaCompact} - Weyl's Lemma)}
  \hspace{1pt}
  \begin{enumerate}
    \item We begin the proof by considering $\Phi$ on a single open coordinate chart of $\mathbb{C}$ and showing that in local charts, $\Phi$ can by identified with an $L^2$ function, by taking a sequence $\phi_i$ that tends to $\Phi$ and showing that the $L^2$ norm $\Vert \phi_i - \phi_j \Vert \leq C \Vert \phi_i - \phi_j \Vert_D$ is bounded by something which converges. Thus, the limit of $\phi_i$ is an $L^2$ function.
    \item We then extend $\Phi$ to be $L^2$ over the whole of $X$, by consider a set $A$ to be the set of points $x \in X$ with the property that $\phi_i$ converges in the $L^2$ sense in a coordinate chart around $x$. This set is open by definition, so the compliment of $A$ is not open unless it is empty since $X$ is connected. Then we prove by contradiction that $A = X$ otherwise, there would be a point in the closure of $A$ that is not in $A$.
    \item We then return to a single coordinate chart and want to prove the Local Weyls lemma, as if we can show it is true, then Weyls lemma must also be true.
    \item We look to then simplify the problem, to a looking for a weak harmonic function, such that if we can prove the weak harmonic function is smooth, then it implies that our weak solution $\Phi$ is also smooth. We do this by constructing a particular harmonic function $\psi = \Phi - \phi^{\prime}$ where $\phi^{\prime}$ is smooth.
    \item We then show that the convolution $\beta(|z|)*\psi - \psi$, where $\beta$ is a bump function in a neighbourhood of the origin, vanishes outside an $\epsilon$ neighbourhood of $\text{supp}(\Delta \psi)$ if and only if $\psi$ is smooth. We use this to show that $\psi$ is smooth and hence $\Phi$ is also smooth by the Local Weyl lemma which implies that our weak solution is actually smooth on the whole of $X$.
  \end{enumerate}
\end{sproof}
%This argument is based on that from \cite[(Chapter 10)]{donaldson}. Note that henceforth, we associate to the weak solution $\Phi$, the Cauchy sequence $\{\phi_i\}$ of elements in $\mathcal{H}(X)$ (or, when considering local coordinate charts, $\mathcal{H}(U))$.
%We begin by fixing a coordinate chart $U$ of $X$. By adding suitable constants to each $\phi_i$, we can modify the value of $\int_{U} \phi dS_{U}$, where $dS_{U}$ is a fixed area form for $U$, so that the integrals are zero (to make the average value of $\phi_i$ over $U$ zero). Then, by Corollary \ref{quotedCorollary6}, we have that, for some $i,j$ the usual $L^2$ norm $\Vert \phi_i - \phi_j\Vert \leq C \Vert \phi_i - \phi_j\Vert_D$, since we can shift each $\phi_i$ by a constant to make the average value $\overline{\phi_i}=0$. Since this is a Cauchy sequence in $\mathcal{H}(U)$ then by the completeness of $L^2(U)$ under the usual norm, we have that the Cauchy sequence $\{\phi_i\}$ converges to a square-integrable function $\phi$. Hence we have identified the abstract object $\Phi$ with a square integrable function which we call $\phi$.

%So now we want to show that this same sequence $\{\phi_i\}$ converges to a $L^2$ function on the whole of $X$ rather than on just $U$. Since we just showed that in any given coordinate chart, $\{\phi_i\}$ converges to a $L^2$ function, we define the set $A \subset X$, to be the set of points $x \in X$ with the property that there exists a coordinate chart around $x$ such that $\{\phi_i\}$ converges to $\phi$ under the $L^2$ norm. Then $A$ by the above discussion is non-empty. Since $X$ is connected, then we have that the compliment of $A$ is not open unless $A=\emptyset$, which cannot be. Therefore, either $A=X$ and we are done, or there exists a $y \in X$ such that $y \in \bar{A}$, the closure of $A$, but not in $A$. If that were so we could find a coordinate chart $U^{\prime}$ about $y$ such that for some real numbers $c_i$, we can subtract them from each $\phi_i$ such that $\phi_i - c_i$ converges to the $L^2$ limit in $U^{\prime}$. However, now we have that there is a point $x \in A \cap U^{\prime}$ such that both $\phi_i$ and $\phi_i - c_i$ converge to $\phi$ under the usual norm. Hence, $c_i$ must tend to $0$ as $i \rightarrow \infty$ giving that $y \in A$, a contradiction. Hence $A=X$ and we have shown that we can extend $\phi$ to $L^2(X)$. But now we want to show $\phi \in \Omega^0(X)$. To do so we need a local version of Weyl's lemma.


%Since we can stitch together locally smooth solutions using a partition of unity subordinate to our choice of finite cover of coordinate charts of $X$, as we saw earlier, we will focus on proving the statement of the local Weyl's lemma, after which, the full Weyl's lemma follows. First we will try to simplify the problem to the case when $\rho = 0$. 

%To do this, we will show that $\phi$ is smooth on $U^{\prime}$, the interior of $U$, where we suppose that for some $\epsilon > 0$, we have an $\epsilon$ neighbourhood of $U^{\prime}\subset U$. We then find a $\rho^{\prime}$ which is equal to $\rho$ on a neighbourhood of the closure of $U^{\prime}$ and of compact support in $U$. If we can find a smooth solution $\phi^{\prime}$ of the equation $\Delta \phi^{\prime} = \rho^{\prime}$ over $U$, then $f = \phi - \phi^{\prime}$ will be a weak solution to $\Delta f = 0 $ in $U^{\prime}$. So, our strategy will boil down to showing that $f$ is smooth, which then implies that $\phi$ must be also be smooth.

%So, we suppose $f$ is a weak solution to $\Delta f = 0$ on $U$. This means that $f$ is a harmonic function, and as such, the mean value theorem for harmonic functions applies to it. Recall, that Lemma \ref{MVT} stated that the value of a harmonic function at the center of a circle in the plane is equal to the average value of the function on the circle. Note that the bump function $\beta$ has integral $2\pi\int_0^{\infty}x\beta(x)dx = 1$. We rewrite our bump function $B(z)=\beta(|z|)$, giving us that $B$ is also a smooth function that has integral $1$ over the whole of $\mathbb{C}$ since the integral is independent of the angular coordinate and hence, $\int_{\mathbb{C}}B(z)dS=\int_0^{2\pi}\int_0^{\infty}B(r,\theta)rdrd\theta = 2\pi\int_0^{\infty}r\beta(r)dr = 1$. Now if we have a smooth, harmonic function $g$ on a neighbourhood of the $\epsilon$ disc centred around the origin. Then we have that the convolution $(B*g)$ gives
%\begin{align*}
%  \int_{\mathbb{C}}B(0-z)g(z)dS_z = \int_0^{2\pi}\int_0^{\infty}r\beta(r)g(r,\theta)drd\theta &= \int_0^{\infty}r\beta(r)\left(\int_0^{2\pi}g(r,\theta)d\theta\right) dr \\
 % &= \left(2\pi\int_0^{\infty}r\beta(r) dr\right) g(0) \\
 % &=g(0)
%\end{align*}
%So we have that $(B*g)(0) = g(0)$. Since we can shift the coordinate $z$ by any complex number $w$ anywhere, then this convolution is translation invariant (this can be seen by for some $a \in \mathbb{C}$ setting $w = z + a$). We can summarise the above in the following proposition.
%\begin{prop}
%  Let $\psi$ be a smooth function on $\mathbb{C}$, and let $\Delta \psi$ be supported in a compact set $J \subset \mathbb{C}$. Then $B*\psi - \psi$ vanishes outside an $\epsilon$ neighbourhood of $J$ for some $\epsilon > 0$.
%\end{prop}

%Though it is not immediately clear, if we compute the $L^2$ inner product $\langle \Delta \psi, (B*\psi) - \psi \rangle$ (with $B$ and $\psi$ as in the proposition), then we get $-\langle \Delta \psi, \psi \rangle + \langle \Delta \psi, B*\psi \rangle$ which equals zero outside of $J$ since $\Delta \psi$ has as support $J$ and as we showed above, in an $\epsilon$ neighbourhood around $J$, $B$ has its support, so both inner products are defined and are zero outside of $J$ but are equal within $J$.  

%Therefore we can use this proposition to help us show that our $f$ is smooth, since, if $f$ were smooth on $U$, then $B*f = f$ on the interior set $U^{\prime}$. Additionally, by the properties of convolutions, the convolution of any $L^2$ function with $B$ is smooth on $U$ since, $B$ itself is smooth. This gives us the condition we want; proving the smoothness of $f$ in $U^{\prime}$ is the equivalent to showing that $B*f = f$ at all points in $U^{\prime}$. Simply put, if we can show $B*f - f = 0$ on $U^{\prime}$, then $f$ must be smooth. 

%We begin by considering the inner product of $f-B*f$ with a smooth function, $\chi$, of compact support in $U^{\prime}$, \[\langle \chi, f-B*f\rangle = \int_{U^{\prime}} (\chi)(f-B*f)dS = 0 \] for any choice of area form $dS$ on $U$. 

%Let $h = V*(\chi-B*\chi)$ where $V(z) = \frac{1}{2\pi}\log(|z|)$, the Newtonian potential. We want to show that $h$ has compact support in $U^{\prime}$. We want to show that we can use this $h$ as the $h$ in the hypothesis of the local Weyl's Lemma, ie, show it has compact support in $U$. Expanding this gives us that $h = V*\chi - V*B*\chi$. Since $V*\chi$ is a smooth function on $\mathbb{C}$ and by the lemma on the properties of convolutions of the Laplacian with the Newtonian potential (Lemma \ref{ConvProperties}), we have that $\Delta V*\chi = \chi$. Hence, $\Delta V*\chi$ must vanish outside the of the support of $\chi$ and so $B*V*\chi = V*\chi$ outside the $\epsilon$ neighbourhood of the support of $\chi$. Therefore, $h$ has compact support, contained in $U$, since $\chi$ has by hypothesis. Using the hypothesis of the local Weyl's lemma we have that $\langle \Delta h, f\rangle = 0$. Now we note that $\Delta h = \Delta V*(\chi - B*\chi) = \chi - B*\chi$, again since both $B$ and $\chi$ are compactly supported. Hence, we have that $\langle \chi - B*\chi, f \rangle = 0 $ which when we apply the triple product identity, gives us $\langle \chi, f - B*f \rangle = 0$ which implies that $f-B*f = 0$ on in $U^{\prime}$. Hence, $f$ is indeed a smooth solution to $\Delta f = 0$, finally implying that $\phi$, our $L^2$ function, is in fact, a smooth solution to the Poisson equation.
%To then extend this solution over the whole of $X$, we must pick a finite collection of coordinate charts, and pick a partition of unity subordinate to this cover. Once we identify this solution with a $2$-form on $X$, by fixing an area form, then we proceed as in the second part of the proof of Proposition \ref{PartitionOfUnity}. \qed

\section{Cohomology groups of compact Riemann surfaces}
The implications of having solved Poissons equation are not immediately obvious, other than the statement of Theorem \ref{compactPoisson}. So how can we use the guaranteed existence of such a function $f$ for a given $2$-form to help us classify connected, compact Riemann surfaces? How \emph{exactly} does Poissons equation help us solve the Uniformisation theorem? In the compact, connected case, we get the following theorem of relations between the Dolbeault and de-Rham cohomology groups.

\begin{thm}
  Let $X$ be a compact, connected Riemann surface. Then we have the following isomorphims:
  \begin{enumerate}
    \item The induced map $\sigma: H^{1,0}(X) \rightarrow \overline{H^{0,1}}(X)$, induced by the mapping $\alpha \mapsto \overline{\alpha}$ is an isomorphism
    \item The bilinear map $B:H^{1,0}(X)\times H^{0,1}(X) \rightarrow \mathbb{C}$ defined by $B(\alpha, \tilde{\theta})=\int_X \alpha \wedge \theta$ gives an isomorphism between $H^{0,1}(X)$ and the dual space $(H^{1,0}(X))^*$
    \item Define the mapping $\mu:H^{1,0}(X) \times H^{0,1}(X) \rightarrow H^1(X)$ given by $\mu(\alpha, \beta) = i(\alpha) + \overline{i(\sigma^{-1}(\beta))}$, where $i:H^{1,0}(X) \rightarrow H^1(X)$ is the inclusion map, defined by mapping a holomorphic $1$-form to its cohomology class in $H^1(X)$. The map $\mu$ defines an isomorphism.
    \item The map $\nu:H^{1,1}(X) \rightarrow H^2(X)$ defined as the natural inclusion of $\im(\overline{\partial}:\Omega^{1,0}(X)\rightarrow \Omega^2(X))$ in $\im(d:\Omega^1(X)\rightarrow \Omega^2(X))$ defines an isomorphism.
  \end{enumerate}
\end{thm}
\begin{sproof}
  We do not prove these all here, though we do prove that the first map is surjective, as this uses the fact that we can solve the Poisson equation on $X$. Given a class $\tilde{\theta} \in H^{0,1}(X)$, we want to find an element  $\theta^{\prime} \in \Omega^{0,1}(X)$ such that $\theta^{\prime} = \theta + \overline{\partial}f$ with $\partial \theta^{\prime} = 0$, since this would mean that $\overline{\theta^{\prime}}$ is a holomorphic $1$-form and that $\tilde{\theta} = -\sigma(\overline{\theta^{\prime}})$. Hence, we need to solve the equation $\partial\overline{\partial}f = -\partial \theta$, as this holds true when $\partial\theta^{\prime}=0$. Note that $\partial\overline{\partial} = \frac{i}{2}\Delta$, is just the Laplacian in disguise, so we can indeed solve this equation, given that $\int_X\partial \theta = 0$. But $\int_X\partial \theta = \int_{\partial X}\theta = 0$ by Stokes theorem. Thus, $\sigma$ is surjective. Injectivity can be proven by composing $\sigma$ with the bilinear map $B$, giving a positive definite bilinear form, thus implying $\sigma$ is injective. 
  The other proofs follow similarly.
\end{sproof}
These ismorphisms provide concrete relations between the de-Rham and Dolbeault cohomology groups of our connected, compact Riemann surface. But do the Dolbeault cohomology groups measure in particular? We notice that the third isomorphism, implies that the cohomology groups $H^{1,0}(X)$ and $H^{0,1}(X)$ not only have the same dimension, but when combined with the first isomorphism, we get that each share half the value of the dimension of $H^1(X)$. We discussed earlier that for a connected, compact, genus $g$ smooth surface, $\Sigma_g$, we had that $H^1(\Sigma_g)\cong\mathbb{R}^{2g}$. Therefore, we have that $\dim_{\mathbb{R}}H^{1,0}(X)=\dim_{\mathbb{R}}H^{0,1}(X)=g$. So our Riemann surface $X$ can be classified by genus. Or can it? Cohomology unfortunately is not a strong enough condition, as two cohomological Riemann surfaces, \emph{may} not be conformally equivalent (such as complex tori of different modulus). In Chapter \ref{RSChapter} we found a result that gave us that we had a conformal equivalence between a compact, connected Riemann surface $X$ and $\widehat{\mathbb{C}}$ if we could construct a meromorphic function on $X$ such that it has a single simple pole. The following proposition generalises this result by giving us which meromorphic functions exist with simple poles on a genus $g$ Riemann surface. This implies that the genus is not strictly a topological property of our surface, but has some underlying effect on the Riemann surface structures possible on such a surface.
\begin{prop}\label{meroFunctionOnGenusGSurface}
  Let $X$ be a connected, compact Riemann surface with $H^{0,1}(X)$ of finite dimension $g$. Then, given any distinct $g+1$ points on $X$, $p_1,\ldots, p_{g+1}$, there exists a meromorphic function on $X$ with a simple pole at some (perhaps all) of the points $p_1,\ldots, p_{g+1}$.
\end{prop} 
\begin{proof}
  We begin by considering a single point $p \in X$. This implies, by hypothesis, that we have a genus $0$ Riemann surface. We now pick a coordinate chart around $p$ such that $p$ maps to $0$, and let $U$ be a small disc encircling $p$. Given an $\epsilon > 0$, we define a bump function $\beta$ on $U$ and define a function, $f(z) = \beta(z)z^{-1}$ and note that whilst $f$ is defined on $U$, we can extend $f$ by $0$ to the whole of $X$. We note further that $f$ is holomorphic in the region $|z| < \epsilon/4$ since $f(z) = z^{-1}$ in this region. We lose this fact as soon as $\beta$ stops being constant however since we can't impose that $\beta$ decays to zero holomorphically. We now construct a useful $(0,1)$-form on $X$ from $f$, namely $A=\overline{\partial}f(z) = \overline{\partial}(\beta(z))z^{-1}$.
  By definition, $A$ is an element of $\Omega^{0,1}(X\setminus \{p\})$. We would like to extend $A$ over the whole of $X$ and undertsand this extensions' cohomology class. Namely, we want to understand the class $[A] \in H^{0,1}(X)=\Omega^{0,1}(X)/\im(\overline{\partial}:\Omega^0(X) \rightarrow \Omega^{0,1}(X))$. \newline
  We see from the definition of $H^{0,1}(X)$ that we can find a smooth function $g$ on $X$ such that $\overline{\partial}(g) = A$ if and only if $[A] = 0$.  We see that $A$ can be easily extended over the whole of $X$. In the annulus $\epsilon /4 < |z| \leq 3\epsilon /4$, we see the $A$ has some value corresponding to the rate of decay of $\beta$. So let us consider when $|z| \leq \epsilon /4$. In this case, we see that $\beta$ is constant so $\overline{\partial}\beta = 0$ for all $|z| \leq \epsilon /4$. Thus, the limit of $A$ as $z$ approaches $0$ vanishes and so we have extended $A$ over the whole of $X$. Thus, we have that because $X$ is a genus $0$ surface, $H^{0,1}(X)=0$ and, $[A] = 0$, and so there exists a smooth function $\psi \in \Omega^0(X)$ such that $\overline{\partial}\psi = A$. Note that, if we multiply the function $f$ by some $\lambda \in \mathbb{C}$ then our $(0,1)$-form $A$ becomes $\lambda A$ and hence, any associated cohomology class becomes $\lambda[A]$.

  Now, we let $\phi$ be a function defined as $\phi = \psi + \lambda f$ for some $\lambda \in \mathbb{C}$. $\psi$ is defined to be smooth over all of $X$ and $f$ is smooth on $X\setminus \{p\}$ so $\phi$ is also smooth on $X \setminus \{p\}$. Since we have that $\phi$ is smooth, if we can show that this $\phi$ is holomorphic near $p$ then it is precisely the meromorphic function we need. We know that $f$, on the punctured disc $\overline{B_{\epsilon/4}(0)} \setminus \{0\}$, is holomorphic, therefore so is $\lambda f$. Therefore, if we can show that $\psi$ is holomorphic on this disc, then $\phi$ will be holomorphic on the punctured disc. But we know that on $X$, we have $\overline{\partial}\psi = A$ and except for in the annulus $\epsilon / 4 < |z| \leq 3\epsilon / 4$ we have that $A=0$. In particular, on the disc of radius $\epsilon / 4$ we have that $A=0$ and so $\overline{\partial}\psi = 0$. Hence, $\psi$ is holomorphic and so it follows that in this punctured disc, $\phi$ is holomorphic. Thus, in this disc around $p$, the function $\phi$ is meromorphic, with a pole at $p$ and smooth over $X$, as required.

  We can now generalise this construction, whereby for $g+1$ points $p_1,\ldots, p_{g+1}$, we can repeat the exact same procedure, creating functions $f_1,\ldots,f_{g+1}$ such that they extend by zero over $X$ and have a single, simple pole at the points $p_1,\ldots,p_{g+1}$ respectively. Then we can define forms $A_i$ and show that since the cohomology group $H^{0,1}(X) \cong \mathbb{R}^g$ (implying it is $g$ dimensional), then the linear combination $\lambda_1[A_1]+\ldots+\lambda_{g+1}[A_{g+1}] = 0$, is a linear dependency, implying that the equality to zero is had even if not all $\lambda_i = 0$. Then we proceed as before with this linear combination equalling zero, rather than with $[A]=0$ as we did in the single point case, thereby constructing a meromorphic function on $X$ with $g+1$ simple poles. 
\end{proof}
So the existance of the linear dependency of the cohomology classes of $(0,1)$-forms over their respective poles, gives a link between the genus of $X$ and the number of poles a meromorphic function must have on $X$. The number of points which have a pole over them correspond to which of the $\lambda_i \neq 0$ in the linear dependency. This result is incredibly powerful as it turns the topological notion of a genus into an invarient of connected, compact Riemann surfaces. It means two compact Riemann surfaces of different genus \emph{cannot} be confomrally equivalent. Therefore, we have the following.
\begin{cor}\label{ClassificationByGenus}
  Let $X$ be a connected, compact Riemann surface. 
  \begin{enumerate}
    \item If $X$ has genus $0$, then $X$ is conformally equivalent to the Riemann Sphere. 
    \item If $X$ has genus $1$, then $X$ is conformally equivalent to a complex torus.
    \item If $X$ has genus $n$, then $X$ is conformally equivalent to an orientable surface of genus $n$.
  \end{enumerate}
\end{cor}
Note especially the first point in Corollary \ref{ClassificationByGenus}: If we have a connected, simply connected, compact Riemann surface, it is conformally equivalent to the Riemann sphere $\widehat{\mathbb{C}}$. Thus, crucially, we have classified all connected, simply connected, compact Riemann surfaces.

\section{Poissons equation on simply connected, non-compact Riemann surfaces}
Before we move onto the uniformisation theorem, we need to briefly also discuss the case of connected, simply connected, non-compact Riemann surfaces. At the end of Chapter \ref{RSChapter}, we showed that all Riemann surfaces arise as a quotient of a simply connected covering Riemann surface. We hypothesised that the three covering spaces are $\widehat{\mathbb{C}} ,\, \mathbb{C},\,\mathbb{H}$ and have hence shown that the only compact simply connected Riemann surface is $\widehat{\mathbb{C}}$, by studying compact connected Riemann surfaces generally. We now wish to fill in the gap. The loss of compactness complicates things slightly so to keep things simple, we study only simply connected non-compact Riemann surfaces.
This section is a discussion of the method of proving the following theorem rather than a formal analysis of the proof as the results are well known and referenced accordingly.
Let us state the problem of Poissons equation for connected, simply connected, non-compact Riemann surfaces.
\begin{thm}\label{NonCompactPoissons}
  Let $X$ be a connected, simply connected, non-compact Riemann surface. Then if $\rho$ is a $2$-form of compact support in $X$ with $\int_X \rho = 0$ then there is a smooth function $\phi \in \Omega^0(X)$ such that $\Delta \phi = \rho$, with $\phi$ tending to $0$ at infinity in $X$.
\end{thm}
We see that this case introduces an subtly weaker constraint on $\phi$, namely that $\phi$ tends to $0$ at infinity. By that we mean the following.
\begin{defn}[A function tending to a value at infinity on $X$]
  Let $X$ be a connected, simply connected, non-compact Riemann surface. If we have a smooth function $f \in \Omega^0(X)$, and a real number $c \in \mathbb{R}$, then we say that $f$ tends to $c$ at infinity in $X$ if, for all $\epsilon > 0$ there is a compact subset $K \subset X$ such that $|f(x) - c| < \epsilon$ for $x \notin K$.
  Conversely, we say that $f$ tends to $+\infty$, at infinity in $X$ if for all $C \in \mathbb{R}$, there is a compact set $K \subset X$ such that $f(x) > C$ for all $x \notin K$ (and we say that $f$ tends to $-\infty$ in $X$ if $-f$ tends to $+\infty$).
\end{defn}
If one thinks about our physical interpretation of the problem, this is equivalent to saying at large enough scales, the effect of an electric potential on a test charge is negligable, which is in fact a real ``boundary condition'' used in electrostatics. So it is reassuring to see such a condition appearing in our study. So what differs in the proof of the non-compact case? We let such a connected, simply connected, non-compact Riemann surface be denoted as $X$. We see that we can still construct the Hilbert space $\overline{H}(X)$, and construct functionals $\underline{\hat{\rho}}$ and $\underline{\mathcal{L}}$ for some $\rho \in \Omega^2_c(X)$ in much the same way so for compact $X$. We have that $\underline{\mathcal{L}}$ is still bounded from below and so we run through the argument of Weyls lemma to show that the minimising function is $L^2$. Locally, the solution still exists. The problem now arises when we try to understand what happens over the whole surface $X$. Now the minimising function, whilst smooth, doesnt behave in quite the same way, leading to the following proposition.
\begin{prop}\cite[(p.135, Proposition 32)]{donaldson}
  Let $\rho$ be a $2$ form of integral $0$ supported in a coordinate disk $D$. Then there is a smooth function $\phi$ on $X$ which satisfies the equation $\Delta \phi = \rho$, and a sequence of functions $\phi_i$ on $X$ which have the following properties:
  \begin{enumerate}
    \item There are real numbers $c_i$ and compact sets $B_i\subset X$ such that $\phi_i = c_i$ outside $B_i$.
    \item For any $1$-form $\alpha \in \Omega^1_c(X)$, the $L^2$ norms $\Vert(\phi - \phi_i)\alpha\Vert$ tend to $0$ as $i \rightarrow \infty$.
    \item The $L^2$ norms $\Vert d\phi -d\phi_i\Vert$ tend to $0$ as $i \rightarrow \infty$.
  \end{enumerate}
\end{prop}
The goal is then to show that outside of $X\setminus\text{supp}(\rho)$, we want to make $\phi$ tend to $0$. Equally, if we can make $\phi$ tend to a constant, then we can replace $\phi$ by $\phi - c$ and have that go to $0$ without changing the solution to the Laplace equation. The proof from here, goes to show that outside of $X\setminus\text{supp}(\rho)$, we can classify that space as being the `End' of the surface $X$. This idea, due to Freudenthal, states that and end of a (locally) connected space $X$ is an assignment to each compact subset $K \subseteq X$, a connected component of $X$ whose closure is not compact, $X\setminus K$ such that if $K\subseteq K^{\prime}$ then $X_K \subseteq X_{K^{\prime}}$. We show that connected, simply connected, non-compact Riemann surfaces have exactly one end, and then arrange our solution to tend to $0$, thus giving us our solution to Poissons equation $\phi$ that tends to $0$ at infinity. This argument is studied in depth in \cite[Chapter 10]{donaldson}, but a gentler, less formal approach is also available in \cite[p.32]{notes}.

\section{The Uniformisation Theorem}
We are finally ready to discuss the uniformisation theorem. We state this as two points, the first of which we already proved when considering the case of compact, connected Riemann surfaces. Thus, we only need to show that the existence of a solution to Poissons equation on connected, simply connected, non-compact Riemann surfaces implies the second point of the uniformisation theorem.
\begin{thm}[The Uniformisation Theorem]\label{Uniformisation}
  Let $X$ be a connected, simply connected Riemann surface. 
  \begin{enumerate}
    \item If $X$ is compact, it is conformally equivalent to $\widehat{\mathbb{C}}$, the Riemann Sphere. 
    \item If $X$ is not compact, the $X$ is conformally equivalent to either $\mathbb{C}$, the complex plane or $\mathbb{H}$, the upper half plane.
  \end{enumerate}
\end{thm}
Thus we can justifiably classify all Riemann surfaces as being quotients of $\widehat{\mathbb{C}}$, $\mathbb{C}$ or $\mathbb{H}$.
\begin{cor}
  Any connected Riemann surface is conformally equivalent to one of the following:
  \begin{itemize}
    \item The Riemann Sphere, $\widehat{\mathbb{C}}$
    \item The Complex Plane $\mathbb{C}$, the Cylinder $\mathbb{C}/2\pi \mathbb{Z}$ or the Torus $\mathbb{C}/\Lambda$, for some lattice $\Lambda \subset \mathbb{C}$
    \item The Upper Half Plane quotiented by a freely acting discrete subgroup of $\Gamma \subset PSL(2,\mathbb{R})$, $\mathbb{H}/\Gamma$ 
  \end{itemize}
  Riemann surfaces may be called Elliptic, Parabolic and Hyperbolic, depending on their universal cover.
\end{cor}
Now, we want to use the fact that we have a solution to Poissons equation on simply connected, non-compact Riemman surfaces, to prove the uniformisation theorem. We do this by showing that Poissons equation for simply connected, non-compact Riemman surfaces implies the uniformisation theorem for the non-compact surfaces.
Our strategy will be to try an find a meromorphic function on our simply connected, non-compact Riemann surface $X$, with a single simple pole, at a point $p \in X$, with this functions' imaginary part tending to $0$ at infinity on $X$.
\begin{proof} \emph{(Theorem \ref{Uniformisation} - The Uniformisation Theorem)}
We pick local coordinates about $p$, and define a $(0,1)$-form $A=\overline{\partial}\beta(z)/z$ for some bump function $\beta$ centred on $p$. Thus, $A$ is supported in an annulus around $p$. Chosing $\rho = \partial A$, then by Stokes theorem, the integral of $\rho$ on $X$ is $0$, and so, when considering our complex valued $2$-form $\rho$ as a sum of a real and real-valued imaginary $2$-form, our analysis on the Poisson equation gives us that we have a complex valued function $g$ such that the real and imaginary parts of $g$ tend to $0$ at infinity and $\partial\overline{\partial}g = \rho$. Letting $B$ the \emph{real} $1$-form $B=(A-\overline{\partial}g)+(\overline{A} - \overline{\overline{\partial}g})$. By construction, the first bracket is holomorphic, so $dB = 0$. Since $X$ is simply connected, we have that $H^1(X)=0$, and so every closed form is exact. Thus we can find a \emph{real} function $\psi$ such that $d\psi = B$. Thus, we can write $A = \overline{\partial}g + \overline{\partial}\psi$. Therefore, in an annulus about $p$, we can write that $\overline{\partial}\Big(\frac{\beta}{z} - (g+\psi)\Big) = 0$. Thus, we have a meromorphic function $f = g + \psi$ on $X$ with a simple pole at $p$ whose imaginary part tends to zero at infinity on $X$, since $\psi$ is a real function. So we have a a meromorphic function $f$ from $X$ to the Riemann sphere. Recall, a meromorphic function to the Riemann sphere is a holomorphic map. 

By considering the topology\footnote{For an in depth proof, see \cite[p.132-133]{donaldson}} of the Riemann sphere, and some subtle topological arguments relating the function $f$ and its restrictions on open sets on $\widehat{\mathbb{C}}$, we can show that $f$ is a proper map that maps our surface $X$ into an open subset $U \subset H_+\cup H_- \cup \{\infty\}$, where $H_+$ and $H_-$ denote the open upper and lower half planes in $\mathbb{C}$, injectively. Thus, we get that that $U = \widehat{\mathbb{C}}\setminus I$, for some compact subset $I \in \mathbb{C}$. Thus, $X$ is conformally equivalent to this set $U$. We know that therefore, $I$ must be connected otherwise, $U$ would not be simply connected. Thus, $I$ is either a closed interval $[a,b]$ for $a < b$ or a single point.

In the case of $I$ being a single point, we see that this is the case of $U$ being conformally equivalent to $\mathbb{C}$ as it is undoing the \emph{compactification} of the complex plane that we used in the first place to describe $\widehat{\mathbb{C}}$. We can imagine the case of $I=[a,b]$ by removing one endpoint of our closed interval, say $b$, and thus having a line coming to the point $a$ on the plane. By shifting and rotating $I$ in $\widehat{\mathbb{C}}$ which we can take $a$ to be the centre of the plane, and the line coming to $a$ be the negative real axis. Thus, we have a branch cut of the plane, which, when we compose with the map $z \mapsto z^2$, gives us a conformal equivalence to $\mathbb{H}$. Thus, we have that there are exactly $2$ simply connected, non-compact Riemann surfaces.
\end{proof}

%
%\references{Bibliography}{}{yes}
\newpage
\begin{thebibliography}{99}
\bibitem{ahlfors} L. V. Ahlfors, {\em Complex Analysis - Second Edition}, McGraw-Hill (1966)

\bibitem{conway} J. B. Conway, {\em Functions of one complex variable}, Springer-Verlag (1973)

\bibitem{chern} S. S. Chern, {\em An Elementary Proof of the Existence of Isothermal Parameters on a Surface}, Proc. Amer. Math. Soc. \textbf{6} (1955), 771-782

\bibitem{donaldson} S. Donaldson, {\em Riemann Surfaces}, Oxford University Press (2011)

\bibitem{electromagentismBook} I. S. Grant \& W. R Phillips {\em Electromagnetism - Second Edition}, John Wiley \& Sons (2011)

\bibitem{Hatchers} A. Hatcher, {\em Algebraic Topology}, Cambridge University Press (2002)

\bibitem{comfun} G. A. Jones \& D. Singerman, {\em Complex Functions}, Cambridge University Press (1986)

\bibitem{jost} J. Jost \& X. Li-Jost, {\em Calculus of Variations}, Cambridge University Press (1999)

\bibitem{algebra} K. Kendig, {\em Algebraic Curves}, Springer-Verlag (1977)

\bibitem{notes} P. Kronheimer, {\em Riemann Surfaces}, Fall 2019, Harvard University. LiveTeXed by M. Jeffs URL: \url{http://people.math.harvard.edu/~jeffs/Complex_Analysis_Class_Notes.pdf} [accessed 23 June 2021]

\bibitem{calcohomo} I. Madsen \& J. Tornehave, {\em From Calculus to Cohomology}, Cambridge University Press (1997)

\bibitem{rudin} W. Rudin, {\em Functional Analysis - Second Edition}, McGraw-Hill (1991)
\bibitem{babyRudin} W. Rudin, {\em Principles of Mathematical Analysis}, McGraw-Hill (1976)

\bibitem{spivak} M. Spivak, {\em Calculus on Manifolds}, Perseus Books Publishing L.L.C. (1998)

\bibitem{volformiste} Volume form. Encyclopedia of Mathematics. URL: \url{http://encyclopediaofmath.org/index.php?title=Volume_form&oldid=32331} [accessed 10 August 2021]

\end{thebibliography}

\end{document}
