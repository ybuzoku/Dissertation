\documentclass[a4paper,12pt]{report}
%\setcounter{tocdepth}{1}%
\usepackage{amssymb,amsmath,amsthm,cmbright}
\usepackage{amsfonts,amsmath,amssymb,graphicx}
\usepackage{bbkproject}
\usepackage{hyperref}
\theoremstyle{plain}
\renewcommand{\thefootnote}{\arabic{footnote}}
\newtheorem{thm}{Theorem}[section]
\newtheorem{lemma}[thm]{Lemma}
\newtheorem{prop}[thm]{Proposition}
\newtheorem{cor}[thm]{Corollary}
\theoremstyle{definition}
\newtheorem{defn}[thm]{Definition}
\newtheorem{example}[thm]{Example}

\subject{Mathematics} %write your degree subject here
\degree{M.Sc.} %write your degree (eg M.Sc., Ph.D.)
\thesis{dissertation} %if it's a thesis, write thesis instead
\title{A survey of Riemann surfaces and their classification, with particular focus on their geometry}
\author{Y. Buzoku}
\supervisor{Dr. B. Fairbairn} 
\department{Department of Economics, Mathematics and Statistics}%
\submissiondate{$30^{\mathrm{th}}$ September 2021}%
\institution{Birkbeck, University of London}%

%
\parindent 0pt
%
\begin{document}
\maketitle{}{}{I have found a beautiful proof, but this abstract is too short to contain it.}
%
\chapter{Introduction}

The aim of this dissertation is to investigate and gain a deeper
understanding of
Riemann Surfaces, their construction, their classification and an analysis
of the
geometry of some classes of these surfaces. Doing so will bring together
ideas
from different, seemingly unrelated fields of mathematics. Often times,
Riemann
Surfaces are studied solely due to these connections as they might not
occur in
higher dimensional analogues. We begin this dissertation with an
introduction to
Riemann surfaces as classical complex-analytical manifolds, and then
showing that
we can define them equivalently as algebraic varieties of complex curves.
We then 
study briefly the automorphisms of a particular set of such Riemann
Surfaces, and 
give a short review of topological notions that will be important in our 
classification later, including quotient spaces. We then show how can
endow a 
Riemann surface structure on such spaces, and give examples of spaces
arising in such a way, and their lifts to their universal covers.

From there we begin to classify Riemann surfaces by stating the
uniformisation 
theorem, and classifying Riemann surfaces with universal cover the Riemann
sphere. We then introduce Teichm\"{u}ller spaces and discuss Riemann
surfaces with the complex plane as universal cover.

The remainder of this dissertation will then focus on the infinite family
of Riemann surfaces with the hyperbolic plane as universal cover. A
discussion of the geometry of some compact Riemann surfaces follows, with
some anecdotal but interesting ideas introduced, such as the interplay
between cubic graphs and Riemann surfaces, naturally arising hyperbolic
Teichm\"{u}ller spaces, and Hubers Theorem relating the length spectrum of
hyperbolic Riemann surfaces and the eigenvalues of the Laplacian on said
surfaces. We conclude with a discussion of the heat kernel on the sphere,
plane and hyperbolic plane and prove the uniformisation theorem for
compact Riemann surfaces and the claim that there exist exactly two
non-compact simply connected Riemann surfaces.


\chapter{Riemann Surfaces}

This chapter shows the basic definitions from Topology, (Complex) Analysis 
and Riemann Surface Theory to define Riemann Surfaces, their mappings and 
boilerplate stuff. Discuss what is needed and constructions in detail.
I guess weird test yadda yadda 


\section{Basic Definitions}\label{bdefns}

We begin by stating the basic definitions of Riemann Surfaces and the 
analogs of some important notions from Complex Analysis in Riemann Surface
Theory.

\begin{defn}[Riemann Surface]\label{rsdefn}
A Hausdorff topological space $X$ is said to be a Riemann Surface if:
\begin{itemize}
\item There exists a collection of open sets $U_{\alpha} \subset X$, where
  $
\alpha$ 
ranges over some index set such that $\bigcup\limits_{\alpha} U_{\alpha}$
cover 
$X$.
\item There exists for each $\alpha$, a homeomorphism, called a chart map,
  $ \psi_{\alpha}\colon U_{\alpha} \rightarrow \tilde{U}_{\alpha}$, where
$\tilde{U}_{\alpha}$ is 
  an open set in $\mathbb{C}$, with the property that for all $\alpha$,
  $\beta$, the composite map $\psi_{\alpha} \circ \psi_{\beta}^{-1}$ is
  holomorphic on its 
domain of 
definition. (These composite maps are sometimes called transition maps).
\end{itemize}
We call the triple $(\{U_\alpha\},\{\tilde{U}_{\alpha}\},
\{\psi_\alpha\})$ an 
atlas of 
charts for the Riemann Surface $X$, though we also use the common
notation $(U,
\tilde{U}, \psi)$ to denote this, where $U=\{U_\alpha\}$,
$\tilde{U}=\{\tilde{U}
_{\alpha}\}$ and $\psi=\{\psi_\alpha\}$.
\end{defn}


\section{Algebraic Curves}\label{algcurv}
ALGBRAIC CURVES ARE COOOOOL MAN!

\section{Proper Discontinuous Group Actions}\label{PropDiscGrpAct}
Geometry and groups episode three, the revenge of the fundamental domain.
\subsection{Fuchsian grouppos}
Empty something.

\chapter{Calculus on Riemann Surfaces}
\section{Introduction}
To gain a deeper understanding of Riemann Surfaces from a more analytical
and geometric perspective, we wish to introduce notions from calculus to
such surfaces. This involves defining the notions of integration,
differential forms and vector fields on Riemann Surfaces. As we will see
however, these notions give rise to certain groups and vector spaces
which will be very important in the following sections, especially in the
proof of the uniformisation theorem.

\section{The Tangent and Cotangent Spaces}
We begin our study of calculus on Riemann surfaces by defining notions of
the tangent and cotangent spaces. The following definitions may be found
in Chapter 5 of \cite{donaldson} unless otherwise noted.

\begin{defn}[Smooth path on a Riemann Surface]\label{Smooth Path}
  Let $X$ be a Riemann Surface and let $\epsilon > 0$. Then we say
  $\gamma:(-\epsilon,\epsilon) \rightarrow X$ is a smooth path on $X$ if
  we can define $\frac{d\gamma}{dt}$ for all $t \in (-\epsilon,\epsilon)$.
\end{defn}

\begin{defn}[Tangent Space of a Riemann Surface]\label{TpX}
  Let $X$ be a Riemann surface. Let $p \in X$ be a point on the Riemann
  surface. We define $T_pX$, the tangent space of $X$ at
  $p$, to be the
  space of equivalence classes of smooth paths $\gamma$ through $p$ such
  that $\gamma(0)=p$. Two paths $\gamma_1, \gamma_2$ are said to be
  equivalent if $\frac{d\gamma_1}{dt}=\frac{d\gamma_2}{dt}$.
\end{defn}

\begin{defn}[Cotangent Space of a Riemann Surface]\label{T*pX}
  Let $X$ be a Riemann surface. For all $p \in X$ we define the real
  cotangent space at $p$ to be
  $T^*_pX = Hom_{\mathbb{R}}(T_pX, \mathbb{R})$ and the complex cotangent
  space at $p$ to be
  $T^*_pX^{\mathbb{C}} = Hom_{\mathbb{R}}(T_pX,\mathbb{C})$.
\end{defn}



\chapter{The Uniformisation Theorem}
\section{Compact Riemann Surfaces}
\subsection{Introduction}
We begin this section by introducing some details about the main steps
involved 
in our approach to understanding the uniformisation theorem for Riemann
surfaces. 
In particular there is a single equation on which our approach hinges.
This 
equation, though abstract, gives a powerful result for Riemann Surfaces.
We will 
also investigate this result by looking at a particular Physical
interpretation 
to try and give us some intuition into why such a statement might hold
true.
\subsection{The Main Theorem for Compact Riemann Surfaces}
We begin by stating the main theorem for compact Riemann surfaces. This
naming is 
non-standard though is adhered to by \cite{donaldson}.
\begin{thm} [The main theorem for compact Riemann surfaces]\label{MTCRS}
~\\
Let $X$ be a compact, closed Riemann Surface with $\rho \in \Omega^2(X)$. 
Then we have that $\Delta \phi = \rho \Leftrightarrow \int_X \rho = 0$,
where $ \phi \in \Omega^0(X)$ and $\phi$ is unique up to addition by a
constant.
\end{thm}
We note that our solution to the equation is a real valued function and
our $2$-form is also similarly identified with a real valued function.
We do so because if we can prove the existance of a solution for real-
valued functions and forms, then we can construct a complex-valued
solution by taking a linear combination of two real-valued functions.
Similarly, we can identify real $1$-forms with $(1,0)$-forms by mapping a
real

Whilst seemingly complicated, this theorem says three things:
\begin{itemize}
\item[1.] If $\phi$ is a solution to the equation $\Delta \phi = \rho$
then $
\int_X \rho = 0$.
\item[2.] If $\phi$ is a solution of the equation, then adding a constant
to $
\phi$ does not change $\rho$
\item[3.] If $\rho \in \Omega^2(X)$ such that $\int_X \rho = 0$ then
there exists 
a $\phi \in \Omega^0(X)$ such that $\Delta \phi = \rho$.
\end{itemize}

Points one and two are easy to prove and we will do so below:
\begin{proof} ~\\
We start by proving point $1.$ ~\\
Let $\phi \in \Omega^0(X)$. Recall that
$\Delta =2i\bar{\partial}\partial$, so 
applying this directly to $\phi$, yields a 2-form
$\rho = 2i\frac{\partial^2\phi}{\partial \bar{z}\partial z}d\bar{z}\wedge
dz = 2i \bar{\partial}\partial \phi$. 
Noting that
$d(\partial \phi) = \partial^2 \phi + \bar{\partial}\partial\phi =
\bar{\partial}\partial \phi $, we can rewrite $\rho$ as
$\rho = 2id(\partial \phi)$ as $\partial^2 = \bar{\partial}^2 = 0$.

Integrating $\rho$ over $X$, gives $\int_X \rho = 2i\int_X
\bar{\partial}\partial \phi = 2i \int_X d(\partial \phi) = 2i
\int_{\partial X}\partial \phi$
, where we have changed our domain of integration from $X$ to its
boundary $\partial X$ by Stokes' theorem. However, a closed, compact
Riemann surface has no boundary, so such an integral would be over an
empty set and as such $\int_X \rho = 2i\int_X d(\partial \phi) =
2i\int_{\partial X} \partial \phi = 0$. ~\\

The proof of point 2 is similarly elementary. ~\\
Let $\phi$ and $\psi$ be such that $\Delta \phi = \rho$ and $\Delta \psi
= \rho$. 
Letting $\theta = \psi - \phi$ allows us to write this as $\Delta \theta
= 0$, 
implying that $\theta$ is a harmonic function.

Since $\psi,\phi \in \Omega^0(X)=C^{\infty}(X,\mathbb{R})$, $\theta$ is a
real-
valued function.
Recall, that we also defined the notion of the Dirichlet norm. We can use
the 
fact that it is a norm to show that $\theta$ is a constant. ~\\
Consider $||\theta||_D = ||d\theta|| = 2 \int_X (\frac{\partial
\theta}{\partial 
z})^2 dx \wedge dy = \int_X \theta \Delta \theta = 0$.
Hence, $d\theta = 0$. Since $\theta$ is real-valued, this implies that 
$\theta = constant$. ~\\
This concludes our proof of points $1$ and $2$. 
\end{proof}
Whilst points $1$ and $2$ were quite easy to prove, it is the final point
to which we now turn our attention to.
To try and understand why this, seemingly abstract, result is true, we
momentarily turn to the theory of Electromagnetism. Maxwells first
equation posits that for a region of space $\Omega$ with boundary, $
\int_\Omega \rho dV = \int_{\partial \Omega} \nabla \cdot E dS $. We see
this as a consequence of Stokes theorem.


%
%\references{Bibliography}{}{yes}
\begin{thebibliography}{99}
\bibitem{donaldson} S. Donaldson, Riemann Surfaces
\bibitem{smith} J. Smith. {\em My favourite Theorems}, Madeup University
Press (2026).
%
\bibitem{website} MacTutor History of Mathematics Archive, at
https://mathshistory.st-andrews.ac.uk/ [accessed 9 May 2030]
\end{thebibliography}
\end{document}
