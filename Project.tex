\documentclass[11pt]{report}
\usepackage{amssymb,amsmath,amsthm}
\usepackage{amsfonts,amsmath,amssymb,graphicx}
%\usepackage{cmbright} %sans serif font if uncommented
\usepackage[parskip]{bbkproject}
%\usepackage{hyperref}

\newtheorem{thm}{Theorem}[section]
\newtheorem{lemma}[thm]{Lemma}
\newtheorem{prop}[thm]{Proposition}
\newtheorem{cor}[thm]{Corollary}
\newtheorem*{thm*}{Theorem}
\theoremstyle{definition}
\newtheorem{defn}[thm]{Definition}
\newtheorem{example}[thm]{Example}
\newtheorem*{problem}{Problem}

%%%% These lines are only needed for the example text which follows
\DeclareMathOperator{\Fix}{Fix}
\DeclareMathOperator{\orb}{orb}




%%%%%%%%%%%%%%%%%%%%%%%%%%%%%%%%%%%%%%%%%%%%%%
%%%%%%%%%%%%%%%%%%%%%%%%%%%%%%%%%%%%%%%%%%%%%%%%
%%%%%%%%%%%%%%%%%%%%%%%%%%%%%%%%%%%%%%%%%%%%%%%%%
%YOU FILL IN THESE AS APPROPRIATE
\subject{Mathematics} %write your degree subject here
\title{The Uniformisation Theorem of Riemann surfaces}
\author{Y. Buzoku}
\supervisor{Dr. B. Fairbairn} %
\submissiondate{$3^{\mathrm{rd}}$ September 2021}%




\begin{document}
\maketitle
\tableofcontents

\newpage
\begin{abstract}
This dissertation is an in-depth discussion of Riemann surfaces and their classification by ways of the uniformisation theorem. The dissertation is split into two parts; the first being a discussion of Riemann surfaces themselves, methods of their construction, and interesting properties and symmetries that they exhibit. The second section focusses on providing an in-depth account of the proof of the classification of Riemann surfaces. It is an analytical treatment motivated by a problem that can be stated using the theory of electrostatics. We classify compact Riemann surfaces in detail, then only sketching the case for non-compact, simply connected Riemann surfaces (since the arguments are similar and only differ by minor technical aspects). The uniformisation theorem then follows naturally. The majority of the work presented generally follows the proofs presented in \cite{donaldson} and \cite{notes}, however in some cases heavily modified to account for our novel and different interpretation of the problem, with many details explicitly filled in to make for an easier reading. 
\end{abstract}
\declaration % Includes the declaration - don't delete this!
\chapter{Introduction}


\chapter{Elementary definitions and notation}
We begin by setting up the notation and tools we will be using from Topology,Real and Complex Analysis. Though this dissertation focuses on Riemann Surfaces,
the climactic result of this dissertation, the uniformisation theorem, is approached from the point of view of Harmonic analyis; the analysis of harmonic functions on sets in $\mathbb{R}^n$. Therefore a review of some results and definitions from real and complex analysis are required. For most of these results, a reference to a proof will be provided, unless where the proof of the theorem might be helpful to our analysis, in which case we shall prove the result here.
\section{An overview of the notations and conventions taken in this dissertation}
$\mathbb{F}=\mathbb{C} \wedge \mathbb{R}$
\section{An overview of definitions and results from Analysis and Topology}
\subsection{Topological results}
\subsection{Real Analysis}
\subsection{Complex Analysis}
\subsection{Functional Analysis}


\chapter{Riemann Surfaces}

Intro to Riemann Surfaces. Talk about $\log(z)$ and it as a motivational entry into the subject.


\section{Basic Definitions}\label{bdefns}

We begin by defining and constructing some Riemann surfaces.

\begin{defn}[Riemann Surface]\label{rsdefn}
A Hausdorff topological space $X$ is said to be a Riemann Surface if:
\begin{itemize}
\item There exists a collection of open sets $U_{\alpha} \subset X$, where
  $
\alpha$ 
ranges over some index set such that $\bigcup\limits_{\alpha} U_{\alpha}$
cover 
$X$.
\item There exists for each $\alpha$, a homeomorphism, called a chart map,
  $ \psi_{\alpha}\colon U_{\alpha} \rightarrow \tilde{U}_{\alpha}$, where
$\tilde{U}_{\alpha}$ is 
  an open set in $\mathbb{C}$, with the property that for all $\alpha$,
  $\beta$, the composite map $\psi_{\alpha} \circ \psi_{\beta}^{-1}$ is
  holomorphic on its domain of definition, with a holomorphic inverse map. (These composite maps are sometimes called transition maps).
\end{itemize}
We call the triple $(\{U_\alpha\},\{\tilde{U}_{\alpha}\},
\{\psi_\alpha\})$ an 
atlas of 
charts for the Riemann Surface $X$, though we also use the common
notation $(U,
\tilde{U}, \psi)$ to denote this, where $U=\{U_\alpha\}$,
$\tilde{U}=\{\tilde{U}
_{\alpha}\}$ and $\psi=\{\psi_\alpha\}$.
\end{defn}
This definition can be quite difficult to work with. Luckily, we can construct some very nice examples using this definition. Let us begin with our first example.
\begin{example}[The Complex Plane]
  Perhaps, unsurprisingly, the complex plane $\mathbb{C}$ can be defined as a Riemann surface. To see this, we pick a cover for the complex plane, say $U_{\alpha}$, and just identity map each set back into the complex plane. Hence the triple $(U_{\alpha},U_{\alpha},\phi_{\alpha})$ is an atlas for $\mathbb{C}$ where $\phi_{\alpha}=Id_{\alpha}$.
\end{example}
In a similar way we can also define the upper half plane $\mathbb{H}$, as we can cover the set $\{ z \in \mathbb{C} \colon Im(z)>0\}$ with an open cover, and identity map them into the upper half plane on $\mathbb{C}$. Though this seems like a natural subspace of $\mathbb{C}$, as we will see, the space $\mathbb{H}$ can produce some of the most interesting examples of Riemann surfaces. Finally, a perhaps more interesting example is that of the Riemann sphere. As we will shortly see, the Riemann sphere comes up in a variety of special ways and is perhaps the nicest and simplest to understand Riemann surface barring $\mathbb{C}$ itself.

\begin{example}[The Riemann sphere]
  The Riemann sphere, commonly denoted as $\hat{\mathbb{C}}$, can be defined as a topological space $\mathbb{C} \cup \{\infty\}$, where all sequences of complex numbers that do not converge to a value in $\mathbb{C}$ can be said to converge at this additional point, denoted by $\infty$, otherwise called the point at infinity (note this implies our space is sequentially compact and hence compact). Let us construct an atlas of charts for the Riemann sphere as this will show us that the Riemann sphere is indeed a Riemann surface. The following system of constructing charts on a sphere is known as stereographic projection. Let $U=\{U_1,U_2\}$, with $U_1=\{z \in \mathbb{C} : |z| < 2\}$ and $U_2=\{z \in \mathbb{C} : |z| > \frac{1}{2}\}\cup \{\infty\}$. Let $U_1=\tilde{U}_1=\tilde{U}_2$ and we let $\psi_1:U_1 \rightarrow \tilde{U}_1$, be the identity map. However, we let $\psi_2:U_2 \rightarrow \tilde{U}_2$ with $\psi_2(z)=\frac{1}{z}$ and $\psi_2(\infty)=0$. Both of these maps are clearly holomorphic so all that is needed is to consider if the transition maps composed of these two maps are also holomorphic.
  It turns out that both $\psi_1 \circ \psi_2 ^{-1}$ and $\psi_2 \circ \psi_1 ^{-1}$ are the map $z \mapsto \frac{1}{z}$, which is also holomorphic on its domain of definition $\{z\in \mathbb{C} : \frac{1}{2} < |z| < 2\}$ and so we have our atlas for the Riemann sphere proving it is a Riemann surface.
\end{example}

We now show quite a unique property of Riemann surfaces, that is that their transition maps have a positive Jacobian. This restricts the types of surfaces which can arise as Riemann surfaces to being only the orientable surfaces. 
\begin{lemma}[Riemann surfaces are orientable]
  Let $X$ be a Riemann surface. Then, the determinant of the Jacobian of the transition maps all have positive determinant.
\end{lemma}
\begin{proof}
  Pick an arbitrary transition map between arbitrary open sets on $X$, with a non-empty intersection. Let us call this map $f$. By the definition of transition maps, $f$ is holomorphic on its domain of definition. Hence we can write for any point $z=x+iy$ in the intersection, $f(z) = f(x+iy)=u(x,y)+iv(x,y)$ for some real valued functions $u,v$, with $u,v$ satisfying the Cauchy-Riemann equations $\frac{\partial u}{\partial x} = \frac{\partial v}{\partial y}$ and $\frac{\partial u}{\partial y} = -\frac{\partial v}{\partial x}$. 
  Thus, the determinant of the Jacobian $J[f]$ of our transition map $f$ is 
  $\text{det}(J[f])=\begin{vmatrix}
    \frac{\partial u}{\partial x} & \frac{\partial u}{\partial y} \\
    \frac{\partial v}{\partial x} & \frac{\partial v}{\partial y} \\
  \end{vmatrix} = \frac{\partial u}{\partial x}\frac{\partial v}{\partial y} -  \frac{\partial v}{\partial x}\frac{\partial u}{\partial y}= (\frac{\partial u}{\partial x})^2 + (\frac{\partial u}{\partial y})^2 \geq 0$, where the last equality holds by the Cauchy-Riemann equations. This is strictly positive barring the fairly uninteresting case of $f=0$, which means that the transition map (and hence the chart maps), are the zero map. However, we discard this case as the inverse map would not be holomorphic (ie $f^{-1}(0)$ would not be a single point).
\end{proof}

So, we have used atlases to construct a few examples of Riemann surface and even prove an interestng property of theirs. However, one can ask, are there any other ways of constructing Riemann surfaces? We see that, it is perhaps, quite difficult to move beyond these examples with only this data. Luckily, Riemann surfaces arise in a number of different, but equivalent, ways, as we shall see.

\section{Algebraic Curves}\label{algcurv}
We begin with the case of algebraic curves in $\mathbb{C}^2$. Certain curves in $\mathbb{C}^2$ define covers the Riemann sphere, often with multiple sheets, except at a certain set of presecribed points. One imagines this process as if one were to cut open a sphere, and stretch the cut sphere over another sphere, wrapping the cover around the sphere a number of times, and then sewing it up along the cut again. We see that there would be a problem at the poles, where material would bunch up and misbehave. This is an example of such a "prescribed" point. We investigate below.

\section{Properly Discontinuous Group Actions}\label{PropDiscGrpAct}
\subsection{Automorphisms of Riemann surfaces}
Geometry and groups episode three, the revenge of the fundamental domain.
\subsection{Fuchsian groups and Tessellations}
Empty something.

\section{Attempting to classify some Riemann surfaces}
\subsection{The Riemann Sphere}
\subsection{Quotients of the Complex Plane}
\subsubsection{The Cylinder}
\subsubsection{The Torus}
\subsection{Moduli space of Tori}

\chapter{Calculus on Riemann Surfaces}
\section{Introduction}
To gain a deeper understanding of Riemann Surfaces from a more analytical
and geometric perspective, we wish to introduce notions from calculus to
such surfaces. This involves defining the notions of integration,
differential forms and vector fields on Riemann Surfaces. As we will see
however, these notions give rise to certain groups and vector spaces
which will be very important in the following sections, especially in the
proof of the uniformisation theorem. This section will be a brief introduction
to doing calculus on Riemann surfaces, complete with definitions and examples to gain a good understanding but with the proofs to most theorems being referenced.

\section{The Tangent and Cotangent Spaces}
We begin our study of calculus on Riemann surfaces by defining the notions of
the tangent and cotangent spaces. This section takes the work found in chapter 5 of \cite{donaldson}, chapters 2, 3, 8, 9 and 10 of \cite{calcohomo} and chapters 4 and 5 of \cite{spivak}, to provide a comprehensive set of consistant definitions and examples that should give the reader a good overview of this broad subject.

\begin{defn}[Smooth path on a Riemann Surface]\label{Smooth Path}
  Let $X$ be a Riemann Surface and let $\epsilon > 0$. Then we say
  $\gamma:(-\epsilon,\epsilon) \rightarrow X$ is a smooth path on $X$ if
  we can define $\frac{d\gamma}{dt}$ for all $t \in (-\epsilon,\epsilon)$.
\end{defn}

\begin{defn}[Tangent Space of a Riemann Surface]\label{TpX}
  Let $X$ be a Riemann surface. Let $p \in X$ be a point on the Riemann
  surface. We define $T_pX$, the tangent space of $X$ at
  $p$, to be the
  space of equivalence classes of smooth paths $\gamma$ through $p$ such
  that $\gamma(0)=p$. Two paths $\gamma_1, \gamma_2$ are said to be
  equivalent if $\frac{d\gamma_1}{dt}=\frac{d\gamma_2}{dt}$.
\end{defn}

\begin{defn}[Cotangent Space of a Riemann Surface]\label{T*pX}
  Let $X$ be a Riemann surface. For all $p \in X$ we define the real
  cotangent space at $p$ to be
  $T^*_pX = Hom_{\mathbb{R}}(T_pX, \mathbb{R})$ and the complex cotangent
  space at $p$ to be
  $T^*_pX^{\mathbb{C}} = Hom_{\mathbb{R}}(T_pX,\mathbb{C})$.
\end{defn}

\section{The Alternating Algebra and Differential forms}
We state these definitions in such a way that will be useful to us, however they all generalise to the case of $n$-dimensional manifolds.
\begin{defn}[Alternating Space]\label{AltSpc}
  Let $V$ be a $\mathbb{F}$ vector space. A $k$-linear map $\omega:V^k\rightarrow \mathbb{R}$ is said to be alternating if $\omega(v_1,\ldots,v_k)=0$ when $v_i=v_j$ where $i\neq j$. The vector space of alternating, $k$-linear maps is denoted by $\text{Alt}^k(V)$.
\end{defn}
\begin{defn}[Exterior Product]
  Let $V$ be a $\mathbb{F}$ vector space. Let $\omega_1 \in \text{Alt}^p(V)$ and $\omega_2 \in \text{Alt}^q(V)$. We define the exterior product $\wedge : \text{Alt}^p(V) \times \text{Alt}^q(V) \rightarrow \text{Alt}^{p+q}(V)$ as 
  \begin{align*}
    &(\omega_1 \wedge \omega_2)(v_1,\ldots,v_{p+q})= \\
    &\sum_{\sigma \in S(p,q)}sign(\sigma)\omega_1(v_{\sigma(1)},\ldots,v_{\sigma(p)})(\omega_2(v_{\sigma(p+1)},\ldots,v_{\sigma(q)})
  \end{align*}
  where $S(p,q)$ is defined as a permutation of $\{1,\ldots,p+q\}$ such that $\sigma(1) < \ldots < \sigma(p)$ and $\sigma(p+1) < \ldots < \sigma(p+q)$.
\end{defn}
The alternating space $\text{Alt}^k(V)$ becomes an alternating $\mathbb{F}$-algebra when considering it with the associative exterior product. Refer to page 11 of \cite{calcohomo} for a more in-depth treatment.

\begin{defn}[Differential $p$-forms]\label{p-form}
  Let $U$ be an open set in $\mathbb{R}^2$. A differential $p$-form on $U$ is a smooth map $\omega : U \rightarrow \text{Alt}^p(\mathbb{R^2})$. The vector space of all such maps on $U$ is denoted by $\Omega^p(U)$.
\end{defn}
  The space of differential $0$-form is also sometimes written as $C^{\infty}(U)$
\begin{defn}[Exterior Derivative]\label{exteriorD}
  aaa
\end{defn}
\begin{defn}[Volume form]\label{volumeform}
  Sometimes known as an area form on surfaces.
\end{defn}

\begin{defn}[Integration]\label{Integration}
  Define integration 
\end{defn}


\section{de-Rham Cohomology}

\begin{defn}[de-Rham Cohomology]\label{deRham}
  aaa
\end{defn}
Examples and quite a few of them

\begin{defn}[Complex valued forms and de-Rham Complexes]\label{Cforms}
  aaa
\end{defn}

\section{Poincar\'{e}'s lemma}

\begin{defn}[Partitions Of Unity]
  
\end{defn}

\begin{defn}[Support of a function]
  
\end{defn}

\begin{defn}[de-Rham complexes and cohomology with compact support]
  
\end{defn}

\begin{thm}[Poincar\'{e}'s lemma]
  
\end{thm}

\section{Complex structures}

\begin{defn}[Complex structure]\label{Cstructure}
  
\end{defn}

\begin{lemma}[Structure splitting lemma]
  Any $\mathbb{R}$-linear map from $V$ to $\mathbb{C}$ can be written in a unique way as a sum of complex linear and antilinear maps.
\end{lemma}

\begin{lemma}[Can split $\Omega^1(X,\mathbb{C})$ into direct sum]
  
\end{lemma}

\begin{defn}[$L_2$ norm and inner-product on space of $(1,0)-$forms]\label{InnerProduct}
  
\end{defn}
\chapter{The Uniformisation Theorem}
\section{Preliminaries}
\subsection{The $\Delta$ operator and Harmonic functions}
We begin our treatment of the uniformisation theorem, with a discussion of a particular and extremely important differential operator on Riemann Surfaces; the Laplacian. The majority of definitions can be found in \cite{donaldson} unless explicitly stated.

\begin{defn}[The Laplacian]\label{LaplacianDef}
  Let $X$ be a Riemann Surface. We define a linear map $\Delta:\Omega^0(X)\rightarrow \Omega^2(X)$, where $\Delta=2i\overline{\partial}\wedge\partial$. We call this linear map the Laplacian. Often, we will suppress the $\wedge$ and simply write $\Delta=2i\overline{\partial}\partial$, with the $\wedge$ being implied, where no confusion may arise from such notation.
\end{defn}

Note that, though we have said that $\Delta$ is a map, it is a map from a space of functions to a space of functions, and hence is technically an operator; in particular, since it defines a differential equation in local coordinates, it is a differential operator and hence will be referred to as so.

In local co-ordinates, $\Delta$ simply recovers the expected Laplacian of a function since for a given function $f\in \Omega^0(X)$ we get that 
\begin{align*}
  \Delta f &= 2i\overline{\partial }\partial f = 2i(\frac{\partial}{\partial\overline{z}})(\frac{\partial}{\partial z})f(d\overline{z}\wedge dz) \\
  &=\frac{i}{2}(\frac{\partial}{\partial x} + i\frac{\partial}{\partial y})(\frac{\partial}{\partial x} - i\frac{\partial}{\partial y})f(2idx\wedge dy) \\
  &= -(\frac{\partial^2 f}{\partial x^2} + \frac{\partial^2 f}{\partial y^2})dx\wedge dy 
\end{align*}
which, when we replace $dx\wedge dy$ with the usual area form $dxdy$, becomes the usual Laplacian from vector calculus, up to a minus sign. We usually do this so we can identify functions with two-forms by choosing a volume form. For example, when given a function $f(x,y)$, it has an associated two-form $\rho = f(x,y) dx\wedge dy$, which can be integrated. Though we can define a notion of the Laplacian without the minus sign, as we will shortly see, that minus sign is important in our setup for the proof of the uniformisation theorem. 

A special class of function that is also important to our theory is that of the 
Harmonic functions. They play an important role in studying the Laplacian operator generally and have a number of very nice properties that we will rely on in the coming sections.
\begin{defn}[Harmonic function]\label{HarmonicDef}
  Let $X$ be a Riemann surface. A function $f\in \Omega^0(X)$ is said to be harmonic if $\Delta f = 0$. If however, $f$ is a smooth complex function, it is said to be harmonic if both its real and imaginary parts are real valued harmonic functions.
\end{defn}
We note that we can relax the differentiability condition considerably to make the functions only twice differentiable. Thus harmonic functions need to be at least twice differentiable. Furthermore, the domain of definition is not restricted to just Riemann surfaces but can also be any open subset of $\mathbb{R}^n$ or $\mathbb{C}^n$ for any $n\geq 1$. 

Let us give some examples of Harmonic functions.

\begin{example}[Examples of various Harmonic functions on different domains]\label{harmonicexamples}
  Over open sets in $\mathbb{C}$ we have:
  \begin{itemize} 
    \item Constant functions are trivially harmonic.
    \item Any holomorphic function is automatically a harmonic function since its real and imaginary parts satisfy the Cauchy-Riemann equations.
    \item $f(z)=e^z=e^{x+iy}=e^xsin(y)$
  \end{itemize}
  Taking our domain of definition to be open subsets of $\mathbb{R}^3$, we have:
  \begin{itemize} 
    \item $f(x,y,z)=\frac{1}{\sqrt{x^2+y^2+z^2}}$
  \end{itemize}
  Finally, on the punctured plane $\mathbb{C}\setminus \{0\}$, we also have a nice family of examples:
  \begin{itemize}
    \item $f(x+iy) = \log(x^2 + y^2)$ and hence $g(x+iy)=K \log(x^2+y^2)$ for all $K \in \mathbb{R}$.
  \end{itemize}
  All of the aforementioned functions satisfy the equation $\Delta f = 0$ on their domain of definition. This concludes the example.
\end{example}

It is enlightening to see the claim that any holomorphic function is harmonic using our language of differential forms.
\begin{lemma}\label{HolIsHarm}
  Holomorphic functions are harmonic functions, ie they have harmonic real and imaginary parts. 
\end{lemma} 
\begin{proof}
  Let $X$ be a Riemann surface and $f$ a holomorphic function defined on $X$.
  Let us consider the Laplacian applied to the sum $\frac{1}{2i}(f \pm \overline{f})$.
  \[\frac{1}{2i}\Delta(f \pm \overline{f}) = \overline{\partial}\partial(f \pm \overline{f})=-\partial(\overline{\partial}f) \pm \overline{\partial}\overline{(\overline{\partial}f)}=-\partial(0) \pm \overline{\partial}(0) = 0 \pm 0 = 0\]
  where both brackets $\overline{\partial} f = 0$ because $f$ is holomorphic.
\end{proof}

Note that whilst holomorphic implies harmonic, the converse isn't always true. However, we can at least say the following.

\begin{lemma}\label{HarmRealHol}
  Let $X$ be a Riemann Surface, $U$ and open set around a point $p \in X$ and let $\phi \in \Omega^0(U)$ be a harmonic function. Then there exists an open neighbourhood $V \subset U$ of the point $p$ and a holomorphic function $f \colon V \rightarrow \mathbb{C}$ with $\phi = Re(f)$.
\end{lemma}
Since this is a local result on $X$, then the proof follows exactly as in classical complex analysis. However, we wish to showcase how using calculus on Riemann surfaces can help solve problems efficiently and as such we provide a proof using the tools we have established.
\begin{proof}
  Let $A \in \Omega^1(U)$ be a real one-form such that: 
  \begin{align*}
    A &= -\frac{\partial \phi}{\partial y} dx + \frac{\partial \phi}{\partial x} dy \\ &= i\overline{\partial}\phi + \overline{(i\overline{\partial}\phi)}
  \end{align*}
  Since $\phi$ is harmonic, $\overline{\partial}\partial \phi = 0$ and $d = \partial + \overline{\partial}$, we get that $dA = 0$. Hence, if $V$ is an open, simply-connected set (such that $H^1(V)=0$), then we can find a function $\psi \in \Omega^0(V)$ such that $d\psi = A$. By equating coefficients we get that $\partial \psi = -i \partial \psi$ and $\overline{\partial} \psi = i \overline{\partial} \phi$. So if we construct a function $f \colon V \rightarrow \mathbb{C}$ where $f = \phi + i \psi$, we see that $\overline{\partial}f = \overline{\partial}(\phi + i\psi) = \overline{\partial}\phi +i(i\overline{\partial}\phi) = 0$ and hence, $f$ is a holomorphic function whose real part is $\phi$. 
\end{proof}

Harmonic functions are, clearly, quite well behaved and have nice properties that are not generally exhibited by other functions. These properties, such as the maximum principle, which we will define and prove below, play an important role in our treatment of the uniformisation theorem.

To prove the maximum principle however, we first need a version of the mean value theorem for harmonic functions.
\begin{lemma}[The Mean Value Theorem for Harmonic Functions]\label{MVT}
  Let $U$ be a domain in $\mathbb{C}$ around a point $a \in U$ and let $\phi \in \Omega^0(U)$ be harmonic on $U$. Let $\gamma$ be a closed circle of radius $R > 0$ contained within $U$ that encircles the point $a$. Then, the mean value of $\phi$ over the set bounded by $\gamma$ is equal to the value of $\phi$ at the point $a$. 
\end{lemma}
We note that while the statement of this lemma talks about domains of $\mathbb{C}$, we could equally talk about domains on any Riemann surface as by definition of a Riemann Surface, we have chart maps that map open sets of a Riemann Surface to domains in $\mathbb{C}$. 
\begin{proof}
  Since $\phi$ is a real-valued harmonic function, then by lemma \ref{HarmRealHol} then we have a holomorphic function $f$ such that the real part of $f$ is $\phi$. We now consider the statement of Cauchys Integral formula about $\gamma$.
  \begin{align*}
    f(a) = \frac{1}{2\pi i}\int_{\gamma}\frac{f(z)}{z}dz
  \end{align*}
  Parameterising by $\gamma$ gives us that $z= a + Re^{i\theta}$ for $\theta \in [0, 2\pi)$ on $\gamma$ and that $dz = iRe^{i\theta}$. Plugging these into the above equation gives us the relation that
  \begin{align*}
    f(a) &= \frac{1}{2\pi i}\int_{\gamma}\frac{f(z)}{z}dz = \frac{1}{2\pi i}\int_0^{2\pi}\frac{f(z)}{Re^{i\theta}}iRe^{i\theta}d\theta \\
    &= \frac{1}{2\pi}\int_0^{2\pi}f(z)d\theta
  \end{align*}
  Since $\phi$ is the real part of $f$, we can split $f$ into real and imaginary parts with the integrals splitting into real and imaginary parts. Hence, we get that 
  \begin{align*}
    \phi(a)=\frac{1}{2\pi}\int_0^{2\pi}\phi(z)d\theta
  \end{align*}
    which implies that the average value of $\phi(z)$ along $\gamma$ equals $\phi(a)$ giving us our mean value theorem.
\end{proof}

Now we are in a position to discuss the (strong) maximum principle for harmonic functions. Some authors may call following the strong maximum principle though since we dont need the "weak" maximum principle, we will simply refer to the following as the maximum principle.
The following can be found in \cite{arnold}
\begin{lemma}[The Maximum Principle]\label{MaximumPrincipleDef}
  Let $X$ be a Riemann Surface, let $U$ be a domain on $X$ and $\phi$ be a real-valued harmonic function on $U$. Then $\phi$ has no extrema in $U$. In other words, if for some $w \in U$ we have that $\phi(w) = max\{\phi(z) \colon \forall z \in U\}$ then $\phi$ must be a constant.
\end{lemma}
\begin{proof}
  Let $w_0$ be the point at which the maximum of $\phi$ occurs in $U$. Consider a small circular neighbourhood of $w_0$, with $w_0$ at the centre with the closed neighbourhood bounded by a circle called $\gamma$. We pick a point $w_1$ on the circle $\gamma$ and suppose that $\phi(w_0)>\phi(w_1)$. Let us connect $w_0$ and $w_1$ with a straight line $\delta$ so that $\delta:[0,1] \rightarrow U$ with $\delta(0)=w_0$ and $\delta(1)=w_1$.
  
  Since $\phi$ is continuous, that implies that for all $t > 0$, we must have that $\phi(\delta(0)) > \phi(\delta(t))$. But this violates lemma \ref{MVT}, the mean value theorem for harmonic functions, for if this is the case, then we would have 
  \[ \phi(w_0) > \frac{1}{2\pi}\int_0^{2\pi}\phi(\theta)d\theta \]
  This clearly cannot be so we must make it so that $\phi(w_0) = \phi(\delta(t))$ for all $t > 0$. So $\phi$ is constant in this small neighbourhood, which is closed. However, for any point in this neighbourhood, there is an open ball contained in $U$ surrounding it. Hence this neighbourhood is also open. This leads to the conclusion that the set of points maximising $\phi$ must either be empty, or must be the entire componant of the domain containing $w_0$. Hence, if $\phi$ has a maximum in $U$, $\phi$ must be constant across the whole of $U$.
\end{proof}

So we see that harmonic functions are indeed quite special. We now move on to our next topics of interest.

\subsection{Hilbert Spaces and the space $\mathcal{H}(X)$}

Hilbert spaces are in some ways, the most familiar class of space that one may study. They crop up everywhere in Mathematics and more and more in Theoretical Physics and indeed they play a major part in the proof of the Uniformisation theorem. Specifially, we will want to construct a particular Hilbert space that will contain a function that we will use to prove the uniformisation theorem. Let us provide some definitions first.

\begin{defn}[Inner Product Space]
  Let $V$ be a vector space over a field $\mathbb{F}$ and $\langle \cdot,\cdot \rangle \colon V \times V \rightarrow \mathbb{F}$ be a positive definite, sesquilinear form on $V$. Then the pair $\{V_{\mathbb{F}}, \langle \cdot,\cdot \rangle\}$ (frequently abbreviated as just $V_{\mathbb{F}}$ or $V$ if the underlying field is known) is called an Inner Product Space or Pre-Hilbert Space. If $\mathbb{F}=\mathbb{R}$, then the sesquilinearity of the inner product becomes a symmetry condition, ie that for some $x,y \in V$ we have that $\langle x, y \rangle = \langle y, x \rangle$. Such inner product spaces are called real inner product spaces.
  Finally, inner product spaces have a norm associated with them, defined as a function $\Vert \cdot\Vert  \colon V \rightarrow \mathbb{F}$ which can be computed by $\Vert v\Vert  = \sqrt{\langle v,v \rangle}$ for all $v \in V$. 
\end{defn}

We note that because inner product spaces have a norm associated with them, that they are, in fact, metric spaces. We note that we can define on any inner product space a metric as follows.
\begin{prop}
  Let $V$ be an inner product space. Then $V$ is a metric space.
\end{prop}
\begin{proof}
  Let $\{V_{\mathbb{F}},\langle \cdot,\cdot \rangle\}$ be an inner product space. Pick two elements of $a,v \in V$. We define the metric $d:V\times V \rightarrow \mathbb{F}$ with the formula $d(a,b)=\Vert a-b\Vert $. By the properties of the inner product, we see that $d(a,b)$ is symmetric, and positive definite. The triangle inequality follows by applying the Cauchy-Schwartz inequality. 
\end{proof}
We can now define a Hilbert space.
\begin{defn}[Hilbert Space]
  An inner product space $V$ is called a Hilbert Space if it is a complete metric space with a norm induced by the inner product.
\end{defn}
Hilbert spaces over the real numbers are called real Hilbert spaces and those over the complex numbers are called complex Hilbert spaces. We see that Hilbert spaces are very similar to just normal inner product spaces and as such a lot of our examples are familiar vector spaces. Examples include
\begin{example}
  \begin{itemize}
    \item $\mathbb{R}^n$ with the usual Euclidean inner product.
  \end{itemize}
\end{example}

%However, the fact Hilbert spaces are complete, allows us to find the limits of all Cauchy sequences of their elements. This completeness property allows us to generalise our notions of calculus as we have guaranteed that well defined limits exist within our space.

%A further important reason to want to use the theory of Hilbert Spaces is the Riesz Representation Theorem; a theorem that allows us, under sufficiently nice conditions, to represent a bounded linear functional acting on an element of a Hilbert space as an inner product of elements. Let us write this more precisely .
%\begin{thm}[The Riesz Representation Theorem]\label{RieszRepTheorem}
%  Let $H$ be a real Hilbert Space, and let $\sigma \colon H \rightarrow \mathbb{R}$ be a bounded linear functional, so there exists a constant $C$ such that $|\sigma(x)| \leq C\Vert x\Vert $ for all $x \in H$. Then there exists an element $z \in H$ such that \[ \sigma(x) = \langle z, x \rangle \]
%\end{thm}
%A proof of this theorem can be found in chapter ??? of ???

We are now in a position to introduce a space of critical importance to our study of the Laplacian on a Riemann surface. This space will be a Hilbert space, though we prove this in steps, first showing it is a vector space, then that given a particular norm, it is an inner product space, and then later on showing that it is complete. We call this inner product space $\mathcal{H}(X)$ and we shall explicitly construct it below, with its completion being constructed later and will be denoted as $\overline{\mathcal{H}}(X)$. 

Note that henceforth we consider only connected Riemann surfaces to alleviate  issues related to having multiple connected componants.

%\begin{defn}[The space $C^{\infty}(X)$]\label{infspace}
%  Let $X$ be a Riemann surface. We denote by 
%  $C^{\infty}(X)$ the space of all infinitely differentiable, real valued functions on $X$.
%\end{defn}

%This space is quite clearly a vector space of functions over the real numbers as with respect to the addition of functions, we have that the sum of any two smooth functions is again smooth. More precisely, we see that for all $f,g, h \in C^{\infty}(X)$ (ie, smooth functions) and $\lambda, \mu \in \mathbb{R}$, we have that $f + g = g + f$ and $f + (g + h) = (f +g) + h$, the zero function is smooth and of course, for each function $f$ there is a unique additive inverse, namely $-f$. Furthermore $\lambda(f + g) = \lambda f + \lambda g$ and $(\lambda + \mu)f = \lambda f + \mu f$.

%Whilst this space is quite clearly interesting in its own right, we wish to study a particular quotient of it. We do so by establishing the following equivalence relation.


\begin{lemma}
  Let $X$ be a connected Riemann Surface. We define a relation on $\Omega^0(X)$ (or on $\Omega^0_c(X)$ if $X$ is non-compact), whereby two functions are equivalent if they differ by a constant function. This relation, given the symbol $\sim$, defines an equivalence relation.
\end{lemma}

\begin{proof}
  We let $f,g,h \in \Omega^0(X)$ (or $\Omega^0_c(X)$ if $X$ is non-compact) be functions and let $a, b \in \mathbb{R}$ be arbitrary constants. 
  Since any function $f = f + 0$ then we have that $f \sim f$. 
  Further, we have that if $f \sim g$ impyling $f = g + a$, then we have that $g + (-a) = f$. Since $a \in \mathbb{R}$ then $-a \in \mathbb{R}$ and so we have $g \sim f$. 
  Finally, if $f \sim g$ and $g \sim h$ implying that $f = g + a$ and $g = h + b$, then we have that $f \sim h$ since $f = g + a = (h + b) + a = h + (b + a)$ and $b+a \in \mathbb{R}$. Hence, $\sim$ is indeed an equivalence relation on $\Omega^0(X)$. 
\end{proof}

With this equivalence relation we can now define the space of cosets of functions differing by a constant.

\begin{defn}[The space $\mathcal{H}(X)$]
  Let $X$ be a connected Riemann Surface. The space of cosets under the above defined equivalence relation $\sim$, is written as $\mathcal{H}(X) = \Omega^0(X)/\sim$. Each element of this space is a coset of smooth functions on the Riemann Surface $X$, whose elements all differ from the coset representative by a constant. 
  Similarly, if $X$ is non-compact, then we define $\mathcal{H}(X) = \Omega^0_c(X)/\sim$, whose elements all have compact support in $X$.
\end{defn}

Note that in \cite{donaldson}, this space is called both $H$ and $C^{\infty}(X)/\mathbb{R}$, with the former occuring when $X$ is non compact and the latter in the compact case. We will however, distinguish the two cases by identifying $X$ as being compact and non compact, if necessary. Furthermore, the notation $C^{\infty}(X)/\mathbb{R}$, comes from $C^{\infty}(X)$ meaning the space of smooth functions on $X$ in classical functional analysis. 

Perhaps unsurprisingly, $\mathcal{H}(X)$ is a vector space. This follows from the fact that $\Omega^0(X)$ is itself a vector space, and the explicit proof is nearly identical to showing that $\Omega^0(X)$ is a vector space. Though this new space is somehow more complicated than the space of smooth functions, there is an advantage to working with these cosets; namely that we can define a particular inner product and associated norm which will be of particular interest to our study of the Laplacian on Riemann surfaces. This inner product, called the Dirichlet Inner Product, is defined as follows.

\begin{defn}[The Dirichlet Inner Product]\label{dInnerProduct}
  Let $X$ be a connected Riemann Surface and let $f, g \in \mathcal{H}(X)$, where at least one of $f$ or $g$ have compact support within $X$. Then we define the Dirichlet Inner Product $\langle \cdot, \cdot \rangle_D : \mathcal{H}(X) \times \mathcal{H}(X) \rightarrow \mathbb{R}$ as 
  \[
    \langle f, g \rangle_D =\langle df, dg \rangle = 2i \int_X \partial f \wedge \overline{\partial} g 
  \]
  where the inner product on the right hand side is defined in the previous section as definition \ref{InnerProduct}. Notice that this is a real valued inner product.
  This inner product of course, also defines a norm in the usual way:
  \begin{align*}
    \Vert f\Vert _D^2 = \langle f, f \rangle_D
  \end{align*}
\end{defn}

If the functions were to not have compact support within this region then, the value of the inner product and norm might be $+\infty$, so by considering only functions with suitable compact support, we guarantee that we have an inner product and norm. 

So with all of this theoretical set up, it would be nice if we could somehow get back our claimed main object of study for this section; the Laplacian. Although hidden, the Dirichlet norm (and the associated inner product) already contain the Laplacian in their definition. The following proposition shows just how.

\begin{prop}\label{InnerLaplacian}
  Given a connected Riemann surface $X$, if at least one function $f,g$ is in $\mathcal{H}(X)$ (hence having compact support on $X$), then 
  \begin{align*}
    \langle f, g \rangle_D = \int_X f \Delta g = \int_X g \Delta f
  \end{align*}
\end{prop}
\begin{proof}
  We first consider the following two identities:
  \begin{itemize}
    \item $\overline{\partial}\wedge(f\overline{\partial}g) = f (\overline{\partial}\wedge\overline{\partial} g) + \overline{\partial}f \wedge \overline{\partial}g = 0 + 0 = 0$
    \item $\partial\wedge(f\overline{\partial}g)=\partial f \wedge \overline{\partial}g + f(\partial\wedge\overline{\partial}g)=\partial f \wedge \overline{\partial}g - \frac{1}{2i}f\Delta g$
  \end{itemize}
  Now let us expand the inner product. By the first identity above, we have that $f\overline{\partial}g$ is holomorphic, so that $d = \partial$. Hence, we can apply Stokes theorem. Note that since $X$ has no boundary, $\partial X = \emptyset$ and hence, that integral vanishes.
  \begin{align*}
    \langle f, g \rangle_D &= 2i\int_X \partial f \wedge \overline{\partial}g = 2i\int_X \partial(f\overline{\partial}g)+\int_X f\Delta g \\
    &= 2i\int_X \partial f \wedge \overline{\partial}g = 2i\int_{\partial X} f\overline{\partial}g + \int_X f\Delta g \\
    &= \int_X f\Delta g
  \end{align*}
  Since the Dirichlet inner product is a real inner product, it is symmetric and hence we have that $\langle f, g \rangle_D = \langle g, f \rangle_D $ and so $\int_X f\Delta g = \int_X g \Delta f$ as claimed.
\end{proof}

\section{The Dirichlet Energy Functional}

A point that we will not be proving explicitly is that all functionals we consider hereon are "continuous", in the sense that given a linear functional $F:\mathcal{H}(X) \rightarrow \mathbb{R}$, we have that for any $\epsilon > 0$ we have a $\delta > 0$ such that for all $x,y \in V$, $\Vert x - y \Vert < \delta \Rightarrow |F(x) - F(y)| < \epsilon$. This continuity allows us to search for limits, by looking at how the functional acts on Cauchy sequences, as we shall see.
We now begin by defining the $\hat{\rho}$ functional, a simple integral functional from $\mathcal{H}(X)$ to $\mathbb{R}$. 

\begin{defn}[The $\hat{\rho}$ functional]
  Let $X$ be a connected Riemann Surface and $\rho \in \Omega^2_c(X)$ be a $2$-form of compact support in $X$ such that $\int_X \rho = 0$. Then, we define the functional $\hat{\rho} \colon \mathcal{H}(X) \rightarrow \mathbb{R}$ to be defined as, for some $f$ of compact support in $\mathcal{H}(X)$,
  \begin{align*}
    \hat{\rho}(f) = \int_X f\rho
  \end{align*}
\end{defn}

Since $\int_X \rho = 0$ then we have that this functional is linear on functions of compact support in $\mathcal{H}(X)$. Note that, if $X$ is a compact Riemann surface, then all functions and $2$-forms on $X$ automatically have compact support in $X$, so the compact support conditions really only matter in the cases of non-compact Riemann surfaces.

We now prove the claim that $\hat{\rho}$ is a linear functional on $\mathcal{H}(X)$.
\begin{lemma}[$\hat{\rho}$ is a linear functional]\label{rhohatlinear}
  Let $X$ be a connected Riemann surface. Then $\hat{\rho}$ is a linear functional, ie $\hat{\rho}(\lambda f+ \mu g) = \lambda \hat{\rho}(f) + \mu \hat{\rho}(g)$
\end{lemma}
Recall that elements of $\mathcal{H}(X)$ are equivalence classes of smooth functions on $X$ (with compact support on $X$ if $X$ is non-compact), which differ by a constant.
\begin{proof}
  Let $\tilde{f},\tilde{g} \in \mathcal{H}(X)$ and $\lambda,\mu \in \mathbb{R}$. Then for some real constant, $a,b \in \mathbb{R}$ we have that $\tilde{f}(z) = f(z) + a$ and $\tilde{g}(z) = g(z) + b$. Recall also that $\int_X \rho = 0$.
  Hence, we get that:

  \begin{align*}
    \hat{\rho}(\lambda \tilde{f}+ \mu \tilde{g}) &= \hat{\rho}(\lambda(f + a) + \mu(g + b)) \\
    &=\int_X (\lambda(f + a) + \mu(g + b))\rho \\
    &=\lambda \int_X (f + a)\rho + \mu \int_X (g + b)\rho \\
    &=\lambda \int_x f\rho + \lambda a\int_X \rho + \mu \int_X g\rho + \mu b\int_X \rho \\
    &=\lambda \hat{\rho}(f) + \mu \hat{\rho}(g)\\
    &=\lambda \hat{\rho}(\tilde{f}) + \mu \hat{\rho}(\tilde{g})
  \end{align*}
  with the last line following since clearly $f \in \tilde{f}$ and $g \in \tilde{g}$.
\end{proof}

Let us use the fact that $\hat{\rho}$ is a linear functional to define a new functional that will play a central role in our setup for the proof of the uniformisation theorem.

\begin{defn}[Dirichlet Energy Functional]
  Let $X$ be a connected Riemann Surface, and suppose $f \in \mathcal{H}(X)$. Suppose further that we have $2$-form $\rho \in \Omega^2 (X)$ (or $\Omega^2_c (X)$ if $X$ is non-compact) such that $\int_X \rho = 0$. Then we can define a functional, $\mathcal{L}: 
  \mathcal{H}(X) \rightarrow \mathbb{R}$ such that $\mathcal{L}(f) = \Vert f\Vert ^2_D - 2\hat{\rho}(f)$. This $\mathcal{L}$ is called the Dirichlet energy functional, or sometimes Dirichlet energy for short.
\end{defn}

The case of $X$ being non-compact requires special treatment as we have seen in the preceeding definitions. In these cases, we consider functions $f$ that are compactly supported on $X$ as this guarantees that $f$ goes to zero as $\Vert f\Vert _D$ goes to $+\infty$. This condition, along with considering that a non-compact Riemann surface is simply connected, will be sufficient for our discussion of the Dirichlet functional on non-compact Riemann surfaces.

Returning to the Dirichlet functional, we see that we can consider this functional as an example of a Lagrangian; physically, a quantity that encapsulates the difference of the kinetic and potential energy of a dynamic system. With this in mind, let us try to set up a situation, which will give us the main tools needed to prove the Uniformisation theorem. In keeping with the theme of 19th century mathematics, we look to electrostatics for motivation. Let us for a moment, assume that our Riemann surface is in fact made from a conductive surface on which an electric field can permeate. We then pose the question, "How can we distribute charge across this surface in such a way that there is no overall gain or loss of electric charge on the surface?" 

There are two main points to be made here. First, the fact that there cannot be an overall gain or loss in electric charge on the system severely restricts the types of charge distribution that we can apply onto our Riemann surface. In fact, it implies that any charge distribution must be chosen so that its integral over the surface vanishes. If this were not the case, there would be a net source or sink of charge on our surface. We can see this by considering some charge distributions on, say a Riemann sphere. 

After choosing the area form $sin(\theta)d\theta d\phi$ with respect to the standard spherical coordinates, taking the Riemann sphere to have radius $1$, we can define our distribution to be $\rho = cos(\phi)sin(\theta)d\theta d\phi$. We then get that $\int_{\hat{\mathbb{C}}}\rho =\int_0^{2\pi}\int_0^{\pi}cos(\phi)sin(\theta)d\theta d\phi= 0$, which makes sense, since the distribution of charge along the angle $\phi$ is periodic across the sphere, so no single point on the sphere has more charge than any other, and is zero on the poles since $sin(\theta)=0$ at $\theta = 0,\pi$.

Another example distribution, say $\rho^{\prime}$ would be to take two small disjoint circular domains, of diameter $\epsilon > 0$ on our Riemann sphere such that our smooth charge distribution is compactly supported inside these two domains say $A$ and $B$. After fixing an area form on $\hat{\mathbb{C}}$, say $dz \wedge d\overline{z}$, let us say that the function representing the distribution on $A$ is defined as,
\[
\beta(z)=
\begin{cases}
  1, &|z| \leq \frac{1}{4}\epsilon \\
  0, &|z| > \frac{3}{4}\epsilon
\end{cases}
\]
with the function for $B$ being $-\beta$. We see clearly that the integral of our distribution splits into two. Hence, we get that since $\rho^{\prime}$ is zero outside $A \cup B$ that $\int_{\hat{\mathbb{C}}} \rho^{\prime} = \int_A \beta dz\wedge d\overline{z} + \int_B (-\beta) dz\wedge d\overline{z} = 0$, by the definition of $\beta$. So we can have charge distributions over the entire surface, or confine them to particular regions, so long as they satisfy certain conditions. The latter case is similar to defining point charges (and in fact, in the limiting case of $\epsilon \rightarrow 0$ does indeed define a point charge) on our Riemann surface, and so this is similar to defining a dipole on our sphere (a pair of opposing electric point charges). 

A pair of a bad examples would be that of any distribution of non-zero constant value on the Riemann sphere, or even one where it was defined as in the case of $\rho^{\prime}$ but only for a single set $A$. By the above example, the integral of such a distribution would obviously not be $0$. In the case of the constant distribution, we begin by denoting it by $\rho_{\text{bad}}$. Affixing the same spherical area form as in the first example, we see that the integral of $\int_{\hat{\mathbb{C}}} \rho_{\text{bad}} = \int_0^{2\pi}\int_0^{\pi}Dsin(\theta)d\theta d\phi = 4\pi D$ which is $0$ if and only if the constant $D$ is zero.

The second point to note is that Physicists generally agree that nature is lazy, and always acts to minimise the energy of a dynamical system. Therefore, we too should look for functions which minimise the Lagrangian of this problem, as these would be physically desirable. But what does the Lagrangian of this system look like? Since there is no motion (it is a static system), then the kinetic energy term is zero. So the Lagrangian of this system is a pure potential, made from two parts. The first part of the potential is a term related to the potential to the space itself to be charged, outside of any region of charge on the surface. Given a potential function $\phi$ (we will define what is meant by potential function in a moment), this term looks like $\Vert \phi \Vert_D^2$. The second part comes directly from the externally imposed charge density. This distributes positive and negative charges all over our surface according to $\hat{\rho}$. Its potential energy looks like $-2\hat{\rho}(\phi)$. This term describes the energy density inside the regions that contribute to the field on the surface. Therefore, the electrostatic potential energy on an arbitrary Riemann surface would be the sum of these two. This is discussed further in depth in \cite{electromagentismBook}, with a more mathematical treatment of similar problems in chapters $6$ and $12$ of \cite{arnold}. However since this is simply a motivation for the problem and we shall not make use of the electrostatics aspect of the problem any longer.

We therefore have that the Dirichlet energy functional, $\mathcal{L}$, is the Lagrangian of our system. Hence, we can turn this problem around to be the following:
\begin{problem}
  Given a charge distribution $\rho$, of integral zero over our Riemann surface, which functions exist that minimise the total energy of the system?
\end{problem}
This leads us to look for functions in $\mathcal{H}(X)$ that minimise $\mathcal{L}$. We call such functions potential functions. But how do we find such solutions? Do they even exist? To answer this, we appeal to the calculus of variations. We wish to apply the variational principle, a method for finding functions which minimise Lagrangians. To prove the existance of such a minimising function however, we need to show that $\mathcal{L}$ is a bounded functional, bounded from below. This will allow us to apply the variational principle to find which functions minimise $\mathcal{L}$. However, we must be careful as such minimising functions may live outside of $\mathcal{H}(X)$, in the Hilbert space $\overline{\mathcal{H}}(X)$, the completion of our inner product space. In our case, that will not be so, though showing that will be a little trickier.

Let us, for a moment, suppose such a potential function exists and that $\mathcal{L}$ is a bounded linear functional. How do we proceed finding such potential functions?
We begin by computing the first variation of $\mathcal{L}$. To do this we let $\epsilon > 0$ and let $g$ be function on $X$. The conditions on $g$ will be more precisely defined later, so for now, we assume such a $g$ exists. Let us now compute the first variation of $\mathcal{L}$:
\begin{gather*}
  \frac{d}{d\epsilon}\biggr\rvert_{\epsilon = 0}\mathcal{L}(f+\epsilon g) = 0 \\
  \frac{d}{d\epsilon}\biggr\rvert_{\epsilon = 0}(\Vert f+\epsilon g\Vert _D^2 -2\hat{\rho}(f+\epsilon g) +) = 0 \\
  \frac{d}{d\epsilon}\biggr\rvert_{\epsilon = 0} \langle f + \epsilon g, f + \epsilon g \rangle -2\frac{d}{d\epsilon}\biggr\rvert_{\epsilon = 0}(\int_X f\rho  + \epsilon \int_X g\rho) = 0 \\
  \frac{d}{d\epsilon}\biggr\rvert_{\epsilon = 0}(\langle f, f \rangle + 2\epsilon \langle f, g \rangle +\epsilon^2 \langle g, g \rangle) -2 \hat{\rho}(g) = 0 \\
  2\langle f, g \rangle -2\hat{\rho}(g) = 0 \\
  \int_X g\Delta f - \int_X g\rho = 0 \\
  \int_X g(\Delta f - \rho) = 0 
\end{gather*}
The appearence of the Laplacian comes from proposition \ref{InnerLaplacian}.
Since we have that at least one of $f,g$ and $\rho,g$ have compact support (and in this case both $f$ and $\rho$ are compactly supported on $X$), then for this integral to be zero, we need the following statement to be true:

There exists a unique solution $f$, up to addition by a constant, of the equation 
\[ \Delta f = \rho \Longleftrightarrow \int_X \rho = 0\]

The equation on the left, known as Poissons' equation, will be the main tool we will use in our treatment of the uniformisation theorem. As we shall shortly see, we only need to consider its solutions on a small number of Riemann surfaces to allow us to draw the conclusions we seek. We will split our study of the Poisson equation into two families:
\begin{itemize}
  \item The Poisson equation on connected, compact Riemann surfaces.
  \item The Poisson equation on connected, simply connected, non-compact Riemann surfaces.
\end{itemize}
We choose these two families of Riemann surface since in doing so, we can classify all connected, simply connected Riemann surfaces. This then allows us to classify all Riemann surfaces by their universal cover, since as we have seen, all connected, non-simply connected Riemann surfaces arise as a properly discontinuous group action on a connected, simply connected Riemann surface.

We begin by solving Poissons equation on connected, compact Riemann surfaces. A perhaps astounding corollary of our analysis of the Poisson equation on compact, connected Riemann surfaces is that up to conformal equivalence, there exists a single connected, simply connected, compact Riemann surface - the Riemann Sphere. 

\section{Solving the Poisson equation on compact Riemann surfaces}
We begin by trying to prove the following theorem, as the existance of solutions to the Poisson equation play a vital role in our classification of connected, compact Riemann surfaces.
\begin{thm}[Poissons's equation on compact Riemann surfaces]\label{compactPoisson}
  Let $X$ be a connected, compact Riemann surface. Let $\rho \in \Omega^2(X)$ be a $2$-form on $X$. Then, there exists a unique, smooth solution $f \in \Omega^0(X)$ to the equation $\Delta f = \rho$, up to the addition of a constant, if and only if $\int_X \rho = 0$. This equation is called the Poisson equation.
\end{thm}
We can immediately prove two things:
\begin{itemize}
  \item If $f$ is a smooth solution to $\Delta f = \rho$, then $\int_X \rho= 0$
  \item $f$ is unique up to a constant.
\end{itemize}
The remaining statement however, ie given an arbitrary $\rho \in \Omega^2(X)$ such that $\int_X \rho = 0$ then we can find a unique solution to Poisson's equation, up to the addition of a constant, is much tricker to prove and this section is dedicated to proving this point. So let us begin the proof of Theorem \ref{compactPoisson} by proving the two aforementioned bullet points.
\begin{proof}
  We wish to first prove that if $f$ is a smooth solution to $\Delta f = \rho$ then $\int_X \rho = 0$. To show this, we note that since $d = \partial + \overline{\partial}$ and $d(\partial f) = \partial^2 f + \overline{\partial}\partial f = \overline{\partial}\partial f$ since $\partial^2 = 0$. Recall that  $\Delta f = 2i \overline{\partial}\partial f$. Combining these two gives us that $\Delta f =2id(\partial f)$. So let us integrate $\rho$.
  \begin{align*}
    \int_X \rho = \int_X \Delta f = 2i \int_X \overline{\partial}\partial f = 2i \int_X d(\partial f) = 2i\int_{\partial X}\partial f = 0
  \end{align*}
  where the last equality follows from Stokes theorem and the fact that $X$ has no boundary. 

  Now to show that such a solution is unique up to a constant. We assume $f$ and $g$ are both solutions to the Poisson equation, ie $\Delta f = \rho$ and $\Delta g = \rho$. Then we have that $\Delta (f - g) = 0$, ie that $f-g$ is a harmonic function. Thinking of the Dirichlet norm for a moment, if we consider $\Vert f-g\Vert _D$ we get $\int_X (f-g)\Delta(f-g) = 0$. Hence, we have $\Vert f-g\Vert _D = 0$. Unfortunately, this is not a norm on $\Omega^0(X)$ but on $\mathcal{H}(X)$. But elements of $\Omega^0(X)$ can easily be mapped to elements of $\mathcal{H}(X)$. So let us consider $\Vert \tilde{f}-\tilde{g}\Vert _D = \Vert df - dg\Vert $. It follows that $\Vert \tilde{f}-\tilde{g}\Vert _D = 0$ since $\Delta(constant) = 0$ and so by the definition of $d$ and the Dirichlet norm, we get that $ df - dg = d(f-g) = 0$ implying that $f - g = c$ where $c$ is a constant. So any solution of the Poisson equation is unique up to the addition of a constant.
\end{proof} 

As stated, to complete our proof of Theorem \ref{compactPoisson}, now must now show that given a $2$-form $\rho \in \Omega^2(X)$ such that $\int_X \rho = 0$ we can find a smooth function $f$ such that $\Delta f = \rho$. Our strategy will be as follows: Since we are yet to formally motivate that we need to study the Poisson equation, we begin by doing that. To do so, we will first lay out some technical ground-work in the form of briefly studying convolutions of functions on $X$ in certain special cases. We then show that the Dirichlet energy is indeed a bounded functional, which attains a minimum when applied to minimising functions. After that, we show that such minimising functions are indeed obtained as solutions to Poissons equation, but may live "outside" our space of all possible smooth functions on $X$. We thence conclude by identifying these functions with smooth functions on $X$, and so, combined with our proof above, we will have proven the statement of Theorem \ref{compactPoisson}. 

\subsection{Convolutions and related technical lemmas}
As mentioned, we will now delve into a short discussion on convolutions. We begin by defining a convolution of two functions. 
\begin{defn}[Convolution of functions]\label{ConvolutionDefn}
  Let $f,g$ be smooth, complex functions defined on $\mathbb{R}^n$. Then their convolution is defined as $(f * g)(x) = \int_{\mathbb{R}^n}f(y)g(x-y)dy$ where $dy$ is volume form on $\mathbb{R}^n$, provided the integral exists for all $x \in \mathbb{R}^n$ .
\end{defn}
Note that this definition implies we have the identity $\langle f, g*h \rangle = \langle g*f, h \rangle$, and we will refer to this as the triple product identity for convolutions. Also note that, one can relax the requirement for the integral to exist on all $x \in \mathbb{R}^n$ to requiring $x$ to be defined on $\mathbb{R}^n\setminus{U}$, where $U \subset \mathbb{R}^n$ is a set of measure zero. The definition of a set of measure zero can be found on page 50 of \cite{spivak}.
For a detailed treatment of convolutions of functions and the theory of distributions and a more in depth discussion of the theory of Partial Differential Equations, one can refer to Chapter 6 of \cite{rudin}. 
Let us prove some useful lemmas involving convolutions with the Laplacian that we will be using later. 

We also will want to use a theorem from classical vector calculus, known as Green's identity, though we will modify the statement of this theorem for the case of the plane. This is not to be confused with Green's theorem for line integrals.
\begin{thm}[Green's identity]
  Let $u,v \in \mathbb{R}^2$ be smooth functions on a path connected, bounded set $\Omega \subset \mathbb{R}^2$. Then we have that  
  \[\iint_{\Omega} (u \nabla^2 v - v \nabla^2 u) dS= \int_{\partial \Omega} \left(u \frac{\partial v}{\partial n} - v \frac{\partial u}{\partial n}\right) dl, \] 
  where $n$ is the oriented normal of the boundary curve, $dS$ is the standard surface area element and $dl$ is the standard arc length element of the boundary.
\end{thm}
Recall that in local coordinates, the definition of the Laplacian gave us that \[-\Delta f = (\frac{\partial^2 f}{\partial x^2} + \frac{\partial^2 f}{\partial y^2}) dx\wedge dy = (\nabla^2f) dx\wedge dy\]
where $\nabla^2f$ is the notation we shall use for the functional portion of our Laplacian (ie given a volume form $d\mu$, then $\Delta f = (\nabla^2 f) d\mu$).

In the plane, $\mathbb{R}^2$, with regular cartesian coordinates, we have that $dS = dx \wedge dy$ or in more usual notation $dS = dxdy$. $dS$ and $dl$ are both examples of measures which, loosely speaking, "measure" how much one unit of whatever we are integrating over is.
By changing coordinates we can perhaps simplify our problems, but $dS$ still measures the same surface area element. We see that the limits of integration must hence change to account for the potential introduction of new factors in the integral, brought about by this change of coordinates. An example of this is if we change from our standard coordinates to polar coordinates $(r,\theta) \mapsto (rCos(\theta),rSin(\theta)) = (x,y)$. In this case, $dS=dxdy=rdrd\theta$ where the factor of $r$ is recovered by computing the determinant of the jacobian of the change of coordinates, as usual. In short, the important thing to note with Green's identity is that it is independent of choice of coordinates used, and our notation when using it will take advantage of this fact.

Returning to convolutions, we have the following lemma
\begin{lemma}\label{ConvProperties}
  Consider the function $V(z) = \frac{1}{2\pi}\log|z|$. We call this function the Newtonian potential and its significance will be discussed later. $V$ is well defined on all of $\mathbb{C}\setminus\{0\}$. Let us further consider two smooth functions $\sigma, \tau$ of compact support in $\mathbb{C}$. Then we have the following two points:
  \begin{itemize}
    \item $(V * \Delta \sigma)(z)=\sigma(z)$
    \item $\Delta(V * f)(z) = f(z)$
  \end{itemize}
\end{lemma}

\begin{proof}
  Since we can change coordinates linearly, we can set $z=0$ and evaluate the convolution of $(V * \Delta \sigma)(0)$. We also note that since $\log(z)$ is harmonic on $\mathbb{C}\setminus \{0\}$, as we saw in example \ref{harmonicexamples}, then $\Delta \log(|z|) = 0 $. By setting $dS$ to be our volume form, we get:
  \begin{align*}
    (V * \Delta \sigma)(0) &= \int_{\mathbb{C}}\frac{1}{2\pi}\log(|w|)\Delta \sigma dS
  \end{align*}
  Let $\text{supp}(\sigma) = U \subset \mathbb{C}$. By the Heine-Borel theorem, since this is compact subset of the plane, it is also closed and bounded and hence, $U$ has a boundary $\partial U$. Thus we can replace our domain of integration from $\mathbb{C}$ to $U$ since $\sigma$ is zero outside of $U$. Let us also denote, for some $\delta > 0$, a small closed ball $B_{\delta} \subset U$ around $0$. Since $\sigma(w) = 0$ for all $w \in \mathbb{C}\setminus U$ then the integral needs only to be evaluated on $U$.
  We will want however to be careful with how we treat the singularity at the origin. We begin by cutting out from $U$ a small open ball of radius $\delta >0$ for some small $\delta$, such that the ball $B_{\delta} \subset U$. Then we can split $U$ into a union of two sets $U=B_{\delta} \cup U_{\delta}$, where $U_{\delta} = U\setminus B_{\delta}$. In the limit of $\delta \rightarrow 0$ we get that $U=U_{\delta}$. So let us evaluate the above integral in this limit.
  We begin by applying Green's identity, over the set $U_{\delta}$, since by letting $u = \frac{1}{2\pi}\log(|w|)$ and $v = \sigma$ we get the statement of the left hand side of Green's identity since $\frac{1}{2\pi}\log(|w|)$ is harmonic on $U_{\delta}$. So we proceed as follows.
  \begin{align*}
    (V * \Delta \sigma)(0) &= \lim_{\delta \rightarrow 0} \iint_{U_{\delta}} \frac{1}{2\pi}\log(|w|)\Delta\sigma(w) dS \\
    &= \lim_{\delta \rightarrow 0} \int_{\partial U_{\delta}}\left(\frac{1}{2\pi}\log(|w|)\frac{\partial \sigma(w)}{\partial n} - \sigma(w)\frac{\partial}{\partial n}\left(\frac{1}{2\pi}\log(|w|)\right)\right)dl
  \end{align*}
  Careful thought gives us that $\partial U_{\delta}$ has two componants, one on the outer edge of $U_{\delta}$, the boundary of the support of $\sigma$, called the outer boundary, and one on the circle around the origin, of radius $\delta$, the inner boundary. Since $\sigma$ is continuous, then on the outer boundary, the value of $\sigma$ must be zero and so the integral there is zero. Hence, we only need to consider what happens on the interior boundary. So to compute the integral, we must consider the normal outwards vector of this circle. However, the outwards vector of this circle points inwards towards the origin. So we define the normal vector $\hat{\underline{n}}= -\frac{1}{r}\hat{\underline{r}}$ and that the normal derivative is $\frac{\partial}{\partial n} = -\frac{\partial}{\partial r}$.
  Finally, since we are taking a radial limit it makes sense to parameterise $w$ using polar coordinates, so $w=re^{i\theta}$ with $r\in [0, \infty)$ and $\theta \in [0, 2\pi)$. Our integral now becomes
  \begin{align*}
    (V * \Delta \sigma)(0) &= \lim_{\delta \rightarrow 0} \int_{\partial U_{\delta}}\left(\frac{1}{2\pi}\log(|w|)\frac{\partial \sigma(w)}{\partial n} - \sigma(w)\frac{\partial}{\partial n}\left(\frac{1}{2\pi}\log(|w|)\right)\right)dl \\
    &= \lim_{\delta \rightarrow 0} \int_{\partial U_{\delta}}\left( -\frac{1}{2\pi}\log(r)\frac{\partial \sigma(re^{i\theta})}{\partial n} + \sigma(re^{i\theta})\frac{\partial}{\partial r}\left(\frac{1}{2\pi}\log(r)\right)\right)dl \\
    &= \lim_{\delta \rightarrow 0} \int_{\partial U_{\delta}}\left( -\frac{1}{2\pi}\log(r)\frac{\partial \sigma(re^{i\theta})}{\partial n} + \frac{1}{2\pi r}\sigma(re^{i\theta})\right)dl \\
  \end{align*}
  Since we are evaluating this integral along the edge of the circle of radius $\delta$, we can set $r=\delta$ and note that $dl$ hence becomes $d\theta$ since parameterise our position on the circle by the angle $\theta$. Hence our integral becomes
  \begin{align*}
    (V * \Delta \sigma)(0) &= \lim_{\delta \rightarrow 0} \int_{\partial U_{\delta}}\left( -\frac{1}{2\pi}\log(r)\frac{\partial \sigma(re^{i\theta})}{\partial n} + \frac{1}{2\pi r}\sigma(re^{i\theta})\right)dl \\
    &= \lim_{\delta \rightarrow 0} \int_{\partial U_{\delta}}\left( -\frac{1}{2\pi}\log(\delta)\frac{\partial \sigma(\delta e^{i\theta})}{\partial n} + \frac{1}{2\pi \delta}\sigma(\delta e^{i\theta})\right)dl \\
    &= \lim_{\delta \rightarrow 0} \int_{0}^{2\pi}\left( -\frac{1}{2\pi}\log(\delta)\frac{\partial \sigma(\delta e^{i\theta})}{\partial n} + \frac{1}{2\pi \delta}\sigma(\delta e^{i\theta})\right)d\theta \\
    &= \lim_{\delta \rightarrow 0} \left( -\frac{1}{2\pi}\log(\delta) \int_{0}^{2\pi} \frac{\partial \sigma(\delta e^{i\theta})}{\partial n} d\theta + \frac{1}{2\pi \delta}\int_{0}^{2\pi}\sigma(\delta e^{i\theta})d\theta\right) \\
  \end{align*}
  If we now consider the averages of these integrals, we can attempt to directly compute the limit. We note that the average value of each integral will be the total arc length of the circle, $2\pi \delta$, multiplied by some value, as we are intgerating over the whole circle. We let the symbol $\mu_{\delta}$ indicate the average value of the first integral and $\overline{\sigma_{\delta}}$ to mean the average value of the second integral, at radius $\delta$. Substituting these into the integral therefore gives us
  \begin{align*}
    (V * \Delta \sigma)(0) &=\lim_{\delta \rightarrow 0} \left( -\frac{1}{2\pi}\log(\delta) \int_{0}^{2\pi} \frac{\partial \sigma(\delta e^{i\theta})}{\partial n} d\theta + \frac{1}{2\pi \delta}\int_{0}^{2\pi}\sigma(\delta e^{i\theta})d\theta\right) \\
    &= \lim_{\delta \rightarrow 0} \left( -\frac{1}{2\pi}\log(\delta)2\pi \delta \mu_{\delta}  +  \frac{1}{2\pi \delta} 2\pi \delta \overline{\sigma_{\delta}}\right) \\
    &= \lim_{\delta \rightarrow 0} \left( -\delta \log(\delta)  \mu_{\delta}  + \overline{\sigma_{\delta}}\right) \\
    &= \sigma(0)
  \end{align*}
  where the last equality holds since $\lim_{\delta \rightarrow 0}\delta \log(\delta) = 0$ and $\overline{\sigma_{\delta}} \rightarrow \sigma(0)$ as $\delta \rightarrow 0$ since we have that the average value of the integral of $\sigma$ on smaller and smaller circles around the orgin tends to the value of $\sigma$ at the origin.
  Hence we have shown that $(V * \Delta \sigma)(0) = \sigma(0)$.

  To show now that $\Delta(V*f)(z) = f(z)$, is simpler. We begin by bringing in the $\Delta$ into the integral, which we can do since the integral is with respect to the dummy variable $w$ but $\Delta$ is the Laplacian on the coordinate $z$. We denote this volume form as $dS_w$. Hence we have,
  \begin{align*}
    \Delta(V*f)(z) &= \Delta \int_\mathbb{C} V(w)f(z-w)dS_w \\
    &= \int_\mathbb{C} \Delta \left(V(w)f(z-w)\right)dS_w \\
    &= \int_\mathbb{C} V(w)\Delta f(z-w)dS_w
  \end{align*}
  with the last equality following since $V$ is a function of the coordinate $w$ but $\Delta$ is an operator on the coordinate $z$. Hence, it passes through to $f$ which is a smooth function of compact support. Hence, we have shown that $\Delta(V*f)(z) = V*(\Delta f)(z)$ which we have just shown is $f(z)$. Therefore we have that $\Delta(V*f)(z) = f(z)$ as claimed.
\end{proof} 

\subsection{Showing the Dirichlet energy $\mathcal{L}$ is a bounded functional}

Returning to our strategy for showing that given a $2$-form $\rho$ on a connected, compact Riemann surface $X$ such that $\int_X \rho = 0$, then we have a smooth function $f$ on $X$ such that $\Delta f = \rho$, we want to now prove that such an $f$ can exist. To do this, recall that we need to show that the Dirichlet energy, $\mathcal{L}(f) = -2\hat{\rho}(f) + \Vert f \Vert^2_D$ is a bounded functional, bounded from below. Since for any smooth function $f$ on $X$ we have a finite Dirichlet norm (ie $\Vert f \Vert_D < +\infty $ for all $f \in \mathcal{H}(X))$, %since $f$ is continuous and must therefore attain a maximum value at some point on $X$,
we are left with needing to show that the $\hat{\rho}$ functional is also bounded.

To prove this, we need the following theorem and its subsequent corollary. We proceed by working in a bounded, convex, open set in the complex plane. We can do so as we can map any bounded, convex, open set from our Riemann surfaces confomally to the such sets in the complex plane using chart maps. Similarly, due to conformal equivalence, we can assume such sets are circular discs in the complex plane.

We omit the proofs of the following theorem and its corollary due to them being long arguments that are well described in the quoted references.
\begin{thm}[p122. Theorem 11 \cite{donaldson}]\label{quotedTheorem11}
  Let $\Omega$ be a bounded convex open set in $\mathbb{R}^2$ and $\psi$ be a smooth function on an open set containing the closure $\overline{\Omega}$ with $\overline{\psi}$ denoting the average 
  \[\overline{\psi} = \int_{\Omega}\psi dS\] 
  where $A$ is the area of $\Omega$. Then, for $x \in \Omega$, we have 
  \[|\psi(x) - \overline{\psi}| \leq \frac{d^2}{2A}\int_\Omega \frac{1}{|x - y|}|\nabla\psi(y)|dS_y\] where $dS_y$ indicates that $y$ is the variable of integration.
\end{thm}
Note that in \cite{donaldson}, the author uses the notation $d\mu$ instead of $dS$, calling $d\mu$ the Lebesgue measure on $\mathbb{R}^2$. As mentioned before briefly, $dS$ is an example of a measure. However to properly define a measure would be too large of a detour for such a minor point, and an in depth expos\'{e} can be avoided to some degree. Measure theory as a field of study however is a modern and analytical approach to integration amongst other things and is indeed extremely interesting. The reader is invited to read chapter 11 of \cite{babyRudin} for a more depth introduction into the subject.

Returning to the theorem, an important corollary is the following.
\begin{cor}[p123. Corollary 6 \cite{donaldson}]\label{quotedCorollary6}
  Under the hypothesis from Theorem \ref{quotedTheorem11}, we have that 
  \[\int_\Omega |\psi(x) - \overline{\psi}|^2 dS_x\leq \left(\frac{d^3\pi}{A}\right)^2\int_\Omega |\nabla\psi|^2 dS\]
\end{cor}
Using these two estimates we can prove the following theorem.
\begin{prop}\label{PartitionOfUnity}
  Let $X$ be a compact, connected Riemann surface. Then the functional $\hat{\rho}:\mathcal{H}(X) \rightarrow \mathbb{R}$ is bounded, ie there exists a constant $C$ such that $|\hat{\rho}(\tilde{f})| \leq C \Vert \tilde{f} \Vert_D$, for all $\tilde{f} \in \mathcal{H}(X)$.
\end{prop}
The following proof follows the proof found on page 125 of \cite{donaldson}, though we have modified our notation to make certain aspects of the proof clearer and easier to follow.
\begin{proof}
  We split this proof into two parts. First we show that $\hat{\rho}$ is bounded in the case when $\text{supp}(\rho)$ is contained within a single coordinate chart, and then we show that we can construct a bound for $\hat{\rho}$ over the whole of our compact, connected Riemann surface $X$.
  So let us consider a bounded, convex set $U$ in $\mathbb{C}$. We also state once again that, because chart maps between coordinate patches on $\mathbb{C}$ and open sets in $X$ are conformal equivalences, we can work in these coordinate patches directly, passing functions from $X$ to sets in $\mathbb{C}$ using precomposition with charts and transition maps if necessary (recall the the definition of an atlas).
  So let $\tilde{f} \in \mathcal{H}(X)$. We write 
  \[\hat{\rho}(\tilde{f}) = \int_U (f + a) \rho\] 
  for some $a \in \mathbb{R}$. We set the constant $a$ be equal to $-\overline{f}$, the average value of $f$ on $U$, hence writing 
  \[\hat{\rho}(\tilde{f}) = \int_U (f - \overline{f}) \rho\]
  By fixing an area form in this coordinate chart, say $dS$, we can write $\rho = g dS$, where $g \in \Omega^0(X)$. This hence gives us that 
  \[\hat{\rho}(\tilde{f}) = \int_U (f - \overline{f}) g dS\] which by the Cauchy-Schwarz inequality gives
  \[ |\hat{\rho}(\tilde{f})| = \left| \int_U (f - \overline{f}) g dS \right| \leq \Vert g \Vert \Vert f-\tilde{f} \Vert  \]
  Hence, by corollary \ref{quotedCorollary6}, we get that 
  \[ |\hat{\rho}(\tilde{f})| \leq C \Vert \nabla f \Vert = C \Vert df \Vert\] where $C = d^3\pi A \Vert g \Vert$, and we have used the fact that the norm of the gradient and exterior derivative of a function are equal.
  Finally, if we now compose with a chart map to map back to $X$ from $U$, and letting $ \Vert \cdot \Vert_U$ indicate the usual norm on $U$, we have that \[ \Vert df \Vert_U \leq \Vert df \Vert_X = \Vert \tilde{f} \Vert_{D,X}\]
  Hence we have that if $\hat{\rho}$ is supported in a single coordinate chart on $X$, then $\hat{\rho}$ is a bounded operator.

  So to extend this over multiple coordinate charts on $X$ we first fix a finite cover of $X$ by coordinate charts of the type considered in the previous case. This is possible since $X$ is compact. We call this family of coordinate charts $U_{\alpha}$. Since we showed that for each coordinate chart, a $2$-form of compact support in that chart of integral zero is bounded, we wish to somehow stitch together these forms to form a $2$-form of integral zero over the whole of $X$. To do this, we introduce a partition of unity $\chi_{\alpha}$, subordinate to our choice of finite cover $U_{\alpha}$
  
  Since integration of two forms defines an isomorphism between $H^2(X)$ and $\mathbb{R}$, we can write the $2$-form from our functional $\hat{\rho}$ as $\rho = d\theta$ for some $\theta \in \Omega^1(X)$. Now on each $U_{\alpha}$ we derive a $2$-form of compact support from $\theta$ using our partition of unity by setting $\rho_{\alpha} = d(\chi_{\alpha}\theta)$. Each $2$-form $\rho_{\alpha}$ is of compact support in $U_{\alpha}$ which it gets from $\chi_{\alpha}$. Furthermore, we have that,
  \[ \int_X\rho_{\alpha} = \int_X d(\chi_{\alpha}\theta) = \int_{\partial X} \chi_{\alpha}\theta = 0 \] where we get the second to last equals sign by Stokes theorem, and the equality to zero since $X$ has no boundary.
  So $\int_X \rho_{\alpha} = 0$. Finally, since $\sum_{\alpha} \chi_{\alpha} = 1$, we have that 
  \[ \rho = d\theta = d\left(\sum_{\alpha}\chi_{\alpha}\theta\right) = \sum_{\alpha}d(\chi_{\alpha}\theta) = \sum_{\alpha}\rho_{\alpha}\]
  As such, we have constructed a $2$-form $\rho$ such that $\int_X \rho = 0$. Since we showed in the first part that each $\hat{\rho}_{\alpha}$ is bounded in its coordinate chart, and since $\rho$ is equal to a finite sum of $\rho_{\alpha}$ we conclude that the functional $\hat{\rho} = \sum_{\alpha}\hat{\rho}_{\alpha}$ is also bounded since it is a finite sum of bounded linear maps.
\end{proof}

So we have that the functional $\hat{\rho}$ is bounded. Let us directly show that $\mathcal{L}$ is hence bounded from below.
We have that $\mathcal{L}(f) = -2\hat{\rho}(f) + \Vert f \Vert^2_D$ for some function $f \in \mathcal{H}(X)$. We showed above that $|\hat{\rho}(f)| \leq C \Vert f \Vert_D$ for the constant $C$ defined above. Since $\hat{\rho}(f) \leq |\hat{\rho}(f)|$ we have that $ \mathcal{L}(f) \geq -2C \Vert f \Vert_D + \Vert f \Vert^2_D = (\Vert f \Vert_D - C)^2 - C^2 \geq - C^2$. Hence, we have that the Dirichlet energy is indeed bounded from below. We now proceed to seek this minimising function. Recall, that a function which minimises $\mathcal{L}$ is a solution to the Poisson equation for a given $\rho$.

\subsection{The completion of $\mathcal{H}(X)$ and Weyl's lemma}
Let us start this section by discussing the completion of the space $\mathcal{H}(X)$. What does this space look like? We can construct this space abstractly, by defining it as follows.
\begin{defn}[The space $\overline{\mathcal{H}}(X)$]\label{completeH}
  Let $X$ be a connected Riemann surface. The space $\overline{\mathcal{H}}(X)$, the abstract completion of $\mathcal{H}(X)$, can be constructed from the inner product space $\mathcal{H}(X)$ with the Dirichlet inner product, as the space of equivalence classes of Cauchy sequences of elements in $\mathcal{H}(X)$, under the equivalence relation $\sim$ where two Cauchy sequences $\{\psi_i\}$ and $\{\phi_i\}$ are equivalent if $\lim_{i \rightarrow \infty}\Vert \psi_i - \phi_i\Vert_D \rightarrow 0 $.
  Therefore, a point in $\overline{\mathcal{H}}(X)$ is an equivalence class of Cauchy sequences of functions that differ by a constant, that have the same limit.
\end{defn}
This space is clearly very complicated, but its usefulness to us comes from the fact that we can guarantee that Cauchy sequences of functions will attain their limit in $\overline{\mathcal{H}}(X)$. This space is indeed a vector space because we can add elements of this vector space term by term. The space $\overline{\mathcal{H}}(X)$ also has an inner product, which we define as, for any two points $\psi, \phi \in \overline{\mathcal{H}}(X)$, where $\psi = \{\psi_i\}$ and $\phi = \{\phi_i\}$, the product $( \psi, \phi )_D = \lim_{i \rightarrow \infty} \langle \psi_i, \phi_i \rangle_D$. Note, that we use the notation $( \cdot, \cdot)_D$ for the inner product in $\overline{\mathcal{H}}(X)$ and $\langle \cdot, \cdot \rangle_D$ for the inner product in $\mathcal{H}(X)$.

An interesting point to note is that each inner product in the limit, is just a sequence that is Cauchy in $\mathbb{R}$. Since we have a complete inner product space, with a norm that we can derive from our new inner product, $( \cdot, \cdot )_D$, then $\overline{\mathcal{H}}(X)$ is in fact, a Hilbert space. 
% The benefit of this is that we can now apply the Riesz Representation Theorem, which will allow us to find a representative in $\overline{\mathcal{H}}(X)$ for our functionals. In the case of $\hat{\rho}$ this representative will turn out to be a "weak solution" to the Poisson equation (weak in the sense that it might not be in $\mathcal{H}(X)$). However, as we will show, Weyl's lemma guarantees that in fact, all weak solutions to the Poisson equation, do in fact live in $\mathcal{H}(X)$, and are hence smooth.

To do this, we begin by extending our functionals to $\overline{\mathcal{H}}(X)$; a process of redefining their domains from $\mathcal{H}(X)$ to $\overline{\mathcal{H}}(X)$. We identify these extended functionals with an underline, except for the extended Dirichlet inner product. This is necessary, as elements of this new Hilbert space look vastly different to that of $\mathcal{H}(X)$. We have already extended the definition of the norm, since we have that a norm on $\overline{\mathcal{H}}(X)$ is defined as, for some $\psi \in \overline{\mathcal{H}}(X)$ that $\underline{\Vert\psi\Vert}_D^2 = (\psi, \psi)_D$.
So we now need to extend $\hat{\rho}$. As stated earlier, $\hat{\rho}$ is a bounded functional on $\mathcal{H}(X)$. Its extension to $\overline{\mathcal{H}}(X)$ can be accomplished as follows; note first that for some $\psi = \{\psi_i\} \in \overline{\mathcal{H}}(X)$, we can define the functional 
\[\underline{\hat{\rho}}(\psi) = \lim_{i \rightarrow \infty} \hat{\rho}(\psi_i) = \lim_{i \rightarrow \infty} \int_X \psi_i\rho\]
in similar vein to the inner product. We note that this functional is Cauchy in $\mathbb{R}$ since if we take a Cauchy sequence from $\mathcal{H}(X)$, say $\{\phi_i\}$ then the sequence $\hat{\rho}(\phi_i)$ is Cauchy in $\mathbb{R}$.
Furthermore, $\underline{\hat{\rho}}$ is bounded since we have that 
\[ |\underline{\hat{\rho}}(\psi)| = \lim_{i \rightarrow \infty}|\hat{\rho}(\psi_i)| \leq \lim_{i \rightarrow \infty}C\Vert \psi_i \Vert = C\underline{\Vert \psi\Vert} \] for some $\psi \in \overline{\mathcal{H}}(X)$.
As such, we can define our extended version of the Dirichlet energy as being
$\underline{\mathcal{L}}(f) =  -2\underline{\hat{\rho}}(f) + \underline{\Vert f \Vert}_D^2$, for any $f \in \overline{\mathcal{H}}(X)$. It follows that $\underline{\mathcal{L}}$ is also a functional that is bounded from below. The proof is identical to that of $\mathcal{L}$. However, now with our extended functional $\underline{\mathcal{L}}$, we can guarantee that if there is an abstract element $\Phi$ for which $\underline{\mathcal{L}}(\Phi) = -C^2$ then $\Phi \in \overline{\mathcal{H}}(X)$. To obtain such an element, we need to construct a Cauchy sequence of functions in $\mathcal{H}(X)$ that converges in $\overline{\mathcal{H}}(X)$, for which $\mathcal{L}$ attains this minimum, thereby guranteeing the existance of an element of $\overline{\mathcal{H}}(X)$ for which $\underline{\mathcal{L}}$ attains its minimum. We proceed with the following lemma.
\begin{lemma}
  Let $X$ be a compact, connected Riemann surface. We let $\mathcal{L}$ be the Dirichlet energy functional, and let it be bounded from below by a number $M \in \mathbb{R}$, ie $\mathcal{L} \geq M$ (which holds true by the previous section with $M=-C^2$).
  Let $\{f_i\}$ be a sequence of functions such that $\lim_{i \rightarrow \infty}\mathcal{L}(f_i) = M$. Then $\{f_i\}$ is a Cauchy sequence in $\mathcal{H}(X)$ which defines our abstract element $\Phi \in \overline{\mathcal{H}}(X)$, such that $\underline{\mathcal{L}}(\Phi) = M$.
\end{lemma}
Note that the work in this lemma and proof follow the proposition on pages 30 and 31 of \cite{notes}.
\begin{proof}
  Let $\epsilon > 0$. Then there exists some $N \in \mathbb{N}$ such that for $i,j \geq N$ we have that $\mathcal{L}(f_i) \leq M + \frac{1}{4}\epsilon$ and $\mathcal{L}(f_j) \leq M + \frac{1}{4}\epsilon$. We define a function $I:[0,1] \rightarrow \mathbb{R}$, as a function of t, with $I(0) = \mathcal{L}(f_j)$ and $I(1) = \mathcal{L}(f_i)$, which interpolates between the two giving
  \begin{align*}
    I(t) &= \mathcal{L}(tf_i + (1-t)f_j) \\
         &= \Vert tf_i + (1-t)f_j \Vert^2_D - 2\hat{\rho}(tf_i + (1-t)f_j) \\
         &= \Vert f_j + (f_i - f_j)t \Vert^2_D - 2\hat{\rho}(f_j + (f_i - f_j)t) \\
         &= \Vert f_i - f_j \Vert^2_Dt^2 + 2\langle f_i, f_j\rangle_Dt + \Vert f_j \Vert^2_D - 2\hat{\rho}(f_j) - 2\hat{\rho}(f_i - f_j)t \\
         &= \Vert f_i - f_j \Vert^2_Dt^2 + 2(\langle f_i, f_j\rangle_D - \hat{\rho}(f_i - f_j))t + (\Vert f_j \Vert^2_D - 2\hat{\rho}(f_j)) \\
         &= \Vert f_i - f_j \Vert^2_Dt^2 + 2(\langle f_i, f_j\rangle_D - \hat{\rho}(f_i - f_j))t + \mathcal{L}(f_j)
  \end{align*}
  So we have that $I(t)$ is a quadratic polynomial. By expanding $2(I(0) - 2I(\frac{1}{2})+I(1))$ we get the coefficient of $t^2$. Considering that $I(0)= \mathcal{L}(f_j) \leq  M + \frac{1}{4}\epsilon$ and $I(1) = \mathcal{L}(f_i) \leq  M + \frac{1}{4}\epsilon$ and that for all $t$ we have that $I(t) \geq M$, since $\mathcal{L} \geq M$ for all functions, then we can apply the following approximation
  \begin{align*}
    \Vert f_i - f_j \Vert^2_D &= 2(I(0) - 2I(1/2)+I(1)) \\
     &\leq 2(M + \frac{1}{4}\epsilon -2M + M + \frac{1}{4}\epsilon) \\
     &\leq \epsilon
  \end{align*}
  Hence, giving us that the sequence $\{f_i\}$ is Cauchy and as such, by the definition of $\overline{\mathcal{H}}(X)$, there exists an element $\Phi = \lim_{i \rightarrow \infty}f_i$ such that $\underline{\mathcal{L}}(\Phi) = M$.
\end{proof}
The fact that this sequence is Cauchy means that it defines an equivalence class of sequences of functions from $\mathcal{H}(X)$, or in other words, it, along with the limit $\lim_{i \rightarrow \infty} f_i$ belong to $\overline{\mathcal{H}}(X)$.

So we have proven the existance of the minimising element $\Phi$ of $\underline{\mathcal{L}}$ in $\overline{\mathcal{H}}(X)$. But we want to understand it a bit better. As things stand, it is simply an equivalence class of Cauchy sequences; not a very usable object. We want to instead show it is an element of $\mathcal{H}(X)$, that is, show it can be identified with an equivalence class of smooth functions (up to the addition of a constant). This implies Weyl's lemma; a theorem that states that every solution of the Poisson equation is a smooth function, which in our case, allows us to conclude our proof of the reverse direction of Theorem \ref{compactPoisson}.

Before we embark on this final part of our proof, let us first show that indeed, we are on the right track and that this minimising element does indeed give a solution to the Poisson equation. The method mirrors what we did before for $\mathcal{L}$ with the arbitrary function $g$, however, now we use $\underline{\mathcal{L}}$ and $\Phi$ instead of an arbitrary function. Recall, earlier on we \emph{assumed} that there existed a function $g$ which minimised $\mathcal{L}$. As things stand, we still dont know that. What we do know however is that we can minimise $\underline{\mathcal{L}}$ by applying it to $\Phi$. Let us compute the first variation of $\underline{\mathcal{L}}$, ie for some $f\in \mathcal{H}(X)$ computing
 \[ \delta\underline{\mathcal{L}} =  \frac{d}{d\epsilon}\biggr\rvert_{\epsilon = 0} \underline{\mathcal{L}}(\Phi + \epsilon f) = 0\]
In an argument, largely the same as before we reach the following conlcusion
\begin{align*}
  (\Phi,f)_D &= \underline{\hat{\rho}}(f) \\
  \lim_{i \rightarrow \infty}\langle \phi_i,f\rangle_D &= \underline{\hat{\rho}}(f) \\
  \lim_{i \rightarrow \infty}\int_X f \Delta \phi_i &= \underline{\hat{\rho}}(f) 
\end{align*}
Since $\underline{\hat{\rho}}$ is a bounded linear functional, we have that the sequence of real numbers, defined by the integrals as the limit is taken to infinity is, in fact, a Cauchy sequence in $\mathbb{R}$ and so, we can precisely write that 
\begin{align*} 
  &\int_X f\Delta\Phi = \underline{\hat{\rho}}(f) \\
  &\int_X f\Delta\Phi - f\rho = 0 \\
  &\int_X f(\Delta\Phi - \rho) = 0
\end{align*}
In other words, $\Phi$ is indeed a solution to the Poisson equation. Such a $\Phi$ is often called a \emph{Weak Solution} to the Poisson equation since it is not necessarily a smooth, or even continuous function.
Now all that is left to do, is show $\Phi$ can be identified with a smooth function. We begin by stating the infamous Weyl's Lemma.
\begin{thm}[Weyl's Lemma]\label{WeylsLemmaCompact}
  Let $X$ be a compact, connected Riemann Surface. If $\rho \in \Omega^2(X)$ such that $\int_X \rho = 0$, then a weak solution to Poissons equation $\phi \in \overline{\mathcal{H}}(X)$ in fact is an element of $\mathcal{H}(X)$, i.e. $\phi$ is a smooth function.
\end{thm}

We begin the proof as we did earlier, when we showed $\hat{\rho}$ is a bounded functional, by considering $\Phi$ on a single open coordinate chart of $\mathbb{C}$ and showing that in local charts, $\Phi$ can indeed by identified with a function and then showing it can be exteneded over the whole of $X$. Then, returning to one coordinate chart, we will show that $\Phi$ is locally smooth, so using a partition of unity to stitch together the local smooth functions obtained from $\Phi$, we obtain a smooth function from $\Phi$ that solves the Poisson equation on the whole of $X$. This argument is based on that from Chapter 10 of \cite{donaldson}. Note that henceforth, we associate to the weak solution $\Phi$, the Cauchy sequence $\{\phi_i\}$ of elements in $\mathcal{H}(X)$ (or, when considering local coordinate charts, $\mathcal{H}(U))$.


We begin by fixing a coordinate chart $U$ of $X$. By adding suitable constants to each $\phi_i$, we can modify the value of $\int_{U} \phi dS_{U}$, where $dS_{U}$ is a fixed area form for $U$, so that the integrals are zero (to make the average value of $\phi_i$ over $U$ zero). Then, by corollary \ref{quotedCorollary6}, we have that, for some $i,j$ the norm $\Vert \phi_i - \phi_j\Vert \leq C \Vert \phi_i - \phi_j\Vert_D$. Since this is a Cauchy sequence in $\mathcal{H}(U)$ then by the completeness of the space of square-integrable functions on $U$ under the usual norm, we have that the Cauchy sequence $\{\phi_i\}$ converges to a square-integrable function $\phi$. Hence we have identified the abstract object $\Phi$ with a square integrable function which we call $\phi$. Square integrable functions, also called $L^2$ functions, are functions for whom $\int |f|^2$ is finite, over the domain of definition.

So now we want to show that this same sequence $\{\phi_i\}$ converges to a $L^2$ function on the whole of $X$ rather than on just $U$. Since we just showed that in any given coordinate chart, $\{\phi_i\}$ converges to a $L^2$ function, we define the set $A \subset X$, to be the set of points $x \in X$ with the property that there exists a coordinate chart around $x$ such that $\{\phi_i\}$ converges to $\phi$ under the usual norm. Then $A$ by the above discussion is non-empty. Since $X$ is connected then we have that the compliment of $A$ is not open unless $A=\emptyset$, which cannot be. Therefore, either $A=X$ and we are done, or there exists a $y \in X$ such that $y \in \bar{A}$, the closure of $A$, but not in $A$. But in that case, we could find a coordinate chart $U^{\prime}$ about $y$ such that for some real numbers $c_i$, we can subtract them from each $\phi_i$ such that $\phi_i - c_i$ converges to the $L^2$ limit in $U^{\prime}$. However, now we have that there is a point $x \in A \cap U^{\prime}$ such that both $\phi_i$ and $\phi_i - c_i$ converge to $\phi$ under the usual norm. Hence, $c_i$ must tend to $0$ as $i \rightarrow \infty$ giving that, in fact, $y \in A$, a contradiction. Hence $A=X$ and we have shown that we can extend $\phi$ to be a $L^2$ function on the whole of $X$. 

Now to show $\phi$ is smooth. To do so we need a local version of Weyl's lemma.
\begin{lemma}[Local Weyl's Lemma]\label{WeylsLemmaLocal}
  Let $U$ be a bounded open set in $\mathbb{C}$ and let $\rho \in \Omega^2(U)$ be a $2$-form on $U$. Suppose $\phi$ is a $L^2$ function on $U$ with the property that, for any smooth function $h$ of compact support in $U$ we have that 
  \[\int_U \Delta h\phi = \int_U \chi\rho\]
  Then $\phi$ is smooth and satisfies the equation $\Delta \phi = \rho$.
\end{lemma}

Since we can stitch together locally smooth solutions using a partition of unity subordinate to our choice of finite cover of coordinate charts of $X$, as we saw earlier, we will focus on proving the statement of the local Weyl's lemma, after which, the full Weyl's lemma follows. First we will try to simplify the problem to the case when $\rho = 0$. 

To do this, we will show that $\phi$ is smooth on $U^{\prime}$, the interior of $U$, where we suppose that for some $\epsilon > 0$, we have an $\epsilon$ neighbourhood of $U^{\prime}\subset U$. We then find a $\rho^{\prime}$ which is equal to $\rho$ on a neighbourhood of the closure of $U^{\prime}$ and of compact support in $U$. If we can find a smooth solution $\phi^{\prime}$ of the equation $\Delta \phi^{\prime} = \rho^{\prime}$ over $U$, then $f = \phi - \phi^{\prime}$ will be a weak solution to $\Delta f = 0 $ in $U^{\prime}$. So, our strategy will boil down to showing that $f$ is smooth, which then implies that $\phi$ must be also be smooth.

So, we suppose $f$ is a weak solution to $\Delta f = 0$ on $U$. This implies that $f$ is a harmonic function (in fact, this is precisely the definition of a harmonic function), an as such, we can use the mean value theorem for harmonic functions. Recall, that lemma \ref{MVT} stated that the value of a harmonic function at the center of a circle in the plane is equal to the average value of the function on the circle. We define a smooth function $\beta:[0,\infty)\rightarrow \mathbb{R}$ such that for $\epsilon > 0$, we have that $\beta(r)$ is a constant for values of $r < \epsilon / 2$ and $0$ for $r \geq \epsilon$ and such that $2\pi \int_0^{\infty}r\beta(r)dr = 1$. Extending $\beta$ to $U \subset \mathbb{C}$ by defining $B(z)=\beta(|z|)$, gives us that $B$ is also smooth and has integral $1$ over the whole of $\mathbb{C}$ since the integral is independent of the angular coordinate and hence, $\int_{\mathbb{C}}B(z)dS=\int_0^{2\pi}\int_0^{\infty}B(r,\theta)rdrd\theta = 2\pi\int_0^{\infty}r\beta(r)dr = 1$. Now if we have a smooth, harmonic function $g$ on a neighbourhood of the $\epsilon$ disc centred around the origin. Then we have that 
\begin{align*}
  \int_{\mathbb{C}}B(0-z)g(z)dS_z &= \int_0^{2\pi}\int_0^{\infty}r\beta(r)g(r,\theta)drd\theta \\
  &= \int_0^{\infty}r\beta(r)\left(\int_0^{2\pi}g(r,\theta)d\theta\right) dr \\
  &= \int_0^{\infty}r\beta(r)2\pi g(0) dr \\
  &= g(0)\left(2\pi\int_0^{\infty}r\beta(r)dr\right) \\
  &=g(0)
\end{align*}
Note that the first integral is the definition of the convolution of $B$ and $g$. So we have that $(B*g)(0) = g(0)$. Since we can shift the coordinate $z$ by any complex number $w$ anywhere, then this convolution is translation invariant (this can be seen by for some $a \in \mathbb{C}$ setting $w = z + a$). Therefore, we can summarise the above in the following proposition.

\begin{prop}
  Let $\psi$ be a smooth function on $\mathbb{C}$, and let $\Delta \psi$ be supported in a compact set $J \subset \mathbb{C}$. Then $B*\psi - \psi$ vanishes outside an $\epsilon$ neighbourhood of $J$ for some $\epsilon > 0$.
\end{prop}

Though it is not immediately clear, if we compute the $L^2$ inner product $\langle \Delta \psi, (B*\psi) - \psi \rangle$ (with $B$ and $\psi$ as in the proposition), then we get $-\langle \Delta \psi, \psi \rangle + \langle \Delta \psi, B*\psi \rangle$ which equals zero outside of $J$ since $\Delta \psi$ has as support $J$ and as we showed above, in an $\epsilon$ neighbourhood around $J$, $B$ has its support, so both inner products are defined and are zero outside of $J$ but are equal within $J$.  

Therefore we can use this proposition to help us show that our $f$ is smooth, since, if $f$ were smooth on $U$, then $B*f = f$ on the interior set $U^{\prime}$. Additionally, by the properties of convolutions, the convolution of any $L^2$ function with $B$ is smooth on $U$ since, $B$ itself is smooth. This gives us the condition we want; proving the smoothness of $f$ in $U^{\prime}$ is the equivalent to showing that $B*f = f$ at all points in $U^{\prime}$. Simply put, if we can show $B*f - f = 0$ on $U^{\prime}$, then $f$ must be smooth. 

We begin by considering the inner product of $f-B*f$ with a smooth function, $\chi$, of compact support in $U^{\prime}$, \[\langle \chi, f-B*f\rangle = \int_{U^{\prime}} (\chi)(f-B*f)dS = 0 \] for any choice of area form $dS$ on $U$. 

Let $h = V*(\chi-B*\chi)$ where $V(z) = \frac{1}{2\pi}\log(|z|)$, the Newtonian potential. We want to show that $h$ has compact support in $U^{\prime}$. We want to show that we can use this $h$ as the $h$ in the hypothesis of the local Weyl's Lemma, ie, show it has compact support in $U$. Expanding this gives us that $h = V*\chi - V*B*\chi$. Since $V*\chi$ is a smooth function on $\mathbb{C}$ and by the lemma on the properties of convolutions of the Laplacian with the Newtonian potential (Lemma \ref{ConvProperties}), we have that $\Delta V*\chi = \chi$. Hence, $\Delta V*\chi$ must vanish outside the of the support of $\chi$ and so $B*V*\chi = V*\chi$ outside the $\epsilon$ neighbourhood of the support of $\chi$. Therefore, $h$ has compact support, contained in $U$, since $\chi$ has by hypothesis. Using the hypothesis of the local Weyl's lemma we have that $\langle \Delta h, f\rangle = 0$. Now we note that $\Delta h = \Delta V*(\chi - B*\chi) = \chi - B*\chi$, again since both $B$ and $\chi$ are compactly supported. Hence, we have that 
\begin{align*}
  \langle \chi - B*\chi, f \rangle = 0 
\end{align*}
which when we apply the triple product identity, gives us 
\begin{align*}
  \langle \chi, f - B*f \rangle = 0 
\end{align*}
which implies that $f-B*f = 0$ on in $U^{\prime}$. Hence, $f$ is indeed a smooth solution to $\Delta f = 0$, finally implying that $\phi$, our $L^2$ function, is in fact, a smooth solution to the Poisson equation.
To then extend this solution over the whole of $X$, we must pick a finite collection of coordinate charts, and pick a partition of unity subordinate to this cover. Once we identify this solution with a $2$-form on $X$, by fixing an area form, then we proceed as in the second part of the proof of proposition \ref{PartitionOfUnity}. \qed

\section{Consequences of solving the Poisson equation on compact Riemann surfaces}
The implications of theorem \ref{compactPoisson} are not immediately obvious. Let us recall the statement of this theorem.
\begin{thm*}
  Let $X$ be a connected, compact Riemann surface. Let $\rho \in \Omega^2(X)$ be a $2$-form on $X$. Then, there exists a unique, smooth solution $f \in \Omega^0(X)$ to the equation $\Delta f = \rho$, up to the addition of a constant, if and only if $\int_X \rho = 0$. This equation is called the Poisson equation.
\end{thm*}
So how can we use the guaranteed existence of such a function $f$ for a given $2$-form to help us classify connected, compact Riemann surfaces? We begin this section by using this result to help us prove four very useful isomorphisms between various cohomology groups of $X$.

\begin{thm}
  Let $X$ be a compact, connected Riemann surface. Then we have the following isomorphims:
  \begin{itemize}
    \item The induced map $\sigma: H^{1,0}(X) \rightarrow \overline{H^{0,1}}(X)$, induced by the mapping $\alpha \mapsto \overline{\alpha}$ is an isomorphism
    \item The bilinear map $B:H^{1,0}(X)\times H^{0,1}(X) \rightarrow \mathbb{C}$ defined by $B(\alpha, \tilde{\theta})=\int_X \alpha \wedge \theta$ gives an isomorphism between $H^{0,1}(X)$ and the dual space $(H^{1,0}(X))^*$
    \item Define the mapping $\mu:H^{1,0}(X) \oplus H^{0,1}(X) \rightarrow H^1(X)$ given by $\mu(\alpha, \beta) = i(\alpha) + \overline{i(\sigma^{-1}(\beta))}$, where $i:H^{1,0}(X) \rightarrow H^1(X)$ is the inclusion map, defined by mapping a holomorphic $1$-form to its cohomology class in $H^1(X)$. The map $\mu$ defines an isomorphism.
    \item The map $\nu:H^{1,1}(X) \rightarrow H^2(X)$ defined as the natural inclusion of $Im(\overline{\partial}:\Omega^{1,0}(X)\rightarrow \Omega^2(X))$ in $Im(d:\Omega^1(X)\rightarrow \Omega^2(X))$ defines an isomorphism.
  \end{itemize}
\end{thm}
\begin{proof}
  We take these each in turn.
\end{proof}
So as we have seen, these ismorphisms which provide nice relations between the real and complex cohomology groups of our connected, compact Riemann surface, depend on this solution to the Poisson equation existing. But what can we say about their classification? We notice that the third isomorphism, imply that the cohomology groups $H^{1,0}(X)$ and $H^{0,1}(X)$ not only have the same dimension, but when combined with the first isomorphism, we get that each share half the value of the dimension of $H^1(X)$. We showed earlier that for a connected, compact genus $g$ smooth surface, $\Sigma_g$, we had that $H^1(\Sigma_g)\cong\mathbb{R}^{2g}$. Therefore, we have that $dim_{\mathbb{R}}H^{1,0}(X)=dim_{\mathbb{R}}H^{0,1}(X)=g$. So our Riemann surface $X$ can be classified by genus. Or can it? Cohomology unfortunately is not a strong enough condition, as two cohomological Riemann surfaces, \emph{may} not be confomally equivalent. We still need to show that all compact, connected genus $g$ surfaces have compatible Riemann surface structures, i.e that we can find a conformal map between any of them. This next proposition allows us to say something about that.
\begin{prop}\label{meroFunctionOnGenusGSurface}
  Let $X$ be a connected, compact Riemann surface with $H^{0,1}(X)$ of finite dimension $g$. Then, given any distinct $g+1$ points on $X$, $p_1,\ldots, p_{g+1}$, there exists a meromorphic function on $X$ with a simple pole at some (perhaps all) of the points $p_1,\ldots, p_{g+1}$.
\end{prop} 
As we noted in our first attempts at classifying Riemann surfaces, we found a result that allows us to distinguish if, given a Riemann surface $X$, is it confomally equivalent to the Riemann sphere. We showed that if we could construct a meromorphic function on $X$ such that it has a simple pole at a point $p \in X$ then it must be conformally equivalent to the Riemann sphere. The above stated proposition, proposition \ref{meroFunctionOnGenusGSurface}, generalises this result considerably by constructing such a function directly and generalises it for all genus $g$ Riemann surfaces. This therefore means that we can classify \emph{all} connected, compact Riemann surfaces simply by considering their genus.
\begin{proof}
  We begin by considering a single point $p \in X$. This implies by hypothesis that we have a genus $0$ Riemann surface. We now pick a coordinate chart around $p$ such that $p$ maps to $0$, and let $U$ be a small disc encircling $p$. We define $\beta(z)$ to be a smooth function on $U$ such that $\beta$ is supported on $U$. More precisely, for some fixed $\epsilon > 0$, let us write $\beta$ as 
  \[
  \beta(z)=
  \begin{cases}
    1, &|z| \leq \frac{1}{4}\epsilon \\
    0, &|z| > \frac{3}{4}\epsilon
  \end{cases}
  \]
  
  We define a function, $f^{\prime}(z) = \beta(z)z^{-1}$ and note that $f^{\prime}$ is defined on $U$. However, very easily, we can extend the domain of definition of $f^{\prime}$ by noticing that as $|z|$ grows in $U$, $f^{\prime}$ goes to $0$. Therefore, we can extend $f^{\prime}$ to the whole of $X$. We call this extension to $X$, $f(z)$ and define $f$ such that 
  
  \[
    f(z)=
    \begin{cases}
      0, &z \in X \setminus U \\
      f^{\prime}(z), &z \in U
    \end{cases}
    \]
  We call this process "extending $f^{\prime}$ to $X$ by zero". We further note that $f^{\prime}$ is holomorphic in the region $|z| < \epsilon/4$ since $f^{\prime} = z^{-1}$ in this region. We lose this fact as soon as $\beta$ stops being constant however since we can't impose that $\beta$ decays to zero holomorphically. Hence $f$ is holomorphic in this region too.

  We can also construct a useful $(0,1)$-form on $X$ from $f$, namely $A=\overline{\partial}f(z) = \overline{\partial}(\beta(z))z^{-1}$.
  By definition, $A$ is an element of $\Omega^{0,1}(X\setminus \{p\})$. We would like to extend $A$ over the whole of $X$ and undertsand this extensions' cohomology class. Namely, we want to understand the class $[A] \in H^{0,1}(X)=\frac{\Omega^{0,1}(X)}{Im(\overline{\partial}:\Omega^0(X) \rightarrow \Omega^{0,1}(X))}$.


  We see from the definition of $H^{0,1}(X)$ that we can find a smooth function $g$ on $X$ such that $\overline{\partial}(g) = A$ if and only if $[A] = 0$. An interesting and important fact to note is that, if we multiply the function $f$ by some $\lambda \in \mathbb{C}$ then our $(0,1)$-form $A$ becomes $\lambda A$ and hence, any associated cohomology class becomes $\lambda[A]$. We see that $A$ can be easily extended over the whole of $X$. In the annulus $\epsilon /4 < |z| \leq 3\epsilon /4$, we see the $A$ has some value corresponding to the rate of decay of $\beta$. So let us consider when $|z| \leq \epsilon /4$. In this case, we see that $\beta$ is constant so $\overline{\partial}\beta = 0$ for all $|z| \leq \epsilon /4$. Thus, the limit of $A$ as $z$ approaches $0$ vanishes and so we have extended $A$ over the whole of $X$. Thus, we have that because $X$ is a genus $0$ surface, $H^{0,1}(X)=0$ and, $[A] = 0$, and so there exists a smooth function $\psi \in \Omega^0(X)$ such that $\overline{\partial}\psi = A$. 

  Now, we let $\phi$ be a function defined as $\phi = \psi + \lambda f$ for some $\lambda \in \mathbb{C}$. $\psi$ is defined to be smooth over all of $X$ and $f$ is smooth on $X\setminus \{p\}$ so $\phi$ is also smooth on $X \setminus \{p\}$. Since we have that $\phi$ is smooth, if we can show that this $\phi$ is holomorphic near $p$ then it is precisely the meromorphic function we need. We know that $f$, on the punctured disc $\{|z| \leq \epsilon / 4\} \setminus \{p\}$, is holomorphic, therefore so is $\lambda f$. 
  
  Therefore, if we can show that $\psi$ is holomorphic on this disc, then $\phi$ will be holomorphic on the punctured disc. But we know that on $X$, we have $\overline{\partial}\psi = A$ and except for in the annulus $\epsilon / 4 < |z| \leq 3\epsilon / 4$ we have that $A=0$. In particular, on the disc of radius $\epsilon / 4$ we have that $A=0$ and so $\overline{\partial}\psi = 0$. Hence, $\psi$ is holomorphic and so it follows that in this punctured disc, $\phi$ is holomorphic. Therefore, in this disc around $p$, the function $\phi$ is meromorphic, with a pole at $p$ and smooth over $X$, as required.

  We can now generalise this construction, whereby for $g+1$ points $p_1,\ldots, p_{g+1}$, we can repeat the exact same procedure, creating functions $f_1,\ldots,f_{g+1}$ such that they extend by zero over $X$ and have a single, simple pole at the points $p_1,\ldots,p_{g+1}$ respectively. Then we can define forms $A_i$ and show that since the cohomology group $H^{0,1}(X) \cong \mathbb{R}^g$ (implying it is $g$ dimensional), then the linear combination $\lambda_1[A_1]+\ldots+\lambda_{g+1}[A_{g+1}] = 0$, is a linear dependency, implying that the equality to zero is had even if not all $\lambda_i = 0$. Then we proceed as before with this linear combination equalling zero, rather than with $[A]=0$ as we did in the single point case, thereby constructing a meromorphic function on $X$ with $g+1$ simple poles. 
\end{proof}
So we see that the existance of a linear dependency of the cohomology classes of $(0,1)$-forms over their respective poles, gives us a link between the genus of $X$ and the number of poles a meromorphic function must have on $X$. The number of points which have a pole over them correspond to which of the $\lambda_i \neq 0$ in the linear dependency. This result is incredibly powerful as it unifies the topological notion of a genus and turns it into an invarient of connected, compact Riemann surfaces. Hence, we can state the following corollary.
\begin{cor}\label{ClassificationByGenus}
  Let $X$ be a connected, compact Riemann surface. 
  \begin{itemize}
    \item If $X$ has genus $0$, then $X$ is confomally equivalent to the Riemann Sphere. 
    \item If $X$ has genus $1$, then $X$ is conformally equivalent to a Torus.
    \item If $X$ has genus $n$, then $X$ is conformally equivalent to an orientable surface of genus $n$.
  \end{itemize}
\end{cor}

The latter point holds due to the fact that we have equated the classification of closed, orientable surfaces and compact, connected Riemann surfaces. For an in depth discussion on the classification of surfaces, see.

We conclude this section with the observation that the first bullet point in Corollary \ref{ClassificationByGenus}. If we have a connected, simply connected, compact Riemann surface, it is conformally equivalent to the Riemann sphere $\hat{\mathbb{C}}$.

\section{The Uniformisation Theorem}
We are finally ready to discuss the main topic of this dissertation. In fact, due to all the hard work we have put in to understanding holomorphic and harmonic functions, differential forms and the general topology involved in studying Riemann surfaces, stating and proving the Uniformisation theorem becomes quite a bit easier as we shall see. 

For the moment, let us assume that we have also classified all non-compact, simply connected Riemann surfaces. Recall when we introduced the Poisson equation, we said that we would be solving it for two separate cases; the first being the compact connected case, and the second being the non-compact, simply connected case. In a similar vein to how working with the Poisson equation helped us classify compact, connected Riemann surfaces, it will also help us classify non-compact, simply connected Riemann surfaces. So for now, we assume that we have done this classification. It turns out that there are two non-compact, simply connected Riemann surfaces; the complex plane $\mathbb{C}$ and the upper half plane $\mathbb{H}$.
Now we can state the uniformisation theorem. We shall proceed in a similar way to that of the start of chapter 10 of \cite{donaldson}.

\begin{thm}[The Uniformisation Theorem]\label{Uniformisation}
  Let $X$ be a connected, simply connected Riemann surface. 
  \begin{itemize}
    \item If $X$ is compact, it is conformally equivalent to $\hat{\mathbb{C}}$, the Riemann Sphere. 
    \item If $X$ is not compact, the $X$ is conformally equivalent to either $\mathbb{C}$, the complex plane or $\mathbb{H}$, the upper half plane.
  \end{itemize}
\end{thm}

Since we have classified all connected, simply connected Riemann surfaces, we can actually classify all Riemann surfaces. This follows since any Riemann surface is confomally equivalent to its universal cover (recall, the universal cover is a simply connected space) quotiented by some group action of the fundamental group on the universal cover (called a Deck transformation).
\begin{cor}
  Any connected Riemann surface is conformally equivalent to one of the following:
  \begin{itemize}
    \item The Riemann Sphere, $\hat{\mathbb{C}}$
    \item The Complex Plane $\mathbb{C}$, the Cylinder $\mathbb{C}/\mathbb{Z}$ or the Torus $\mathbb{C}/\Lambda$, for some lattice $\Lambda \subset \mathbb{C}$
    \item The Upper Half Plane quotiented by a freely acting discrete subgroup of $\Gamma \subset PSL(2,\mathbb{R})$, $\mathbb{H}/\Gamma$ 
  \end{itemize}
\end{cor}
So how does Poissons equation for simply connected, non-compact Riemann surfaces come into the proof of the uniformisation theorem? Let us first state the theorem, similar to the compact, connected case, but for the simply connected, non-compact case.
\begin{thm}\label{NonCompactPoissons}
  Let $X$ be a connceted, simply connected, non-compact Riemann surface. Then if $\rho$ is a $2$-form of compact support on $X$ (ie $\rho \in \Omega^2_c(X)$) with $\int_X \rho = 0$ then there is a smooth function $\phi \in \Omega^0(X)$ such that $\Delta \phi = \rho$, with $\phi$ tending to $0$ at infinity in $X$.
\end{thm}
That last part about tending to $0$, is a much weaker condition than compact support. Let us make that notion precise.
\begin{defn}[A function tending to a value at infinity on $X$]
  Let $X$ be a connected, simply connected, non-compact Riemann surface. If we have a smooth function $f \in \Omega^0(X)$, and a real number $c \in \mathbb{R}$, then we say that $f$ tends to $c$ at infinity in $X$ if, for all $\epsilon > 0$ there is a compact subset $K \subset X$ such that $|f(x) - c| < \epsilon$ for $x \notin K$.

  Conversely, we say that $f$ tends to $+\infty$, at infinity in $X$ if for all $C \in \mathbb{R}$, there is a compact set $K \subset X$ such that $f(x) > C$ for all $x \notin K$ (and we say that $f$ tends to $-\infty$ in $X$ if $-f$ tends to $+\infty$).
\end{defn}

Now, we want to use the fact that we have a solution to Poissons equation on simply connected, non-compact Riemman surfaces, to prove the uniformisation theorem. That way, rather than prove the uniformisation theorem directly, we can prove the statement of Poissons equation, which will imply uniformisation for non compact surfaces. So, let us show that given the statement of Theorem \ref{NonCompactPoissons}, we can classify all simply connected non-compact Riemann surfaces.

\section{The Poisson equation on simply connected, non-compact Riemann surfaces}

%
%\references{Bibliography}{}{yes}
\begin{thebibliography}{99}
\bibitem{donaldson} S. Donaldson, {\em Riemann Surfaces},
\bibitem{arnold} V. I. Arnol'd, {\em Lectures on Partial Differential Equations},
\bibitem{rudin} W. Rudin, {\em Functional Analysis},
\bibitem{babyRudin} W. Rudin, {\em Principles of Mathematical Analysis}, 

\bibitem{comfun} Jones \& Singerman. {\em Complex Functions}, Cambridge University Press (DATE).
\bibitem{calcohomo} I. Madsen \& J. Tornehave, {\em From Calculus to Cohomology}, Cambridge University Press (1997)
\bibitem{spivak} M.Spivak, {\em Calculus on Manifolds}, 

\bibitem{notes} The alternative to Donaldsons Notes

\bibitem{electromagentismBook} The EM book

\bibitem{Hatchers} Hatchers book
%
\bibitem{website} MacTutor History of Mathematics Archive, at https://mathshistory.st-andrews.ac.uk/ [accessed 9 May 2030]


\end{thebibliography}

\end{document}
