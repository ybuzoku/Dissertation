\documentclass[a4paper,12pt]{report}
%\setcounter{tocdepth}{1}%
\usepackage{amssymb,amsmath,amsthm,cmbright}
\usepackage{amsfonts,amsmath,amssymb,graphicx}
\usepackage{bbkproject}
\usepackage{hyperref}
\theoremstyle{plain}
\renewcommand{\thefootnote}{\arabic{footnote}}
\newtheorem{thm}{Theorem}[section]
\newtheorem{lemma}[thm]{Lemma}
\newtheorem{prop}[thm]{Proposition}
\newtheorem{cor}[thm]{Corollary}
\theoremstyle{definition}
\newtheorem{defn}[thm]{Definition}
\newtheorem{example}[thm]{Example}

\subject{Mathematics} %write your degree subject here
\degree{M.Sc.} %write your degree (eg M.Sc., Ph.D.)
\thesis{dissertation} %if it's a thesis, write thesis instead
\title{A survey of Riemann surfaces and their classification}
\author{Y. Buzoku}
\supervisor{Dr. B. Fairbairn} 
\department{Department of Economics, Mathematics and Statistics}%
\submissiondate{$30^{\mathrm{th}}$ September 2021}%
\institution{Birkbeck, University of London}%

%
\parindent 0pt
%
\begin{document}
\maketitle{}{}{I have found a beautiful proof, but this abstract is too short to contain it.}
%
\chapter{Introduction}

The aim of this dissertation is to investigate and gain a deeper
understanding of
Riemann Surfaces, their construction, their classification and an analysis
of the
geometry of some classes of these surfaces. Doing so will bring together
ideas
from different, seemingly unrelated fields of mathematics. Often times,
Riemann
Surfaces are studied solely due to these connections as they might not
occur in
higher dimensional analogues. We begin this dissertation with an
introduction to
Riemann surfaces as classical complex-analytical manifolds, and then
showing that
we can define them equivalently as algebraic varieties of complex curves.
We then 
study briefly the automorphisms of a particular set of such Riemann
Surfaces, and 
give a short review of topological notions that will be important in our 
classification later, including quotient spaces. We then show how can
endow a 
Riemann surface structure on such spaces, and give examples of spaces
arising in such a way, and their lifts to their universal covers.

From there we begin to classify Riemann surfaces by stating the
uniformisation 
theorem, and classifying Riemann surfaces with universal cover the Riemann
sphere. We then introduce Teichm\"{u}ller spaces and discuss Riemann
surfaces with the complex plane as universal cover.

The remainder of this dissertation will then focus on the infinite family
of Riemann surfaces with the hyperbolic plane as universal cover. A
discussion of the geometry of some compact Riemann surfaces follows, with
some anecdotal but interesting ideas introduced, such as the interplay
between cubic graphs and Riemann surfaces, naturally arising hyperbolic
Teichm\"{u}ller spaces, and Hubers Theorem relating the length spectrum of
hyperbolic Riemann surfaces and the eigenvalues of the Laplacian on said
surfaces. We conclude with a discussion of the heat kernel on the sphere,
plane and hyperbolic plane and prove the uniformisation theorem for
compact Riemann surfaces and the claim that there exist exactly two
non-compact simply connected Riemann surfaces.

\section{Elementary definitions and notation}
We begin by setting up the notation and tools we will be using from Complex Analysis and Topology.

\chapter{Riemann Surfaces}

Intro to Riemann Surfaces. Talk about $log(z)$ and it as a motivational entry into the subject.


\section{Basic Definitions}\label{bdefns}

We begin by stating the basic definitions of Riemann Surfaces and the 
analogs of some important notions from Complex Analysis in Riemann Surface
Theory.

\begin{defn}[Riemann Surface]\label{rsdefn}
A Hausdorff topological space $X$ is said to be a Riemann Surface if:
\begin{itemize}
\item There exists a collection of open sets $U_{\alpha} \subset X$, where
  $
\alpha$ 
ranges over some index set such that $\bigcup\limits_{\alpha} U_{\alpha}$
cover 
$X$.
\item There exists for each $\alpha$, a homeomorphism, called a chart map,
  $ \psi_{\alpha}\colon U_{\alpha} \rightarrow \tilde{U}_{\alpha}$, where
$\tilde{U}_{\alpha}$ is 
  an open set in $\mathbb{C}$, with the property that for all $\alpha$,
  $\beta$, the composite map $\psi_{\alpha} \circ \psi_{\beta}^{-1}$ is
  holomorphic on its 
domain of 
definition. (These composite maps are sometimes called transition maps).
\end{itemize}
We call the triple $(\{U_\alpha\},\{\tilde{U}_{\alpha}\},
\{\psi_\alpha\})$ an 
atlas of 
charts for the Riemann Surface $X$, though we also use the common
notation $(U,
\tilde{U}, \psi)$ to denote this, where $U=\{U_\alpha\}$,
$\tilde{U}=\{\tilde{U}
_{\alpha}\}$ and $\psi=\{\psi_\alpha\}$.
\end{defn}


\section{Algebraic Curves}\label{algcurv}
ALGBRAIC CURVES ARE COOOOOL MAN!

\section{Proper Discontinuous Group Actions}\label{PropDiscGrpAct}
Geometry and groups episode three, the revenge of the fundamental domain.
\subsection{Fuchsian grouppos}
Empty something.

\section{Attempting to classify some Riemann surfaces}
\subsection{The Riemann Sphere}
\subsection{Quotients of the Complex Plane}
\subsubsection{The Cylinder}
\subsubsection{The Torus}
\subsection{Moduli space of Tori}

\chapter{Calculus on Riemann Surfaces}
\section{Introduction}
To gain a deeper understanding of Riemann Surfaces from a more analytical
and geometric perspective, we wish to introduce notions from calculus to
such surfaces. This involves defining the notions of integration,
differential forms and vector fields on Riemann Surfaces. As we will see
however, these notions give rise to certain groups and vector spaces
which will be very important in the following sections, especially in the
proof of the uniformisation theorem.

\section{The Tangent and Cotangent Spaces}
We begin our study of calculus on Riemann surfaces by defining notions of
the tangent and cotangent spaces. The following definitions may be found
in Chapter 5 of \cite{donaldson} unless otherwise noted.

\begin{defn}[Smooth path on a Riemann Surface]\label{Smooth Path}
  Let $X$ be a Riemann Surface and let $\epsilon > 0$. Then we say
  $\gamma:(-\epsilon,\epsilon) \rightarrow X$ is a smooth path on $X$ if
  we can define $\frac{d\gamma}{dt}$ for all $t \in (-\epsilon,\epsilon)$.
\end{defn}

\begin{defn}[Tangent Space of a Riemann Surface]\label{TpX}
  Let $X$ be a Riemann surface. Let $p \in X$ be a point on the Riemann
  surface. We define $T_pX$, the tangent space of $X$ at
  $p$, to be the
  space of equivalence classes of smooth paths $\gamma$ through $p$ such
  that $\gamma(0)=p$. Two paths $\gamma_1, \gamma_2$ are said to be
  equivalent if $\frac{d\gamma_1}{dt}=\frac{d\gamma_2}{dt}$.
\end{defn}

\begin{defn}[Cotangent Space of a Riemann Surface]\label{T*pX}
  Let $X$ be a Riemann surface. For all $p \in X$ we define the real
  cotangent space at $p$ to be
  $T^*_pX = Hom_{\mathbb{R}}(T_pX, \mathbb{R})$ and the complex cotangent
  space at $p$ to be
  $T^*_pX^{\mathbb{C}} = Hom_{\mathbb{R}}(T_pX,\mathbb{C})$.
\end{defn}

\section{The Alternating Algebra and Differential forms}

\begin{defn}[Alternating Algebra]\label{AltAlg}
  cf Calc to cohomo and Spivak  
\end{defn}

\begin{defn}[Differential $p$-forms]\label{p-form}
  Calc to cohomo. ALL FUNCTIONS SMOOTH AND REAL HERE! We only need to consider $p=0,1,2$
\end{defn}
  Space of differential $0$-form is also written as $C^{\infty}(U)$
\begin{defn}[Exterior Derivative]\label{exteriorD}
  aaa
\end{defn}

\begin{defn}[Integration]\label{Integration}
  Define integration 
\end{defn}

\section{de-Rham Cohomology}

\begin{defn}[de-Rham Cohomology]\label{deRham}
  aaa
\end{defn}

\begin{defn}[Complex valued forms and de-Rham Complexes]\label{Cforms}
  aaa
\end{defn}

\section{Complex structures}

\begin{defn}[Complex structure]\label{Cstructure}
  
\end{defn}

\begin{lemma}[Structure splitting lemma]
  Any $\mathbb{R}$-linear map from $V$ to $\mathbb{C}$ can be written in a unique way as a sum of complex linear and antilinear maps.
\end{lemma}

\begin{lemma}[Can split $\Omega^1(X,\mathbb{C})$ into direct sum]
  
\end{lemma}

\begin{defn}[$L_2$ norm and inner-product on space of $(1,0)-$forms]\label{InnerProduct}
  
\end{defn}
\chapter{The Uniformisation Theorem}
\section{Preliminaries}
\subsection{The $\Delta$ operator and Harmonic functions}
We begin our treatment of the uniformisation theorem, with a discussion of a particular and extremely important differential operator on Riemann Surfaces; the Laplacian. The majority of definitions can be found in \cite{donaldson} unless explicitly stated.

\begin{defn}[The Laplacian]\label{LaplacianDef}
  Let $X$ be a Riemann Surface. We define a linear map $\Delta:\Omega^0(X)\rightarrow \Omega^2(X)$, where $\Delta=2i\bar{\partial}\wedge\partial$. We call this linear map the Laplacian. Often, we will suppress the $\wedge$ and simply write $\Delta=2i\bar{\partial}\partial$, with the $\wedge$ being implied, where no confusion may arise from such notation.
\end{defn}

Note that though, we have said that $\Delta$ is a map, it is a map from a space of functions to a space of functions, and hence is technically an operator; in particular, since it defines a differential equation in local coordinates, it is a differential operator and hence will be referred to as so.

In local co-ordinates, $\Delta$ simply recovers the expected Laplacian of a function since for a given function $f\in \Omega^0(X)$ we get that 
\begin{align*}
  \Delta f &= 2i\bar{\partial }\partial f = 2i(\frac{\partial}{\partial\bar{z}})(\frac{\partial}{\partial z})f(d\bar{z}\wedge dz) \\
  &=\frac{i}{2}(\frac{\partial}{\partial x} + i\frac{\partial}{\partial y})(\frac{\partial}{\partial x} - i\frac{\partial}{\partial y})f(2idx\wedge dy) \\
  &= -(\frac{\partial^2 f}{\partial x^2} + \frac{\partial^2 f}{\partial y^2})dx\wedge dy 
\end{align*}
which when we equate $dx\wedge dy$ with the usual area from $dxdy$ becomes the usual Laplacian from vector calculus, up to a minus sign. We usually do this so we can identify functions with two-forms. For example, when given a function $f(x,y)$, it has an associated two-form $\rho = f(x,y) dx\wedge dy$, which can be integrated. Though we can define a notion of the Laplacian without the minus sign, as we will shortly see, that minus sign is important to our setup for the proof of the uniformisation theorem. Hence, we will leave it so. 

A special class of function that is also important to our theory is that of the 
Harmonic functions. They play a crucial role in studying the Laplacian operator generally and have a number of very nice properties that we will be exploiting in the coming section.
\begin{defn}[Harmonic function]\label{HarmonicDef}
  Let $X$ be a Riemann surface. A function $f\in \Omega^0(X)$ is said to be harmonic if $\Delta f = 0$. A  function $g:X \rightarrow \mathbb{C}$ is said to be harmonic if both its real and imaginary parts are real valued harmonic functions.
\end{defn}
We note that we can relax the differentiability condition considerably to make the functions only twice differentiable. Thus harmonic functions need to be at least twice differentiable. Furthermore, the domain of definition is not restricted to just Riemann surfaces but can also be any open subset of $\mathbb{R}^n$ or $\mathbb{C}^n$ for any $n\geq 1$. 

Let us give some examples of Harmonic functions.
\newpage

\begin{example}[Examples of various Harmonic functions on different domains]
  Over open sets in $\mathbb{C}$ we have:
  \begin{itemize} 
    \item Constant functions are trivially harmonic.
    \item Any holomorphic function is automatically a harmonic function since its real and imaginary parts satisfy the Cauchy-Riemann equations.
    \item $f(z)=e^z=e^{x+iy}=e^xsin(y)$
  \end{itemize}
  Taking our domain of definition to be open subsets of $\mathbb{R}^3$, we have:
  \begin{itemize} 
    \item $f(x,y,z)=\frac{1}{\sqrt{x^2+y^2+z^2}}$
  \end{itemize}
  Finally, on the punctured plane $\mathbb{C}\setminus \{0\}$, we also have a nice family of examples:
  \begin{itemize}
    \item $f(x+iy) = ln(x^2 + y^2)$ and hence $g(x+iy)=K ln(x^2+y^2)$ for all $K \in \mathbb{R}$.
  \end{itemize}
  All of the aforementioned functions satisfy the equation $\Delta f = 0$ on their domain of definition. This concludes the example.
\end{example}

It is enlightening to see the claim that any holomorphic function is harmonic using our language of differential forms.
\begin{lemma}\label{HolIsHarm}
  Holomorphic functions are harmonic functions, ie they have harmonic real and imaginary parts. 
\end{lemma} 
\begin{proof}
  Let $X$ be a Riemann surface and $f$ a holomorphic function defined on $X$.
  Let us consider the Laplacian applied to the sum $\frac{1}{2i}(f \pm \bar{f})$.
  \[\frac{1}{2i}\Delta(f \pm \bar{f}) = \bar{\partial}\partial(f \pm \bar{f})=-\partial(\bar{\partial}f) \pm \bar{\partial}\overline{(\bar{\partial}f)}=-\partial(0) \pm \bar{\partial}(0) = 0 \pm 0 = 0\]
  where both brackets $\bar{\partial} f = 0$ because $f$ is holomorphic.
\end{proof}

Note however, that whilst holomorphic implies harmonic, the converse isn't always true. However we can say the following.

\begin{lemma}\label{HarmRealHol}
  Let $X$ be a Riemann Surface, $U$ and open set around a point $p \in X$ and let $\phi \in \Omega^0(U)$ be a harmonic function. Then there exists an open neighbourhood $V \subset U$ of the point $p$ and a holomorphic function $f \colon V \rightarrow \mathbb{C}$ with $\phi = Re(f)$.
\end{lemma}
Since this is local result on $X$, then the proof follows exactly as in classical complex analysis. However, we wish to showcase how using calculus on Riemann surfaces can help solve problems efficinetly and as such we provide a proof using the tools we have established.
\begin{proof}
  Let $A \in \Omega^1(U)$ be a real one-form such that: 
  \begin{align*}
    A &= -\frac{\partial \phi}{\partial y} dx + \frac{\partial \phi}{\partial x} dy \\ &= i\bar{\partial}\phi + \overline{(i\bar{\partial}\phi)}
  \end{align*}
  Since $\phi$ is harmonic, so $\bar{\partial}\partial \phi = 0$ and $d = \partial + \bar{\partial}$, we get that $dA = 0$. Hence, if $V$ is an open, simply-connected set (such that $H^1(V)=0$), then we can find a function $\psi \in \Omega^0(V)$ such that $d\psi = A$. By equating coefficients we get that $\partial \psi = -i \partial \psi$ and $\bar{\partial} \psi = i \bar{\partial} \phi$. So if we construct a function $f \colon V \rightarrow \mathbb{C}$ where $f = \phi + i \psi$, we see that $\bar{\partial}f = \bar{\partial}(\phi + i\psi) = \bar{\partial}\phi +i(i\bar{\partial}\phi) = 0$ and hence, $f$ is a holomorphic function whose real part is $\phi$. 
\end{proof}

Harmonic functions are clearly quite well behaved and have nice properties that are not generally exhibited by other functions. These properties, such as the maximum principle which we will define and prove below, play a crucial role in our proof of the setup for the uniformisation theorem.

To do this, we need a version of the mean value theorem for harmonic functions.
\begin{lemma}[The Mean Value Theorem for Harmonic Functions]\label{MVT}
  Let $U$ be a domain in $\mathbb{C}$ around the a point $a \in U$ and let $\phi \in \Omega^0(U)$ be harmonic on $U$. Let $\gamma$ be a closed circle of radius $R > 0$ contained within $U$ that encircles the point $a$. Then, the mean value of $\phi$ over the set bounded by $\gamma$ is equal to the value of $\phi$ at the point $a$. 
\end{lemma}
We note that while the statement of this lemma talks about domains of $\mathbb{C}$, we could equally talk about domains on any Riemann surface as by definition of a Riemann Surface, we have chart maps that map open sets of a Riemann Surface to domains in $\mathbb{C}$. 
\begin{proof}
  Since $\phi$ is a real-valued harmonic function, then by lemma \ref{HarmRealHol} then we have a holomorphic function $f$ such that the real part of $f$ is $\phi$. We now consider the statement of Cauchys Integral formula about $\gamma$.
  \begin{align*}
    f(a) = \frac{1}{2\pi i}\int_{\gamma}\frac{f(z)}{z}dz
  \end{align*}
  Parameterising by $\gamma$ gives us that $z= a + Re^{i\theta}$ for $\theta \in [0, 2\pi)$ on $\gamma$ and that $dz = iRe^{i\theta}$. Plugging these into the above equation gives us the relation that
  \begin{align*}
    f(a) &= \frac{1}{2\pi i}\int_{\gamma}\frac{f(z)}{z}dz = \frac{1}{2\pi i}\int_0^{2\pi}\frac{f(z)}{Re^{i\theta}}iRe^{i\theta}d\theta \\
    &= \frac{1}{2\pi}\int_0^{2\pi}f(z)d\theta
  \end{align*}
  Since $\phi$ is the real part of $f$, we can split $f$ into real and imaginary parts with the integrals splitting into real and imaginary parts. Hence, we get that 
  \begin{align*}
    \phi(a)=\frac{1}{2\pi}\int_0^{2\pi}\phi(z)d\theta
  \end{align*}
    which implies that the average value of $\phi(z)$ along $\gamma$ equals $\phi(a)$ giving us our mean value theorem.
\end{proof}

Now we are in a position to discuss the (strong) maximum principle for harmonic functions. Some authors may call following the strong maximum principle though since we dont need the "weak" maximum principle, we will simply refer to the following as the maximum principle.
The following can be found in \cite{arnold}
\begin{lemma}[The Maximum Principle]\label{MaximumPrincipleDef}
  Let $X$ be a Riemann Surface, let $U$ be a domain on $X$ and $\phi$ be a real-valued harmonic function on $U$. Then $\phi$ has no extrema in $U$. In other words, if for some $w \in U$ we have that $\phi(w) = max\{\phi(z) \colon \forall z \in U\}$ then $\phi$ must be a constant.
\end{lemma}
\begin{proof}
  Let $w_0$ be the point at which the maximum of $\phi$ occurs in $U$. Consider a small circular neighbourhood of $w_0$, with $w_0$ at the centre with the closed neighbourhood bounded by a circle called $\gamma$. We pick a point $w_1$ on the circle $\gamma$ and suppose that $\phi(w_0)>\phi(w_1)$. Let us connect $w_0$ and $w_1$ with a straight line $\delta$ so that $\delta:[0,1] \rightarrow U$ with $\delta(0)=w_0$ and $\delta(1)=w_1$.
  
  Since $\phi$ is continuous, that implies that for all $t > 0$, we must have that $\phi(\delta(0)) > \phi(\delta(t))$. But this violates lemma \ref{MVT}, the mean value theorem for harmonic functions, for if this is the case, then we would have 
  \[ \phi(w_0) > \frac{1}{2\pi}\int_0^{2\pi}\phi(\theta)d\theta \]
  This clearly cannot be so we must make it so that $\phi(w_0) = \phi(\delta(t))$ for all $t > 0$. So $\phi$ is constant in this small neighbourhood, which is closed. However, for any point in this neighbourhood, there is an open ball contained in $U$ surrounding it. Hence this neighbourhood is also open. This leads to the conclusion that the set of points maximising $\phi$ must either be empty, or must be the entire componant of the domain containing $w_0$. Hence, if $\phi$ has a maximum in $U$, $\phi$ must be constant across the whole of $U$.
\end{proof}

So we see that harmonic functions are indeed quite special. We now move on to our next topics of interest.

\subsection{Hilbert Spaces and the space $\mathcal{H}(X)$}

Hilbert spaces are in some ways, the most familiar class of space that one may study. They crop up everywhere in Mathematics and more and more in Theoretical Physics and indeed they play a major part in the proof of the Uniformisation theorem. Specifially, we will want to construct a particular Hilbert space that will contain a function that we will use to prove the Uniformisation theorem. Let us provide some definitions first.

\begin{defn}[Inner Product Space]
  Let $V$ be a vector space over a field $\mathbb{F}$ and $\langle \cdot,\cdot \rangle \colon V \times V \rightarrow \mathbb{F}$ be a positive definite sesquilinear form on $V$. Then the pair $\{V_{\mathbb{F}}, \langle \cdot,\cdot \rangle\}$ (frequently abbreviated as just $V$ or $V_{\mathbb{F}}$ if the underlying field is known) is called an Inner Product Space or Pre-Hilbert Space. Furthermore, such spaces have a norm associated with them, defined as a function $||\cdot|| \colon V \rightarrow \mathbb{F}$ which can be computed by $||v|| = \sqrt{\langle v,v \rangle}$ for all $v \in V$. 
\end{defn}

We note that inner product spaces are in fact metric spaces. We note that we can define on any inner product space a metric as follows.
\begin{prop}
  Let $V$ be an inner product space. Then $V$ is a metric space.
\end{prop}
\begin{proof}
  Let $\{V_{\mathbb{F}},\langle \cdot,\cdot \rangle\}$ be an inner product space. Pick two elements of $a,v \in V$. We define the metric $d:V\times V \rightarrow \mathbb{F}$ with the formula $d(a,b)=||a-b||$. By the properties of the inner product, we see that $d(a,b)$ is symmetric, and positive definite. The triangle inequality follows by applying the Cauchy-Schwartz inequality. 
\end{proof}
We can now define a Hilbert space.
\begin{defn}[Hilbert Space]
  An inner product space $\mathcal{H}$ is called a Hilbert Space if it is a complete metric space with a norm induced by the inner product.
\end{defn}

We see that Hilbert spaces are very similar to just normal inner product spaces and as such a lot of our examples are familiar vector spaces. Examples include
\begin{example}
  \begin{itemize}
    \item $\mathbb{R}^n$ with the usual Euclidean inner product.
  \end{itemize}
\end{example}

However, the fact Hilbert spaces are complete, allows us to find the limits of all Cauchy sequences. This completeness property allows us to generalise our notions of calculus as we have well defined limits.

A further important reason to want to study Hilbert Spaces is the Riesz Representation Theorem; a theorem that allows us, under sufficiently nice conditions that we will satisfy, to represent a bounded linear functional acting on an element of a Hilbert space, as an inner product of elements. Let us write this more precisely .
\begin{thm}[The Riesz Representation Theorem]\label{RieszRepTheorem}
  Let $H$ be a real Hilbert Space, and let $\sigma \colon H \rightarrow \mathbb{R}$ be a bounded linear functional, so there exists a constant $C$ such that $|\sigma(x)| \leq C||x||$ for all $x \in H$. Then there exists a $z \in H$ such that \[ \sigma(x) = \langle z, x \rangle \]
\end{thm}
A proof of this theorem can be found in chapter ??? of ???

We are now in a position to introduce a space of critical importance to our study of the laplacian on a Riemann surface. This space will be a Hilbert space, though we prove this in steps, first showing it is a vector space, then that given a particular norm, it is an inner product space, and then show that it is complete. We call this space $\mathcal{H}(X)$ and we shall explicitly construct it below. 

Note that henceforth we consider only connected Riemann surfaces to alleviate  issues related to having multiple connected componants.

%\begin{defn}[The space $C^{\infty}(X)$]\label{infspace}
%  Let $X$ be a Riemann surface. We denote by 
%  $C^{\infty}(X)$ the space of all infinitely differentiable, real valued functions on $X$.
%\end{defn}

%This space is quite clearly a vector space of functions over the real numbers as with respect to the addition of functions, we have that the sum of any two smooth functions is again smooth. More precisely, we see that for all $f,g, h \in C^{\infty}(X)$ (ie, smooth functions) and $\lambda, \mu \in \mathbb{R}$, we have that $f + g = g + f$ and $f + (g + h) = (f +g) + h$, the zero function is smooth and of course, for each function $f$ there is a unique additive inverse, namely $-f$. Furthermore $\lambda(f + g) = \lambda f + \lambda g$ and $(\lambda + \mu)f = \lambda f + \mu f$.

%Whilst this space is quite clearly interesting in its own right, we wish to study a particular quotient of it. We do so by establishing the following equivalence relation.


\begin{lemma}
  Let $X$ be a connected Riemann Surface. We define a relation on $\Omega^0(X)$ (or on $\Omega^0_c(X)$ if $X$ is non-compact), whereby two functions are equivalent if they differ by a constant function. This relation, given the symbol $\sim$, defines an equivalence relation on $\Omega^0(X)$.
\end{lemma}

\begin{proof}
  We let $f,g,h \in \Omega^0(X)$ (or $\Omega^0_c(X)$ if $X$ is non-compact) be functions and let $a, b \in \mathbb{R}$ be arbitrary constants. 
  Since any function $f = f + 0$ then we have that $f \sim f$. 
  Further, we have that if $f \sim g$ impyling $f = g + a$, then we have that $g + (-a) = f$. Since $a \in \mathbb{R}$ then $-a \in \mathbb{R}$ and so we have $g \sim f$. 
  Finally, if $f \sim g$ and $g \sim h$ implying that $f = g + a$ and $g = h + b$, then we have that $f \sim h$ since $f = g + a = (h + b) + a = h + (b + a)$ and $b+a \in \mathbb{R}$. Hence, $\sim$ is indeed an equivalence relation on $\Omega^0(X)$. 
\end{proof}

With this equivalence relation we can now define the space of cosets of functions differing by a constant.

\begin{defn}[The space $\mathcal{H}(X)$]
  Let $X$ be a connected Riemann Surface. The space of cosets under the above defined equivalence relation $\sim$, is written as $\mathcal{H}(X) = \Omega^0(X)/\sim$. Each element of this space is a coset of smooth functions on the Riemann Surface $X$, whose elements all differ from the coset representative by a constant. 
  Similarly, if $X$ is non-compact, then we define $\mathcal{H}(X) = \Omega^0_c(X)/\sim$, whose elements all have compact support in $X$.
\end{defn}

Note that in \cite{donaldson}, this space is called both $H$ and $C^{\infty}(X)/\mathbb{R}$, with the former occuring when $X$ is non compact and the latter in the compact case. We will however, distinguish the two cases by identifying $X$ as being compact and non compact, if necessary. Furthermore, the notation $C^{\infty}(X)/\mathbb{R}$, comes from $C^{\infty}(X)$ meaning the space of smooth functions on $X$ in analysis. 

Interestingly, $\mathcal{H}(X)$ is also a vector space, and this follows from the fact that $\Omega^0(X)$ is itself a vector space. Though this new space is more somehow more complicated than the space of smooth functions, there is an advantage to working with these cosets; namely that we can define a particular inner product and associated norm which will be of particular interest to our study of the laplacian on Riemann surfaces. This inner product, called the Dirichlet Inner Product, is defined as follows.

\begin{defn}[The Dirichlet Inner Product]\label{dInnerProduct}
  Let $X$ be a connected Riemann Surface and let $f, g \in \mathcal{H}(X)$, where at least one of $f$ or $g$ have compact support within $X$. Then we define the Dirichlet Inner Product $\langle \cdot, \cdot \rangle_D : \mathcal{H}(X) \times \mathcal{H}(X) \rightarrow \mathbb{R}$ as 
  \[
    \langle f, g \rangle_D =\langle df, dg \rangle = 2i \int_X \partial f \wedge \bar{\partial} g 
  \]
  where the inner product on the right hand side is defined in the previous section as definition \ref{InnerProduct}. Notice that this is a real valued inner product.
  This inner product of course, also defines a norm in the usual way:
  \begin{align*}
    ||f||_D^2 = \langle f, f \rangle_D
  \end{align*}
\end{defn}

If the functions were to not have compact support within this region then, the value of the inner product and norm might be $+\infty$, so by considering only functions with suitable compact support, we get that we have a bona-fide inner product and norm.

So with all of this theoretical set up, it would be nice if we could somehow get back our claimed main object of study for this section; the laplacian. Although hidden, the Dirichlet norm (and the associated inner product) in fact return to us, the laplacian.

\begin{prop}\label{InnerLaplacian}
  Given a connected Riemann surface $X$, if at least one function $f,g$ is in $\mathcal{H}(X)$ (hence having compact support on $X$), then 
  \begin{align*}
    \langle f, g \rangle_D = \int_X f \Delta g = \int_X g \Delta f
  \end{align*}
\end{prop}
\begin{proof}
  First we consider two identities:
  \begin{itemize}
    \item $\bar{\partial}\wedge(f\bar{\partial}g) = f (\bar{\partial}\wedge\bar{\partial} g) + \bar{\partial}f \wedge \bar{\partial}g = 0 + 0 = 0$
    \item $\partial\wedge(f\bar{\partial}g)=\partial f \wedge \bar{\partial}g + f(\partial\wedge\bar{\partial}g)=\partial f \wedge \bar{\partial}g - \frac{1}{2i}f\Delta g$
  \end{itemize}
  Now let us expand the inner product. By the first identity above, we have that $f\bar{\partial}g$ is holomorphic, so that $d = \partial$. Hence, we can apply Stokes theorem. Note that since $X$ has no boundary, $\partial X = \emptyset$ and hence, that integral vanishes.
  \begin{align*}
    \langle f, g \rangle_D &= 2i\int_X \partial f \wedge \bar{\partial}g = 2i\int_X \partial(f\bar{\partial}g)+\int_X f\Delta g \\
    &= 2i\int_X \partial f \wedge \bar{\partial}g = 2i\int_{\partial X} f\bar{\partial}g + \int_X f\Delta g \\
    &= \int_X f\Delta g
  \end{align*}
  Since the Dirichlet inner product is a real inner product, it is symmetric and hence we have that $\langle f, g \rangle_D = \langle g, f \rangle_D $ and so $\int_X f\Delta g = \int_X g \Delta f$ as claimed.
\end{proof}

\section{The Dirichlet Energy Functional}
The Dirichlet Energy Functional is an interesting object of study in its own right. 

We begin by defining the $\hat{\rho}$ functional, a simple integral functional from $\mathcal{H}(X)$ to $\mathbb{R}$.
\begin{defn}
  Let $X$ be a connected Riemann Surface and $\rho \in \Omega^2_c(X)$ be a $2$-form of compact support in $X$ such that $\int_X \rho = 0$. Then, we define the functional $\hat{\rho} \colon \mathcal{H}(X) \rightarrow \mathbb{R}$ to be defined as, for some $f$ of compact support in $\mathcal{H}(X)$,
  \begin{align*}
    \hat{\rho}(f) = \int_X f\rho
  \end{align*}
\end{defn}

Since $\int_X \rho = 0$ then we have that this functional is linear on functions of compact support in $\mathcal{H}(X)$. Note that, if $X$ is a compact Riemann surface, then all functions and $2$-forms  onautomatically have compact support on $X$, so the compact support conditions really only apply to the cases of non-compact Riemann surfaces. In those cases, we demand compact support so that the our functional doesnt go to inifinity. For example, if we let $X=\mathbb{C}$ and let $\rho = sin(z)dz\wedge d\bar{z}$ and let $f(z) = z$, then we have that $\int_{\mathbb{C}}zsin(z)dz\wedge d\bar{z} = \infty$ with neither $f$ or $\rho$ being of compact support even though $\int_X \rho = 0$
%SHOULD I INCLUDE THIS EXAMPLE? FLESH IT OUT? IS IT EVEN RIGHT?
%supp(f) = \mathbb{C}\setminus \{0\} and supp(\rho)=\mathbb{C}\setminus \{nz| \forall z \in \mathbb{Z}\}

We now prove the claim that $\hat{\rho}$ is a linear functional on $\mathcal{H}(X)$.
\begin{lemma}[$\hat{\rho}$ is a linear functional]\label{rhohatlinear}
  Let $X$ be a connected Riemann surface. Then $\hat{\rho}$ is a linear functional, ie $\hat{\rho}(\lambda f+ \mu g) = \lambda \hat{\rho}(f) + \mu \hat{\rho}(g)$
\end{lemma}
Recall that elements of $\mathcal{H}(X)$ are equivalence classes of smooth functions on $X$ (with compact support on $X$ if $X$ is non-compact), which differ by a constant.
\begin{proof}
  Let $\tilde{f},\tilde{g} \in \mathcal{H}$ and $\lambda,\mu \in \mathbb{R}$. Then for some real constant, $a,b \in \mathbb{R}$ we have that $\tilde{f}(z) = f(z) + a$ and $\tilde{g}(z) = g(z) + b$. Recall also that $\int_X \rho = 0$.
  Hence, we get that:
  \begin{align*}
    \hat{\rho}(\lambda \tilde{f}+ \mu \tilde{g}) &= \hat{\rho}(\lambda(f + a) + \mu(g + b)) \\
    &=\int_X (\lambda(f + a) + \mu(g + b))\rho \\
    &=\lambda \int_X (f + a)\rho + \mu \int_X (g + b)\rho \\
    &=\lambda \int_x f\rho + \lambda a\int_X \rho + \mu \int_X g\rho + \mu b\int_X \rho \\
    &=\lambda \hat{\rho}(f) + \mu \hat{\rho}(g)\\
    &=\lambda \hat{\rho}(\tilde{f}) + \mu \hat{\rho}(\tilde{g})
  \end{align*}
  with the last line following since clearly $f \in \tilde{f}$ and $g \in \tilde{g}$.
\end{proof}

We can now define a linear functional of great importance.

\begin{defn}[Dirichlet Energy Functional]
  Let $X$ be a connected Riemann Surface, and suppose $f \in \mathcal{H}(X)$. Suppose further that we have $2$-form $\rho \in \Omega^2 (X)$ such that $\int_X \rho = 0$. Then we can define a functional, $\mathcal{L}: 
  \mathcal{H} \rightarrow \mathbb{R}$ such that $\mathcal{L}(f) = -2\hat{\rho}(f)+||f||^2_D$. This $\mathcal{L}$ is called the Dirichlet energy functional, or sometimes Dirichlet energy for short.
\end{defn}

The case of $X$ being non-compact requires special treatment as we have seen in the preceeding definitions. In these cases, we consider functions $f$ that are compactly supported on $X$ as this guarantees that $f$ goes to zero as $||f||_D$ goes to $+\infty$. This condition, along with considering that a non-compact Riemann surface is simply connected, will be sufficient for our discussion of the Dirichlet functional on non-compact manifolds.

Returning to the Dirichlet functional, we see that this functional is an example of a Lagrangian; physically, a quantity that encapsulates the difference of the kinetic and potential energy of a system. Often in physics, one writes $L=T-V$ (where $T$ is the symbol for kinetic energy and $V$ is the symbol for potential energy) to mean the lagrangian of a physical system. If we take a moment to look at our Dirichlet energy, we can take our kinetic energy term to be $\hat{\rho}f$, and the potential energy term to be $||f||^2_D$. But what is this the kinetic or potential energy of? In Chapter 9 of \cite{donaldson}, the conclusions of our current argument are compared to those of the solutions of the steady state heat equation, meaning we are looking for solutions to the heat equation which are constant in time over a Riemann surface. Though I find a more interesting, and perhaps easier to understand interpretation can be had by considering electric fields. Suppose for a moment that we want to apply an electric charge to the surface of our connected Riemann surface. What stable (non-changing) distributions of this charge are allowed on our Riemann surface? Naturally, the electric field and electric potential will want to be minimal, as nature tends to want to balance out forces, and spend as little energy as possible doing so. This leads us to look for functions in $\mathcal{H}(X)$ that minimise $\mathcal{L}$ as much as possible. But how do we find such solutions? Do they even exist? To answer this, we appeal to the calculus of variations. We wish to apply the variational principle, a method for finding (to the first order) functions which minimise $\mathcal{L}$. To prove the existance of such minimising functions however, we need to show that $\mathcal{L}$ is a bounded functional, bounded from below. This will allow us to apply the variational principle to find which functions minimise $\mathcal{L}$.

Let us, for a moment, suppose such functions exist and that $\mathcal{L}$ is a bounded linear functional. What are the functions that minimise our problem?
We begin by computing the first variation of $\mathcal{L}$. To do this we let $\epsilon > 0$ and let $g$ be a function on $X$. The conditions on $g$ will be more precisely defined later, so for now, we assume such a $g$ exists. Let us now compute the first variation of $\mathcal{L}$:
\begin{gather*}
  \frac{d}{d\epsilon}\biggr\rvert_{\epsilon = 0}\mathcal{L}(f+\epsilon g) = 0 \\
  \frac{d}{d\epsilon}\biggr\rvert_{\epsilon = 0}(-2\hat{\rho}(f+\epsilon g) + ||f+\epsilon g||_D^2) = 0 \\
  -2\frac{d}{d\epsilon}\biggr\rvert_{\epsilon = 0}(\int_X f\rho  + \epsilon \int_X g\rho) + \frac{d}{d\epsilon}\biggr\rvert_{\epsilon = 0} \langle f + \epsilon g, f + \epsilon g \rangle = 0 \\
  -2\hat{\rho}(g)+\frac{d}{d\epsilon}\biggr\rvert_{\epsilon = 0}(\langle f, f \rangle + 2\epsilon \langle f, g \rangle +\epsilon^2 \langle g, g \rangle) = 0 \\
  -2\hat{\rho}(g) + 2\langle f, g \rangle = 0 \\
  \int_X g\Delta f - \int_X g\rho = 0 \\
  \int_X g(\Delta f - \rho) = 0 
\end{gather*}
The appearence of the Laplacian comes from proposition \ref*{InnerLaplacian}.
Since we have that at least one of $f,g$ and $\rho,g$ have compact support (and in this case both $f$ and $\rho$ are compactly supported on $X$), then for this integral to be zero, we need a crucial condition to hold. Namely the following statement:

There exists a unique solution $f$, up to addition by a constant, of the equation 
\[ \Delta f = \rho \Longleftrightarrow \int_X \rho = 0\]

This equation, known as Poissons' equation, will be our main tool in solving the uniformisation theorem. As we shall shortly see, we only need to consider its solutions on a small number of Riemann surfaces to allow us to draw the conclusions we seek. We begin by solving Poissons equation on compact, connected Riemann surfaces.

\section{Solving the Laplace equation on compact \\Riemann surfaces}
\section{Weyl's Lemma}
\section{The Uniformisation Theorem}
\section{The Laplace equation on simply connected non compact Riemann surfaces}
\section{Consequences of solving the Laplace equation on Riemann surfaces and the Riemann-Roch Theorem}
%
%\references{Bibliography}{}{yes}
\begin{thebibliography}{99}
\bibitem{donaldson} S. Donaldson, {\em Riemann Surfaces}
\bibitem{arnold} V. I. Arnol'd, {\em Lectures on Partial Differential Equations}
\bibitem{rudin} W. Rudin, {\em Functional Analysis}
\bibitem{comfun} Jones \& Singerman. {\em Complex Functions}, Cambridge University Press (DATE).
\bibitem{calcohomo} I. Madsen \& J. Tornehave, {\em From Calculus to Cohomology}, Cambridge University Press (1997)
\bibitem{spivak} M.Spivak, {\em Calculus on Manifolds}, 

%
\bibitem{website} MacTutor History of Mathematics Archive, at https://mathshistory.st-andrews.ac.uk/ [accessed 9 May 2030]
\end{thebibliography}
\end{document}
